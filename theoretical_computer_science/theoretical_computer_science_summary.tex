% Configuration
\documentclass[a4paper, 10pt]{article}

% Formatting
\usepackage[landscape, left=0.75cm, top=1.0cm, right=0.75cm, bottom=1.5cm, footskip=15pt]{geometry}
\setlength{\columnsep}{0.5cm}
\usepackage{flowfram}
\ffvadjustfalse
\Ncolumn{3}
\usepackage[compact]{titlesec}

% ------------------------
% Imports and commands
% ------------------------

% Language stuff
\usepackage[german]{babel}
\usepackage[utf8]{inputenc}

% Math stuff
\usepackage{amsthm}
\usepackage{amssymb}
\usepackage{amsmath}
\usepackage{mathtools}
\usepackage{bm}

\newtheorem*{corollary}{Cor}

% Enumerated math stuff
\theoremstyle{definition}
\newtheorem{definition}{Def}[section]
\newtheorem{lemma}{Lemma}[section]
\newtheorem{theorem}{Thm}[section]
\newtheorem{note}{Bmk}[section]

% Math stuff without enumeration
\newtheorem*{definition*}{Def}
\newtheorem*{example}{Bsp}

% Miscellaneous
\usepackage{hyperref}
\hypersetup{colorlinks=true, urlcolor=blue, linkcolor=blue, citecolor=blue}

\usepackage{enumitem}
\setitemize{itemsep=0.5pt, topsep=0pt}
\setenumerate{itemsep=0.75pt, topsep=0pt}

% Custom commands
\newcommand{\N}{\mathbb{N}}
\newcommand{\R}{\mathbb{R}}
\newcommand{\words}{\Sigma^*}
\newcommand{\A}{\Sigma}
\newcommand{\Kon}{\text{Kon}}
\newcommand{\op}{\mathcal{U}}
\newcommand{\cost}{\text{cost}}
\newcommand{\opm}{\mathcal{M}}
\newcommand{\goal}{\text{goal}}

% Metadata
\title{Theoretische Informatik}
\author{Nicola Studer \\ \href{mailto:nicstuder@student.ethz.ch}{nicstuder@student.ethz.ch}}
\date{\today}


% ------------------------
% Document
% ------------------------

\begin{document}
\maketitle

\section{Gruppentheorie}
\begin{definition}[Monoid]
    \(\langle M; *, e\rangle\) mit \(*\) assoziativ und \(e\) als neutrales Element.
\end{definition}

\begin{definition}[Gruppe]
    \(\langle G; *, ^, e\rangle\) mit \(*\) assoziativ, \(e\) als neutrales Element und jedes element \(x \in G\) hat Inverses \(\hat{x}\).
\end{definition}

\section{Alphabete, Wörter, Sprachen und die Darstellung von Problemen}

\begin{definition}
    Eine endliche nichtleere Menge \(\A\) heisst \textbf{Alphabet}. Die Elemente eines Alphabets heissen \textbf{Buchstaben} (\textbf{Zeichen}, \textbf{Symbole}).
\end{definition}

\begin{example}
    \(\A_{\text{bool}} = \{0, 1\}\), \(\A_{\text{lat}} = \{a, b, c, \ldots, z\}\), \\
    \(\A_{\text{Keyboard}} = \A_{\text{lat}} \cup \{A, B, \ldots, Z, \text{\textvisiblespace}, >, <, (, ), \ldots, !\}\), \\
    \(\A_{\text{logic}} = \{0, 1, x, (, ), \land, \lor, \lnot \}\)
\end{example}

\begin{definition}
    Ein \textbf{Wort} über \(\A\) ist eine endliche (eventuell leere) Folge von Buchstaben aus \(\A\). Das \textbf{leere Wort} \(\lambda\) ist die leere Buchstabenfolge. \(|\lambda| = 0\). \\
    \(\bm{\words}\) ist die Menge aller Wörter über \(\A\), \(\A^+ = \words - \{\lambda\}\). \\
    \(\bm{\A^i} = \{x \in \words \ | \ |x| = i\}\)
\end{definition}

\begin{definition}[Verkettung]
    \(\Kon: \words \times \words \to \words\).
    \[\Kon(x, y) = x \cdot y = xy\]
\end{definition}

\begin{note}
\((\words, \Kon)\) ist Monoid mit neutralem Element $\lambda$.
\end{note}

\begin{note}
    \(\forall x,y \in \words: |xy| = |x \cdot y| = |x| + |y|\).
\end{note}

\begin{definition}[Umkehrung]
    \(a^\text{R} = a_na_{n-1} \ldots a_1\)
\end{definition}

\begin{definition}
    \(i\)-te Iteration \(x^i\) von \(x \in \A\) wird definiert als \(x^0 = \lambda\), \(x^1 = x\) und \(x^i = xx^{i-1}\).
\end{definition}

\begin{definition}\(\;\)
    \begin{itemize}
        \item \(v\) heisst \textbf{Teilwort} von \(w \iff \exists x,y \in \words: w = xvy\).
        \item \(v\) heisst \textbf{Präfix} von \(w \iff \exists y \in \words: w = vy\).
        \item \(v\) heisst \textbf{Suffix} von \(w \iff \exists x \in \words: w = xv\).
        \item \(v \neq \lambda\) heisst \textbf{echtes Teilwort} (Präfix, Suffix) von \(w\) genau dann, wenn \(v \neq w\) und \(v\) ein Teilwort (Präfix, Suffix) von \(w\) ist.
    \end{itemize}
\end{definition}

\begin{definition}
    \(|x|_a\) Anzahl der Vorkommen von \(a\) in \(x\).
\end{definition}

\begin{definition}
    \(\A = \{s_1, s_2, \ldots, s_m\}\), \(m \geq 1\) mit \(s_1 < s_2 \ldots < s_m\) als Ordnung auf \(\A\). Wir definieren die \textbf{kanonische Ordnung} auf \(\words\) für \(u, v \in \words\):
    \[u < v \iff |u| < |v| \lor |u| = |v| \land u = x \cdot s_i \cdot u' \land v = x \cdot s_j \cdot v'\] für irgendwelche \(x, u', v' \in \words\) und \(s_i < s_j\).
\end{definition}

\begin{definition}[Sprache]
    \(L \subseteq \words\) mit Komplement \(L^\text{C} = \words - L\).
    \begin{itemize}
        \item \(\bm{L_\emptyset} = \emptyset\) ist die \textbf{leere Sprache}
        \item \(\bm{L_1 \cdot L_2 = L_1 L_2} = \{vw \ | \ v \in L_1 \ \text{und} \ w \in L_2\}\)
        \item \(\bm{L^0} := L_\lambda\) und \(\bm{L^{i + 1}} = L^i \cdot L\) für alle \(i \in \N\)
        \item \(\bm{L^*} = \bigcup\limits_{i \in \N} L^i\) und \(\bm{L^+} = \bigcup\limits_{i \in \N - \{0\}} L^i = L \cdot L^*\)
    \end{itemize}
\end{definition}

\begin{lemma}
    \(L_1 L_2 \cup L_1 L_3 = L_1 (L_2 \cup L_3)\)
\end{lemma}

\begin{lemma}
    \(L_1 (L_2 \cap L_3) \subseteq L_1 L_2 \cap L_1 L_3\)
\end{lemma}

\begin{lemma}
    Es existieren \(U_1 , U_2, U_3 \in (\A_{\text{bool}})^*\), so dass \(U_1(U_2 \cap U_3) \subsetneq U_1 U_2 \cap U_1 U_3\)
\end{lemma}

\begin{definition}
    \(\A_1, \A_2\) zwei beliebige Alphabete. Homomorphismus: \(h : \A_1^* \to \A_2^*\) mit:
    \begin{enumerate}
        \item \(h(\lambda) = \lambda\)
        \item \(h(uv) = h(u) \cdot h(v)\) für alle \(u, v \in \words_1\).
    \end{enumerate} 
\end{definition}

\begin{definition}
    Das \textbf{Entscheidungsproblem} (\(\bm{\A, L}\)) ist, für jedes \(x \in \words\) zu entscheiden, ob \(x \in L\) oder \(x \not\in L\).

    Ein Algorithmus \(A\) löst das Entscheidungsproblem \\
    (\textbf{erkennt} \(L\)), falls \(\forall x \in \words\):
    \[A(x) = \begin{cases}
        1, & \text{falls} \ x \in L \\
        0, & \text{falls} \ x \not\in L
    \end{cases}\]
\end{definition}

\begin{definition}
    \(\A, \Gamma\) zwei Alphabete. Algorithmus \(A\) \textbf{berechnet (realisiert)} eine \textbf{Funktion (Transformation)} \(\bm{f: \words \to \Gamma^*}\) falls \(\forall x \in \words: A(x) = f(x)\).
\end{definition}

\begin{definition}
    \(\A, \Gamma\) zwei Alphabete, \(R \subseteq \words \times \Gamma^*\) eine Relation. Ein Algorithmus \(\bm{A}\) \textbf{berechnet} \(\bm{R}\) (oder löst das Relationsproblem \(R\)), falls für alle \(x \in \words\), für das ein \(y \in \Gamma^*\) nut \((x, y) \in R\) exisitert gilt: \((x, A(x)) \in R\).
\end{definition}

\begin{definition}[Optimierungsproblem]
    Ist ein 6-Tupel \\
    \(\op = (\A_1, \A_0, L, \opm, \cost, \goal)\), wobei:
    \begin{enumerate}
        \item \(\A_1\) ist das Eingabealphabet
        \item \(\A_2\) ist das Ausgabealphabet
        \item \(L \subseteq \words\) ist die Sprache der \textbf{zulässigen Eingaben}. \(x \in L\) ist ein \textbf{Problemfall (Instanz) von} \(\bm{\op}\).
        \item \(\opm : L \to \mathcal{P}(\words_0)\) und für jedes \(x \in L\) ist \(\opm(x)\) die \textbf{Menge der zulässigen Lösungen für} \(\bm{x}\).
        \item \textbf{Kostenfunktion} \(\cost: \bigcup_{x \in L}(\opm(x) \times \{x\}) \to \R^+\)
        \item \textbf{Optimierungsziel} \(goal \in \{Minimum, Maximum\}\) 
    \end{enumerate}
\end{definition}

\begin{definition*}
    Eine zulässige Lösung \(\alpha \in \opm(x)\) heisst \textbf{optimal} für den Problemfall \(x\), falls \[\cost(\alpha, x) = \textbf{Opt}_\op(x) = \goal\{\cost(\beta, x) \ | \ \beta \in \opm(x)\}\]
\end{definition*}
\begin{definition*}
    Ein Algorithmus \(A\) \textbf{löst} \(\op\), falls für jedes \(x \in L\)
    \begin{enumerate}
        \item \(A(x) \in \opm(x)\)
        \item \(\cost(A(x), x) = \goal\{\cost(\beta, x) \ | \ \beta \in \opm(x)\}\)
    \end{enumerate}
\end{definition*}

\begin{definition}
    Algorithmus \(A\) \textbf{generiert} das Wort \(x\), falls \(A\) für die Eingabe \(\lambda\) die Ausgabe \(x\) liefert.
\end{definition}

\begin{definition}
    \(A\) ist ein \textbf{Auszählungsalgorithmus für} \(L\), falls \(A\) für jede Eingabe \(n \in \N\) die Wortfolge \(x_1, x_2, \ldots, x_n\) ausgibt, wobei das die kanonisch \(n\) erster Wörter in \(L\) sind.
\end{definition}

\subsection*{Shannon Entropie}
Claude Shannon versuche den Informationsgehalt von String herauszufinden, hat es aber nicht geschafft Positionen darzustellen.

Betrachte das Alphabet \(\A = \{a, b, c, d\}\) und das Wort mit Vorkommen \(a \times 16, b \times 8, c \times 4, d \times 4\). Seine Codierung ist \(a \mapsto 0, b \mapsto 10, c \mapsto 110, d \mapsto 111\). Man erkennt das nur die Präfixe erweitert werden.

Die Codierung hat nun die Folgende Länge mit \(|w|\) als Wortlänge, \(H_a\) Häufigkeit von \(a\), \(h_a\) relative Häufigkeit und \(p_a\) die Wahrscheinlichkeit:
\begin{align*}
    56 &= 16 \cdot 1 + 8 \cdot 2 + 4 \cdot 3 + 4 \cdot 3 \\
    &= H_a \log_2\frac{|w|}{H_a} +  H_b \log_2\frac{|w|}{H_b} + \ldots \\
    &= H_a \cdot (- \log_2 \frac{H_a}{|w|}) + \ldots \\
    &= H_a \cdot (- \log_2 h_a) + \ldots \\
    &= H_a \cdot (- \log_2 p_a) + \ldots \\
    &= - \sum_{a \in \A} H_a \cdot \log_2 p_a
\end{align*}

\subsection*{Kolmogorov-Komplexität}
\begin{definition}
    Für jedes Wort \(x \in \{0, 1\}^*\) ist die \(K(x)\) des Wortes das Minimum der binären Länge der Pascal-Programme, die \(x\) generieren.
\end{definition}

\begin{lemma}
    \(\exists d \in \N \ \forall x \in \{0, 1\}^* : K(x) \leq |x| + d\)
\end{lemma}

\begin{definition}
    \(\forall n \in \N: K(n) = K(\text{Bin}(n))\)
\end{definition}

\begin{lemma}
    \(\forall n \in \N_1 \ \exists w_n \in \{0, 1\}^n: K(w_n) \geq |w| = n\)
\end{lemma}

\begin{theorem}
    Sei \(A\) und \(B\) Programmiersprachen. Es existiert eine Konstante \(c_{A, B}\) die nur von \(A\) und \(B\) abhängt, so dass \(\forall x \in \{0, 1\}^*: |K_A(x) - K_B(x)| \leq c_{A, B}\)
\end{theorem}

\begin{definition}
    \(x \in \{0, 1\}^*\) heisst zufällig \(\iff K(x) \geq |x|\).
    
    Eine Zahl \(n\) heisst zufällig \(\iff K(n) = K(\text{Bin}(n)) \geq \lceil \log_2(n + 1)\rceil - 1\). Die \(-1\) kommt von der führenden \(1\), welche bei jeder Binären Zahl führend ist.
\end{definition}

\begin{theorem}
    \(L \subseteq \{0, 1\}^*\), sei \(z_n\) das kanonisch \(n\)-te Wort in \(L\).  Wenn ein Programm \(A_L\) existiert, das das \(\{0, 1\}^*, L\) löst, dann gilt für alle \(n \in \N - \{0\}\): \(K(z_n) \leq \lceil \log_2(n + 1) \rceil + c\), wobei \(c\) eine von \(n\) unabhängige Konstante ist.
\end{theorem}

\begin{theorem}[Primzahlsatz]
    \[\lim_{n\to\infty} \frac{\text{Prim}(n)}{n / \ln n} = 1\]
\end{theorem}

\begin{lemma}
    Sei \(n_1, n_2, n_3, \ldots\) eine steigende unendliche folge natürlicher Zahlen mit \(K(n_i) \geq \lceil \log_2 n_i \rceil / 2\). Für jedes \(i \in \N - \{0\}\) sei \(q_i\) die grösste Primzahl, die die Zahl \(n_i\) teilt. Dann ist die Menge \(Q = \{q_i \ | \ i \in \N - \{0\}\}\) unendlich.
\end{lemma}

\begin{theorem}
    Für unendlich viele \(k \in \N\) gilt
    \[\text{Prim(k)} \geq \frac{k}{2^{17}} \log_2 k \cdot (\log_2\log_2k)^2\]
\end{theorem}

\section{Endliche Automaten}
\begin{definition}[\textit{deterministischer} endlicher Automat] \(\;\) \\
    \(M = (q, \A, \delta, q_0, F)\) mit:
    \begin{enumerate}
        \item \(Q\): endliche Menge von Zuständen
        \item \(\A\): Eingabealphabet
        \item \(q_0 \in Q\): der Anfangszustand
        \item \(F \subseteq Q\): Menge der akzeptierenden Zustände
        \item \(\delta: Q \times \A \to Q\): Übergangsfunktion
    \end{enumerate}
\end{definition}

\begin{definition*}[Weitere Terminologie] \(\;\)
    \begin{enumerate}
        \item \textbf{Konfiguration} von \(M\) ist \(x \in Q \times \words\).
        \item \textbf{Startkonfiguration} von \(M\) auf \(x\) ist \((q_0, x) \in \{q_0\}\)
        \item \textbf{Endkonfiguration}: Jede Konfiguration von \(Q \times \{\lambda\}\)
        \item \textbf{Schritt} von \(M\): Relation \(\vdash_M \subseteq (Q \times \words) \times (Q \times \words)\) definiert durch: \\ \((q, w) \vdash_M (p, x) \iff w = ax, a \in \A\) und \(\delta(q, a) = p\)
        \item \textbf{Berechnung} \(C\) von \(M\): Folge Folge von Konfigurationen \(C = C_0, \ldots, C_n\) so dass \(\forall 0 \leq i < n: C_i \vdash_M C_{i+1}\)
        \item \textbf{Berechnung von} \(\bm{M}\) \textbf{auf Eingabe} \(\bm{x \in \words}\) : Falls \(C_0 = (q_0, x)\) und \(C_n \in Q \times \{\lambda\}\)
        \item \textbf{Akzeptierende Berechnung von \(M\) auf \(x\)}: Falls \(C_n \in F \times \{\lambda\}\) und \(M\) das Wort \(x\) akzeptiert.
        \item \textbf{Verwerfende Berechnung von \(M\) auf \(x\)}: Falls \\ \(C_n \in (Q - F) \times \{\lambda\}\) und \(M\) das Wort \(x\) verwirft. 
        \item \textbf{Die von \(M\) akzeptierte Sprache \(L(M) =\)}
        \begin{align*}
            \{w \in \words \ | \ & \text{Die Berechnung von \(M\) auf \(w\) endet in}\\
            & \text{einer Endkonfiguration \((q, \lambda)\) mit \(q \in F\)}
        \end{align*}
        \item \textbf{Klasse der regulären (akzeptierten) Sprachen}: \\ \(\mathcal{L}_{\text{EA}} = \{L(M) \ | \ \text{\(M\) ist ein EA}\}\)
    \end{enumerate}
\end{definition*}

\begin{definition}
    \((q, w) \vdash_M^*(p, u) \iff (q = p \land w = u) \lor k \in \N_1:\)
    \begin{enumerate}[label=(\roman*)]
        \item \(w = a_1a_2 \ldots a_ku, a_i \in \A\) für \(i = 1, 2, \ldots, k\)
        \item \(\exists r_1, r_2, \ldots, r_{k-1} \in Q\), so dass \\
        \((q, w) \vdash_M (r_1, a_2\ldots a_ku) \ldots \vdash_M (r_{k-1}, a_ku) \vdash_M (p, u)\)
    \end{enumerate}
\end{definition}

\pagebreak
\begin{definition*}
    \(\hat{\delta}: Q \times \words \to Q\) durch:
    \begin{enumerate}[label=(\roman*)]
        \item \(\forall q \in Q: \hat{\delta}(q, \lambda) = q\)
        \item \(\forall a \in \A, w \in \words, q \in Q: \hat{\delta}(q, wq) = \delta(\hat{\delta}(q, w), a)\)
    \end{enumerate}
\end{definition*}

\begin{lemma}
    \(L(M) = \{w \in \{0, 1\}^* \ | \ |w|_0 + |w|_1 \equiv_2 0\)
\end{lemma}

\begin{definition*} \(\text{Kl}[p] =\) \\
    \(\{w \in \words \ | \ \hat{\delta}(q_0, w) = p\} = \{w \in \words \ | \ (q_0, w) \vdash_M^* (p, \lambda)\}\)
\end{definition*}

\begin{lemma}
    \(M_1 = (Q_1, \A, \delta_1, q_{01}, F_1)\), sowie auch \\ \(M_2 = (Q_2, \A, \delta_2, q_{02}, F_2)\) sind zwei EA, es gilt: \\ 
    \(\forall \odot \in \{\cup, \cap, -\} \ \exists M: L(M) = L(M_1) \odot L(M_2)\) \\ \ \\
    \textbf{Konstruktion von \(\bm{M}\)}: \(M = (Q , \A, \delta, q_0, F_{\odot})\)
    \begin{enumerate}[label=(\roman*)]
        \item \(Q = Q_1 \times Q_2\)
        \item \(q_0 = (q_{01}, q_{02})\)
        \item \(\forall q \in Q_1 p \in Q_2 a \in \A: \delta((q, p), a) = (\delta_1(q, a), \delta_2(p, a))\)
        \item \(\odot = \cup \implies F = F_1 \times Q_2 \cup Q_1 \times F_2\) \\
        \(\odot = \cap \implies F = F_1 \times F_2\) \\
        \(\odot = - \implies F = F_1 \times (Q_2 - F_2)\)
    \end{enumerate}
\end{lemma}

\end{document}
