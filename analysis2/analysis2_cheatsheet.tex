% Configuration
\documentclass[a4paper, 10pt]{article}

% Formatting
\usepackage[landscape, left=0.75cm, top=1.0cm, right=0.75cm, bottom=1.5cm, footskip=15pt]{geometry}
\setlength{\columnsep}{0.5cm}
\usepackage{flowfram}
\ffvadjustfalse
\Ncolumn{3}
\usepackage[compact]{titlesec}
\usepackage{parskip}
\setlength{\parskip}{2pt}

% ------------------------
% Imports and commands
% ------------------------

% Language stuff
\usepackage[english]{babel}
\usepackage[utf8]{inputenc}

% Math stuff
\usepackage{amsthm}
\usepackage{amssymb}
\usepackage{amsmath}
\usepackage{bm}
\usepackage{centernot}

\newtheorem*{corollary}{Cor}
\newtheorem*{lemma}{Lemma}
\newtheorem*{proposition}{Prop}

\theoremstyle{definition}
\newtheorem*{theorem}{Thm}
\newtheorem*{definition}{Def}

% Colored boxes
\usepackage{xcolor}
\usepackage{mdframed}
\usepackage{framed}
\mdfsetup{skipabove=2pt,skipbelow=2pt}

% Fix for MDFramed
\makeatletter
\DeclareDocumentCommand{\mdtheorem}{ O{} m o m o }%
 {\ifcsdef{#2}%
   {\mdf@PackageWarning{Environment #2 already exits\MessageBreak}}%
   {%
    \IfNoValueTF {#3}%
     {%#3 not given -- number relationship
      \IfNoValueTF {#5}%
        {%#3+#5 not given
        \@definecounter{#2}%
        \expandafter\xdef\csname the#2\endcsname{\@thmcounter{#2}}%
        \newenvironment{#2}[1][]{%
          \refstepcounter{#2}%
          \ifstrempty{##1}%
            {\let\@temptitle\relax}%
            {%
             \def\@temptitle{\mdf@theoremseparator%
                             \mdf@theoremspace%
                             \mdf@theoremtitlefont%
                             ##1}%
             \mdf@thm@caption{#2}{{#4}{\csname the#2\endcsname}{##1}}%
             }%
          \begin{mdframed}[#1,frametitle={\strut#4\ \csname the#2\endcsname%
                                          \@temptitle}]}%
          {\end{mdframed}}%
        \newenvironment{#2*}[1][]{%
          \ifstrempty{##1}{\let\@temptitle\relax}{\def\@temptitle{\mdf@theoremseparator \mdf@theoremspace ##1}}% <- the problem was here
          \begin{mdframed}[#1,frametitle={\strut#4\@temptitle}]}%
          {\end{mdframed}}%
        }%
        {%#5 given -- reset counter
        \@definecounter{#2}\@newctr{#2}[#5]%
        \expandafter\xdef\csname the#2\endcsname{\@thmcounter{#2}}%
        \expandafter\xdef\csname the#2\endcsname{%
               \expandafter\noexpand\csname the#5\endcsname \@thmcountersep%
                  \@thmcounter{#2}}%
        \newenvironment{#2}[1][]{%
          \refstepcounter{#2}%
          \ifstrempty{##1}%
            {\let\@temptitle\relax}%
            {%
             \def\@temptitle{\mdf@theoremseparator%
                             \mdf@theoremspace%
                             \mdf@theoremtitlefont%
                             ##1}%
             \mdf@thm@caption{#2}{{#4}{\csname the#2\endcsname}{##1}}%
             }
          \begin{mdframed}[#1,frametitle={\strut#4\ \csname the#2\endcsname%
                                          \@temptitle}]}%
          {\end{mdframed}}%
        \newenvironment{#2*}[1][]{%
          \ifstrempty{##1}%
            {\let\@temptitle\relax}%
            {%
             \def\@temptitle{\mdf@theoremseparator%
                             \mdf@theoremspace%
                             \mdf@theoremtitlefont%
                             ##1}%
             \mdf@thm@caption{#2}{{#4}{\csname the#2\endcsname}{##1}}%
             }%
          \begin{mdframed}[#1,frametitle={\strut#4\@temptitle}]}%
          {\end{mdframed}}%
        }%
     }%
     {%#3 given -- number relationship
        \global\@namedef{the#2}{\@nameuse{the#3}}%
        \newenvironment{#2}[1][]{%
          \refstepcounter{#3}%
          \ifstrempty{##1}%
            {\let\@temptitle\relax}%
            {%
             \def\@temptitle{\mdf@theoremseparator%
                             \mdf@theoremspace%
                             \mdf@theoremtitlefont%
                             ##1}%
             \mdf@thm@caption{#2}{{#4}{\csname the#2\endcsname}{##1}}%
             }
          \begin{mdframed}[#1,frametitle={\strut#4\ \csname the#2\endcsname%
                                          \@temptitle}]}%
          {\end{mdframed}}%
        \newenvironment{#2*}[1][]{%
          \ifstrempty{##1}{\let\@temptitle\relax}{\def\@temptitle{:\ ##1}}%
          \begin{mdframed}[#1,frametitle={\strut#4\@temptitle}]}%
          {\end{mdframed}}%
     }%
   }%
 }
\makeatother

\definecolor{cwhite}{HTML}{d7dbd7}
\mdfdefinestyle{important}{
    linecolor=yellow,
    linewidth=0pt,
    innertopmargin=0pt,
    innerbottommargin=2pt,
    innerrightmargin=2pt,
    innerleftmargin=2pt,
    leftmargin=0pt,
    rightmargin=0pt,
    backgroundcolor=cwhite,
    frametitleaboveskip=1pt,
    frametitlebelowskip=1pt,
    theoremseparator={},
    theoremspace={},
}
\mdtheorem[style=important]{ntheorem}{}

\definecolor{bgold}{HTML}{FFED8A}
\mdfdefinestyle{trick}{
    linecolor=yellow,
    linewidth=0pt,
    innertopmargin=0pt,
    innerbottommargin=2pt,
    innerrightmargin=2pt,
    innerleftmargin=2pt,
    leftmargin=0pt,
    rightmargin=0pt,
    backgroundcolor=bgold,
    frametitleaboveskip=1pt,
    frametitlebelowskip=1pt,
    theoremseparator={},
    theoremspace={},
}
\mdtheorem[style=trick]{note}{}

% Table stuff
\usepackage{tabularx} % tabularx since the width should be handled automatically
\usepackage{booktabs}
\usepackage{makecell}

% Graph stuff
\usepackage{pgfplots}

% Miscellaneous
\usepackage{hyperref}
\hypersetup{colorlinks=true, urlcolor=blue, linkcolor=blue, citecolor=blue}

\usepackage{enumitem}
\setitemize{itemsep=0.5pt, topsep=0.5pt}
\setenumerate{itemsep=0.75pt, topsep=0.5pt}

\usepackage{graphics}

\newlist{exanswers}{itemize}{2}
\setlist[exanswers]{itemsep=2pt, topsep=2pt}
\setlist[exanswers,1]{label=$\diamond$,leftmargin=5mm}
\setlist[exanswers,2]{label=\textbullet,leftmargin=1mm}

% Custom commands
\newcommand{\R}{\mathbb{R}}
\newcommand{\Q}{\mathbb{Q}}
\newcommand{\N}{\mathbb{N}}
\newcommand{\Z}{\mathbb{Z}}
\newcommand{\C}{\mathbb{C}}
\newcommand{\J}{\mathcal{J}}
\newcommand{\BO}{\mathcal{O}}
\newcommand{\Hess}{\text{Hess}}
\renewcommand{\labelenumii}{\arabic{enumi}.\arabic{enumii}}
\newcommand{\defi}{\stackrel{\text{def}}{\iff}}

% Metadata
\title{Analysis II Summary}
\author{Nicola Studer \\ \href{mailto:nicstuder@student.ethz.ch}{nicstuder@student.ethz.ch}}
\date{\vspace{-5ex}}


% ------------------------
% Document
% ------------------------

\begin{document}
\maketitle

\section{Ordinary differential equations}
\[F(x, y(x), y'(x), \ldots, y^{(n)}) = 0\]
Given a function \(F\) of \(x, y\), where \(y\) is a function itself. \(F\) is an implicit ODE of \textbf{order} \(n\).

\begin{note*}[Linear ODE's]
    \(y^{(k)} + a_{k-1}(x)y^{(k-1)} + \ldots + a_1(x)y' + a_0(x)y = b(x)\)
    with \(a_{k-1}, \ldots, a_0, b\) as cont. functions of \(x\) in \(I \subset \R\). If \(\bm{b = 0}\) then the ODE is called \textbf{homogeneous}.
\end{note*}

\begin{ntheorem*}[Properties of linear ODEs]
    \begin{enumerate}
        \item all coefficients are continuous functions
        \item no products of \(y\) and its derivatives
        \item no powers of \(y\) and its derivatives
        \item no functions which depend on \(y\) or its derivatives
        \item no leading coefficient in front of the highest derivative
    \end{enumerate}
\end{ntheorem*}

\begin{theorem}[Main result about linear ODEs]
    \(\;\)
    \begin{enumerate}
        \item Let \(\mathcal{S}_0\) be the set of solutions when \(b = 0\). Then \(\mathcal{S}_0\) is a vector space of dimension \(k\). If \(f_1, \ldots, f_k\) are the solutions, then so is \(a_1f_1+ \ldots a_kf_k\).
        \item For any \textbf{initial condition} (i.e. for any choice of \(x_0 \in I\)) there is a unique solution \(f \in \mathcal{S}_0\) such that: \\
        \(f(x_0) = y_0, f'(x_0) = y_1, \ldots, f^{(k-1)}(x_0) = y_{k-1}\)
        \item For any arbitrary \(b(x)\), the set of solutions of the ODE is \(\mathcal{S}_b = \{f + f_p \ | \ f \in \mathcal{S}_0\}\) where \(f_p\) is a particular solution of the ODE.
        \item For any initial condition there is a unique solution \(f \in \mathcal{S}_b\).
    \end{enumerate}
\end{theorem}

\pagebreak
\subsection{Linear ODE of order 1}
\begin{note*}[Solution and derivation]
        \(1.\) Solve the homogeneous ODE:
        \begin{align*}
            && y' + ay &= 0 & \\
            &\implies & y' &= - ay & \\
            &\implies & \tfrac{y'}{y} &= -a & (\text{assume \(y \neq 0\) no \(I\)}) \\
            &\implies & \ln(|y|) &= -A + C & (A(x) = \smallint a(x) \,dx) \\
            &\implies & y &= e^{-A + C} = z\cdot e^{-A} & (\text{simplify})
        \end{align*}
        \(2.\) Find \(f_p: I \to \C\) such that \(f'_p + a(x)f_p = b(x)\).
\end{note*}

\subsubsection*{Method of undetermined coefficients}
\begin{tabular}{|c|c|}
    \hline
    \(\bm{b(x)}\) & \textbf{Guess} \\
    \hline
    \(a e^{\alpha x}\) & \(c e^{\alpha x}\) \\
    \(P_n(x)\) & \(Q_n(x)\) \\
    \hline
    \makecell{\(a \sin(\beta x)\) \\ \(a \cos(\beta x)\)} & \(D \sin(\beta x) + E \cos(\beta x)\) \\
    \hline
    \makecell{\(a e^{\alpha x} \sin(\beta x)\) \\ \(a e^{\alpha x} \cos(\beta x)\)} & \(D e^{\alpha x} \sin(\beta x) + E e^{\alpha x} \cos(\beta x)\) \\
    \hline
    \(P_n(x) e^{\alpha x}\) & \(Q_n(x) e^{\alpha x}\) \\
    \hline
    \makecell{\(P_n(x) e^{\alpha x} \sin(\beta x)\) \\ \(P_n(x) e^{\alpha x} \cos(\beta x)\)} & \(e^{\alpha x} (Q_n(x) \sin(\beta x) + R_n(x) \cos(\beta x))\) \\
    \hline
\end{tabular}

\begin{enumerate}
    \item If \(b(x)\) is a linear combination of the basis functions, use corresponding linear combination of the functions.
    \item If \(f_p = f_0\), try to multiply it with \(x^m\) where \(m\) denotes the multiplicity of the eigenvalue.
\end{enumerate}

\subsubsection*{Variation of parameters}
\begin{enumerate}
    \item Assume \(f_p = z(x) \cdot e^{-A(x)}\) for a function \(z: I \to \C\)
    \item Insert the equation and construct \(z\):
    \begin{align*}
        &&y' + ay &= b \\
        &\implies& z'e^{-A} &= b \\
        &\implies& z' &= be^{A} \\
        &\implies& z &= \smallint_{x_0}^x b(t)e^{A(t)}\,dt \\
        &\implies& f_p &= \smallint_{x_0}^x b(t) e^{A(t)}\,dt \cdot e^{-A(x)}
    \end{align*}
\end{enumerate}

\subsubsection*{Integration Factor}
\begin{align}
    \tag{\(\dagger\)} \frac{dy}{dx} + a(x) y = b(x)
\end{align}
\begin{enumerate}
    \item Multiply both sides of (\(\dagger\)) with \(e^{A(x)} = e^{\smallint a(x)\,dx}\) \par
    \centering
    \(\frac{dy}{dx} e^{\smallint a(x)\,dx} + ya(x)e^{\smallint a(x)\,dx} = b(x)e^{\smallint a(x)\,dx}\)
    \item \raggedright Observe the product rule on the left hand side: \par
    \centering
    \(\frac{d}{dx}ye^{\smallint a(x)\,dx} = b(x)e^{\smallint a(x)\,dx}\)
    \item \raggedright Call \(y e^{\smallint a(x)\,dx}:=z(x) \implies y = z(x)e^{-A(x)}\) (\(\ddagger\)) \par
    \centering
    \(\frac{d}{dx}z(x) = b(x)e^{\smallint a(x)\,dx}\)
    \item \raggedright Solve for \(z(x)\): \(z(x) = \int b(x) e^{A(x)}\,dx\)
    \item \raggedright Insert (\(\ddagger\)):
    \(y = \left(\int b(x)e^{A(x)} \,dx\right) e^{-A(x)}\)
\end{enumerate}

\subsection{Linear ODE with constant coefficients}
\[Dy = b(x) \quad D = \frac{d^k}{dx^k} + a_{k - 1} \frac{d^{k-1}}{dx^{k-1}} + \ldots + a_0\]

\subsubsection*{1. Solve homogeneous equation}
Assume \(y = e^{\lambda x}\) for some \(\lambda \in \C\). We put that guess in the initial formula and get the following (simplified) form:
\[e^{\lambda x}(\lambda^k + a_{k-1}\lambda^{k-1} + a_{k-2}\lambda^{k-2} + \ldots + a_0) = e^{\lambda x} \cdot P(\lambda) = 0\]
Since \(e^{\lambda x}\) can never be \(0\) it follows that \(P(\lambda)\) must be \(0\). \(P(\lambda)\) is called the \textbf{characteristic polynomial} with its roots called \textbf{eigenvalues}.

\begin{theorem}
    \(D e^{\lambda x} = 0 \iff \lambda\) is a root of \(P_D(\lambda)\)
\end{theorem}

\begin{note*}[Solutions]
    The functions \(f_{i, m}: x \mapsto x^m e^{\lambda_i x}\) span the solution space \(S_0\) with \(m\) as the multiplicity of \(\lambda_i\).

    \begin{itemize}
        \item If \(\lambda = a + ib\) is root \(P(\lambda)\), then \(P(\overline{\lambda})\) is a root as well.
        \item Complex root: \(e^{(a + bi) \cdot x} = e^{ax}[\cos(bx) + i \sin(bx)]\)
        \item If \(b = e^{\alpha x}\), but \(\alpha\) is a root of \(P(\lambda)\) with \(m = k\), then try \(zx^k \cdot e^{\alpha x}\)
    \end{itemize}
\end{note*}

\begin{ntheorem*}[Superposition Principle]
    \(D(y_1 + y_2) = D(y_1) + D(y_2) = b_1 + b_2\)
\end{ntheorem*}

\pagebreak
\section{Differential calculus in \(\R^n\)}
\subsection*{Terminology}
\begin{tabular}{>{\bfseries}l l}
    Vector Field & \(f: \R^n \to \R^m \quad (m > 1)\) \\
    Scalar Field & \(f: \R^n \to \R\) \\
    Monomial & \(f :\begin{cases}
        \R^n \to \R \\
        (x_1, x_2, \ldots, x_n) \mapsto \alpha x_1^{d_1} x_2^{d_2} \ldots x_n^{d_n}
    \end{cases}\) \\
    Linear Map & \(f :\begin{cases}
        \R^n \to \R \\
        x \mapsto Ax \quad (A \in \C^{m \times n})
    \end{cases}\) \\
    Affine Map & \(f :\begin{cases}
        \R^n \to \R \\
        x \mapsto Ax + y_p \quad (y_p \in \R^m)
    \end{cases}\) \\
    Cart. Prod. & \(f :\begin{cases}
        \R^n \to \R^{s + t} \\
        x \mapsto (f_1(x), f_2(x))
    \end{cases}\) \\
\end{tabular}

\begin{ntheorem*}[Converges of sequences]
    \((x_k)_{k \in \N} \subset \R^n, \ y \in \R^n\). \(\lim\limits_{k \to \infty}x_k = y\)
    \begin{itemize}
        \item[\(\Leftrightarrow\)] \(\forall \epsilon > 0 \, \exists N \geq 1 \, \forall k \geq N: ||x_k - y|| < \epsilon\)
        \item[\(\Leftrightarrow\)] For each \(i\), \(1 \leq i \leq n\) the sequence \((x_{k, i}) \subset \R\) of real numbers converges to \(y_i \in \R\)
        \item[\(\Leftrightarrow\)] The sequence of real numbers \(||x_k - y|| \to 0\)
    \end{itemize}
\end{ntheorem*}

\begin{definition}
    \(f: X \subset \R^n \to \R^m\), \(x_0 \in X\).
    \(f\) \textbf{has a limit} \(y \in \R^m\) as \(x \to x_0\) (with \(x \neq x_0\)) if
    \begin{enumerate}
        \item \(\forall \epsilon > 0 \, \exists \delta > 0 \, \forall x \in X, x \neq x_0: ||f(x) - y|| < \epsilon\)
        \item \(\forall\) sequences \((x_k)\) in \(X\) with \(\lim x_k = x_0\) and \(x_k \neq x_0\) converges the sequence \(f(x_k)\) to \(y\).
    \end{enumerate}
\end{definition}

\begin{ntheorem*}[Continuity]
    \(f: X \to \R^m\) cont. at \(x_0\) if
    \begin{enumerate}
        \item \(\forall \epsilon > 0 \, \exists \delta > 0 \, \forall x \in X : \)
        \[||x-x_0|| < \delta \implies ||f(x) - f(x_0)|| < \epsilon\]
        \item \(\forall\)seq. \((x_k)\) with \(\lim x_k = x_0: \lim f(x_k) = f(x_0)\)
    \end{enumerate}
    \(f\) cont. on \(X\) if it is cont. \(\forall x_0 \in X\).
\end{ntheorem*}

\begin{corollary}
    \begin{enumerate}
        \item \(f_1: \R^n \to \R^m, f_2: \R^n \to \R^s\) cont., then \(f: (f_1, f_2): \R^n \to \R^{m + s}, x \mapsto (f_1(x), f_2(x))\) is cont.
        \item \(f: \R^n \to \R^m, x \mapsto (f_1(x), f_2(x), \ldots)\) cont.\\  \(\iff  \forall 1 \leq i \leq m \ f_i: \R^n \to \R\) are cont.
        \item \(f: \R^n \to \R^m, x \mapsto Ax\) and polynomials are cont.
        \item Sums/products of cont. functions are cont.
        \item Functions with separated variables are cont. if each variable is cont.
        \item Composition of cont. functions are cont.
        \item If \(f: \R^2 \to \R\) is cont. Fix \(y_0 \in \R\). \\ Define \(g_{y_0}(x) := f(x, y_0)\). Then \(g_{y_0}: \R \to \R\) is cont.
        \[\not\Rightarrow f: \R^2 \to \R \ \text{is cont.}\]
    \end{enumerate}
\end{corollary}

\begin{ntheorem*}[Sandwich lemma]
    \(f, g, h : \R^n \to \R\), \(\forall x \in \R^n: f(x) < g(x) < h(x)\)
    \[\lim_{x \to a} f(x) = L = \lim_{x \to a} h(x) \implies \lim_{x \to a}g(x) = L\]
\end{ntheorem*}

\begin{note*}[Polar Coordinates]
    For \(f: \R^2 \to \R\) polar coordinates are sometimes helpful.
    \(x = r \cos \theta \quad y = r \sin \theta\)
    \[\lim_{(x, y) \to (0, 0)}f(x, y) = \lim_{(r\cos\theta, r\sin\theta)  \to (0,0)}f(x, y) = \ldots = \lim_{r \to 0} \zeta\]
\end{note*}

\subsection*{Sets Bounds \(M \subseteq \R^n\)}
\vspace{-8pt}
\begin{align*}
    M \ \textbf{is bounded} \defi & \{||x|| \in \R \mid x \in M\} \ \text{is bounded} \\
    M \ \textbf{is open} \defi & \forall p \in M: \exists r \in \R^{>0}: B_p(r) \subseteq M \\
    \defi & \R^n \setminus M \ \text{is closed}\\
    M \ \textbf{is closed} \defi & \forall (x_k)_{k \in \N} \subseteq M \ \text{that converge to} \\
    & x \in \R^n: x \in M \\
    M \ \textbf{is compact} \defi& M \ \text{closed and bounded}
\end{align*}

\begin{note*}[Special Sets]
    \begin{itemize}
        \item \(\R^n\) and \(\emptyset\) are the \textbf{only} open and  closed sets of \(\R^n\).
        \item The open disc \(B_r(x_0) = \{x \in \R^n \mid |x - x_0| < r\}\) is bounded and open.
        \item The closed disc \(\overline{B_r(x_0)} = \{x \in \R^n \mid |x - x_0| < r\}\) is closed.
        \item \(I_1 \times \ldots I_n\) is closed (compact) if each interval \(I_i\) is closed (compact)
    \end{itemize}
\end{note*}

\begin{theorem}
    \(f: \R^n \to \R^m\) cont. \(\forall Y \subseteq \R^m\) \textbf{closed}, the set \(f^{-1}(Y) = \{x \in \R^n \mid f(x) \in Y\}\) is closed.
\end{theorem}

\begin{theorem}
    \(f: \R^n \to \R^m\) cont. \(\forall Y \subseteq \R^m\) \textbf{open}, the set \(f^{-1}(Y) = \{x \in \R^n \mid f(x) \in Y\}\) is open.
\end{theorem}

\begin{ntheorem*}[Min-Max theorem]
    \(X \subseteq \R^n\) compact. \(f: X \to \R\) cont. \(\implies\)
    \[\exists x_+, x_- \in X: f(x_+) = \sup_{x \in X}(f(x)), \quad f(x_-) = \inf_{x \in X} f(x)\]
\end{ntheorem*}

\subsection{Partial derivatives}
\begin{definition}
    \(f: X \subseteq \R^n \to \R\), \(X\) open. The \textbf{partial derivative} of \(f\) with respect to \(x_i\) at the point \(a \in \R^n\) is
    \[\frac{\partial f}{\partial x_i}(a) = \lim_{h \to 0} \frac{f(a + h e_i) - f(a)}{h} \hfill \ \ ((e_i)_j = \delta_{ji}, j = 1, \ldots, n)\]

    If \(f: X \to \R^m\) for \(x_0 \in \R^n\), then
    \[\frac{\partial f}{\partial x_i}(a) = \begin{bmatrix}
        \frac{\partial f_1}{\partial x_i}(a) \\
        \vdots \\
        \frac{\partial f_m}{\partial x_i}(a)
    \end{bmatrix}\]
\end{definition}

\begin{corollary}
    \(X \subseteq \R^n\) open, \(f, g: X \to \R^m\):
    \begin{itemize}
        \item \(\partial_{x_i}(f + g) = \partial_{x_i}(f) + \partial_{x_i}(g)\)
        \item \(\partial_{x_i}(f \cdot g) = \partial_{x_i}(f) \cdot g + f \cdot \partial_{x_i}(g)\) if \(m = 1\)
        \item \(\partial_{x_i}(f / g) = (\partial_{x_i}(f) \cdot g - f \cdot \partial_{x_i}(g)) / g^2\) if \(m = 1, g \not\equiv 0\)
    \end{itemize}
\end{corollary}

\begin{definition}
    The \textbf{Jacobi Matrix} of \(f: X \subset \R^n \to \R^M\) at \(x \in X\):
    \[\J_f(x) = \begin{bmatrix}
        \frac{\partial f_i}{\partial x_j}(x)
    \end{bmatrix}_{\substack{1 \leq i \leq m \\ 1 \leq j \leq n}}\]
\end{definition}

\begin{definition}
    \(f: X \subseteq \R^n \to \R\), \(X\) open. The \textbf{gradient} of \(f\):
    \[\nabla f(x) = \J_f(x)^\top\]
    The gradient points in the direction of greatest increase and is perpendicular to the level set.
\end{definition}

\subsection{The differential}
\begin{definition}
    \(f: \R^n \to \R^m\) is diff. at \(x_0\), with \textbf{differential} \(u\), if there exists a linear map \(u: \R^n \to \R^m\) such that
    \[\lim_{\substack{x \to x_0 \\ x \neq x_0}} \frac{f(x) - (f(x_0) + u(x - x_0))}{||x-x_0||} = 0\]

    The linear map \(u: \R^n \to \R^m\) is called (total) \textbf{differential} of \(f\) at \(x_0\), denoted by \(df(x_0), d_{x_0}f\)
\end{definition}

\begin{theorem}
    \(f, g: \R^n \to \R^m\) diff. at \(x_0 \implies\)
    \begin{enumerate}
        \item \(f\) is cont. at \(x_0\)
        \item \(f\) admits partial derivatives on \(x_0\) w.r.t each variable
        \item The matrix that represents the differential in the standard basis is \(\J_f(x_0)\)
        \item \(d_{x_0}(f \pm g) = d_{x_0}f \pm d_{x_0}g\)
        \item \(d_{x_0}(f \cdot g) = (d_{x_0}f)g(x_0) + f(x_0) \cdot (d_{x_0} g)\) if \(m = 1\)
        \item If \(m = 1\) and \(g \not\equiv 0\), the \(f / g\) is diff.
    \end{enumerate}
\end{theorem}

\begin{ntheorem*}[Multivaraible Chain Rule]
    Let \(X \subseteq \R^n\) and \(Y \subseteq \R^m\) be open \(f: X \to Y, g: T \to \R^p\) diff functions, then
    \[d_{x_0}(g \circ f) = d(g \circ f)(x_0) = dg(f(x_0)) \circ df(x_0)\]
    In particular, the Jacobi matrix satisfies
    \[\J_{g \circ f}(x_0) = \J_g(f(x_0)) \cdot \J_f(x_0)\]
\end{ntheorem*}

\begin{ntheorem*}[Partial Convergence]
    If \(f : X \to \R^m\) has all partial derivatives \(\frac{\partial f_i}{\partial x_j}: X \to \R^m\) and if these functions are cont. in \(X\) \(\implies\) \(f\) is diff. on \(X\).
\end{ntheorem*}

\begin{definition}
    The \textbf{tangent space} at \(x_0\) is the graph of the affine linear map
    \[g(x) = f(x_0) + (d_{x_0}f)(x - x_0) \ \text{i.e} \ \{(x, g(x)) \in \R^n \times \R^m\}\]
\end{definition}

\begin{definition}
    \(X \subseteq \R^n\) open, \(f: X \to \R^m\), \(v \in \R^n \neq 0\), \(x_0 \in X\). The \textbf{directional derivative} in direction \(v\) is
    \[\lim_{t \to 0} \frac{f(x_0 + tv) - f(x_0)}{t} = d_vf(x_0) = df(v;x_0) = df(x_0)[v]\]
    The directional derivative be calculated using the Jacobian:
    \[\frac{d}{dt} f(x_0 + tv) |_{t=0} = d_vf(x_0) = \J_f(x_0) \cdot v\]
\end{definition}

\begin{note*}[Summed up]
    \begin{itemize}
        \item \(f\) diff \(\implies f\) cont.
        \item \(f\) has all partial derivatives \(\centernot\implies f\) cont.
    \end{itemize}
\end{note*}

\begin{note*}[Parameterized curve]
    \(\gamma: [a, b] \to \R^n\), cont. and piecewise \(C^1\). \(\gamma\) is a parameterized curve, \(\gamma(t)\) is a parameterization of the curve \(\Im \gamma = \gamma([a, b])\).
\end{note*}

\subsection{Change of variables}
\begin{definition}[Change of variables]
    \(X \subset \R^n\) open, \(f: X \to \R^n\) diff. \(f\) is a change of variables around \(x_0\) if there is a radius \(r > 0\), such that the restriction of \(f\) to Ball \(B_r(x_0) := \{x \in \R^n \ | \ ||x - x_0|| < r\}\) has the property that the image \(Y = f(B_r(x_0))\) is open in \(\R^n\) and \(\exists\) diff. map \(g: Y \to B\) s.t. \(f \circ g = id = g \circ f\).
\end{definition}

\begin{ntheorem*}[Inveres function theorem]
    \(X \subseteq \R^n\) open, \(f: X \to \R^n\) diff. If \(x_0 \in X\) is such that \(\det(\J_f(x_0)) \neq 0\), then \(f\) is a change of variables around \(x_0\). Moreover the Jacobian of \(g\) is determined by \(\J_g(f(x_0)) = \J_f(x_0)^{-1}\).

    This is the analog of the fact that in \(n = 1\) for a function \(f : I \to \R\) \(f\) is bijective from \(I\) to its image if \(f' > 0\) (or \(f' < 0\))
\end{ntheorem*}

\subsection{Higher order partial derivatives}
\begin{definition}
    \(X \subset \R^n\) open, \(f: X \to \R^m\). We say \(f\) is diff. of class \(C^1\) if \(f\) is diff. on \(X\) and all its partial derivatives are continuous.

    The set of all \(C^1\) functions are denoted by \(C^1(X ; \R^m)\).

    Let \(k \geq 2\), then \(f \in C^k(X; R^m)\) if its diff. and each \(\partial_{x_i}f \in C^{k - 1}(X; \R^m)\).

    \(f\) is smooth or \(C^\infty\) if \(f \in C^k(X; \R^m) \ \forall k\).
\end{definition}

\begin{note*}[Known \(C^\infty\) functions]
    All polynomials, trigonometric functions, exponential functions are of class \(C^\infty\)
\end{note*}

\begin{ntheorem*}[Mixed derivatives commute]
    If \(f \in C^k\), \(k \geq 2\) then the partial derivatives of oder \(\leq k\) are independent of the order of differentiation.
    \[\frac{\partial}{\partial x_{i_k}} \dots \left(\frac{\partial}{\partial x_{i_2}}\left(\frac{\partial f}{\partial x_{i_1}}\right)\right) = \frac{\partial^k}{\partial x_{i_k} \cdot \ldots \cdot \partial x_{i_2} \cdot \partial x_{x_{i_1}}}\]
\end{ntheorem*}

\begin{definition}[Hessian]
    \(f: X \to \R, X \subset \R^n\). If \(f \in C^2(X;\R)\), \(x_0 \in X\) the Hessian matrix of \(f\) at \(x\) is the symmetric square matrix
    \[\Hess_f(x_0) = \nabla^2 f(x_0) = \begin{bmatrix}
        \frac{\partial^2 f(x_0)}{\partial x_i \partial x_j}
    \end{bmatrix}_{\substack{1 \leq i \leq n \\ 1 \leq j \leq n}}\]
\end{definition}

\begin{ntheorem*}[Taylor Polynomial for \(f: \R^n \to \R\)]
    Approximation of \(f(y)\) for \(y\) close to \(x_0\).
    \begin{align*}
        T_1 f(x_0; y) &= f(x_0) + \nabla f(x_0) \cdot y \\
        &= f(x_0) + \frac{\partial f}{\partial x_1} (x_0) y_1 + \ldots \frac{\partial f}{\partial x_n} (x_0) y_n
    \end{align*}
    \begin{align*}
        T_2 f(x_0; y) = & f(x_0) + \nabla f(x_0) \cdot y \\
        & \frac{1}{2!} y \cdot \Hess_f(x_0) \cdot y^\top
    \end{align*}
    \begin{multline*}
        T_kf(x_0;y) = f(x_0) + \sum_{i = 1}^n \frac{\partial f}{\partial x_i} (x_0) y_i + \ldots + \\ \sum_{m_1 + \ldots m_n = k} \frac{1}{m_1! \cdot m_2! \cdot m_n!} \frac{\partial^k f}{\partial x_1^{m_1} \ldots \partial x_n^{m_n}}(x_0) \cdot y_1^{m_1 \ldots y_n^{m_n}}
    \end{multline*}
\end{ntheorem*}

\begin{ntheorem*}[Taylor Approximation]
    Let \(f \in C^k(X; \R), x_0 \in X\)
    \[f(x) = T_k f(x_0, x - x_0) + E_k(f, x, x_0)\]
    which implies
    \[\lim_{x \to x_0} \frac{E_k(f, x, x_0)}{||x - x_0||^2} = 0\]
\end{ntheorem*}

\subsection{Critical points}
\begin{definition}
    \(x \in X\) is called a critical point of \(f\) if \(\nabla f(x_0) = 0\)
\end{definition}

\begin{theorem}
    \(f: X \subset R^n \to \R\) diff. If \(X\) closed and bounded, then a global extrema of \(f\) exists and is either at a point \(x_0 \in\) interior of \(X\) for which \(\nabla f(x_0) = 0\) or \(x_0 \in\) boundary of \(x\).
\end{theorem}

\begin{definition}[Non-degenerate critical point of \(f \in C^2(X, \R)\)]
    \[\det(\Hess_f(x_0)) \neq 0\]
\end{definition}

\begin{theorem}
    \(f: X \subset \R^n \to \R\), \(f \in C^2(X, \R)\). Let \(x_0 \in X\) be a critical point of \(f, \nabla f(x_0) = 0\). Then
    \begin{enumerate}
        \item \(\Hess_f(x_0)\) pos. def. \(\implies\) loc. min.
        \item \(\Hess_f(x_0)\) neg. def. \(\implies\) loc. max.
        \item \(\Hess_f(x_0)\) indefinite \(\implies\) saddle point.
    \end{enumerate}
\end{theorem}

\begin{note*}
    If \(\nabla f(x_0) = 0\), but also \(\det \Hess_f(x_0) = 0\), the we have to calculate each case individually.
\end{note*}

\begin{note*}[General recipe for finding \textbf{global} extrema:]
    Let \(f: X \subseteq \R^n \to \R\), compact and \(f \in C^2(X)\).
    \begin{enumerate}
        \item Calculate \(\nabla f(x)\) and \(\Hess_f(x)\)
        \item Find all solutions of \(\nabla_f(x) = 0\), \(S := \{\)
    \end{enumerate}
\end{note*}

\section*{Integration in \(\R^n\)}
\subsection*{Line Integrals}

\begin{definition}
    Let \(\gamma: [a, b] \to \R^n\) be a curve in \(\R^n\). \(X \subset \R^n\) a subset of \(\R^n\) which contains the image of \(\gamma\). \(v: X \to \R^n\) a continuous function. The integral
    \[\int_a^b (v(\gamma(t)) \cdot \gamma'(t)) \, dt\]
    is called the line or path integral of \(v\) along \(\gamma\)
\end{definition}

\begin{note*}[Other notations] \ \\
    \begin{itemize}
        \item \(\int\limits_\gamma v \,ds\) if \(v = (v_1(x), v_2(x), \ldots, v_n(x))\) and \(\gamma(t) = (\gamma_1(t), \gamma_2(t), \ldots \gamma_n(t))\) then \(ds\) represents \(\gamma'(t) \,dt\), multiplied by scalar product.
    \end{itemize}
\end{note*}

\begin{ntheorem*}[Properties of the line intergral] \ \\
    \begin{itemize}
        \item It is independent of orientation preserving reparametrization of the curve.
        
        Reparametrization: if \(\gamma: [a, b] \to \R^n\), \(\sigma: [c, d] \to [a, b]\) which is \(C^1\) s.t. \(\sigma(c) = a\) and \(\sigma(d) = b\) and \(\sigma'(t) > 0\) then \(\tilde{\gamma} = \gamma \circ \sigma: [c, d] \to \R^n\)
    \end{itemize}
\end{ntheorem*}

Suppose \(f: X \to \R^n\), a vector field such that \(\exists g: X \to \R^n\), \(g \in C^1\), ...

\begin{definition}
    A diff. function \(g: X \subset \R^n \to \R\), such that \(\nabla g = f\) is called a potential for \(f\).
\end{definition}

\begin{theorem}
    \(f: X \subseteq \R^n \to \R^n, C^1\) vector field. If \(f\) conservative then \(\frac{\partial f_j}{\partial x_i} = \frac{\partial f_i}{\partial x_j}\) for \(1 \leq i, j \leq n\)
\end{theorem}

\begin{definition}
    \(f: X \subseteq \R^3 \to \R^3, C^1\), then the curl of \(f\) is defined as
    \[\text{curl}(f) := \begin{pmatrix}
        \partial_y f_4 - \partial_z f_2 \\
        \partial_z f_1 - \partial_x f_3 \\
        \partial_x f_2 - \partial_y f_1
    \end{pmatrix}\]
\end{definition}

\begin{theorem}
    \(f: \R^3 \to \R^3\) \(f\) cons. \(\implies \text{curl}(f) = 0\)
\end{theorem}

\begin{note*}
    If \(f\) cons. \(\implies\) sym. of partials. But \(\not\Leftarrow\) whether the symmetry implies conservative or not depends on where \(f\) is defined.
\end{note*}

\begin{definition}
    A subset \(X \subseteq \R^n\) is start shaped if \(\exists x_0 \in X\) such that \(\forall x \in X\) the line segment joining the \(x\) to \(x_0\) is contained in \(X\).
\end{definition}

\begin{definition}
    A subset \(X \subseteq \R^n\) is convex, when for any \(x, y \in X\) the line segment from \(x\) to \(y\) is contained in \(X\). convex \(\implies\) star-shaped
\end{definition}

\begin{theorem}
    If \(X\) start-shaped open and \(f \in C^1\) vector field. Then \(\frac{\partial f_i}{\partial x_j} = \frac{\partial f_j}{\partial x_i}\) for \(\forall 1 \leq i, j \leq n\). Implies that \(f\) is conservative.
\end{theorem}

t.b.d. riemann inteagral defintions

\begin{theorem}
    \(f\) is cont. and bounded on \(Q\), then \(f\) is integrable.
\end{theorem}

\begin{definition}
    For \(1 \leq m \leq n\) a \(m\)-parameterized set or parameterized \(m\)-set is a continuous function \(\phi: [a_1, b_1] \times \ldots \times [a_m, b_m] \to \R^n\) which is \(C^1\) on \((a_1, b_1) \times \ldots \times (a_m, b_m)\)
\end{definition}

\begin{definition}
    A set \(Y \subseteq \R^n\) is called negligible if \(\exists\)finitely mani \(\phi_i\), parameterized \(m_i\)-sets with \(m_i \leq n\) such that \(1 \leq i \leq k\)
    \[Y \subseteq \bigcup_{i = 1}^k \phi_i(x_i)\]
    where \(\phi_i: x_i \to \R^n\)
\end{definition}

\begin{theorem}
    If \(Y \subseteq \R^n\) is negligible closed bounded then
    \[\int_Y f(x_1, \ldots, x_n) \, dx_1 \, \ldots dx_n = 0\]
    \(\forall f: Y \to \R\) continuous
\end{theorem}

\subsection*{Improper Integrals}
\begin{definition}
    We say \(f\) is integrable on \(I \times J\) if
    \[\lim_{b \to \infty} \int_a^b \int_I f(x, y) \,dx \,dy = \lim_{b \to \infty} \int_I \int_a^b f \, dy \, dx\]
    exists and denote the limit with
    \[\int_a^\infty \int_I f \,dx \,dy = \int_{I \times J} f \,dx \,dy\]
\end{definition}

\begin{ntheorem*}[Change of variable]
    In \(\R^n\), we have \(\phi: X \to Y\) with \(X\) bounded and closed and \(Y\) bounded and closed.
\end{ntheorem*}

\subsection*{The Green formula}

\begin{ntheorem*}[Green's formula]
    \[\int_X (\frac{\partial f_2}{\partial x} - \frac{\partial f_1}{y}) \, dx dy = \int_{\partial X} f \cdot ds \quad (f = (f_1, f_2))\]
    We made assumptions \(f = (f_1, f_2)\) is \(C^1\) in the region \(X\). so that \(curl f = \frac{\partial f_2}{\partial x } - \frac{\partial f_1}{\partial y}\) is integrable.

    \noindent The region \(X\) is closed and bounded and its boundary is a simple closed cure if \(\gamma[a, b] \to \R^2\) is the curve which is the boundary of \(X\).
\end{ntheorem*}

\begin{ntheorem*}
    let \(f: X \to \R^2\) \(C^1\) vector field, \(X\) closed and bounded where \(\partial X = \bigcup_{i = 1}^n \gamma_i\) union of simple close curves so that \(X\)  is always to the left of the curve \(\gamma = \bigcup_{i = 1}^n \gamma_i\) then
    \[\iint_{X} \left(\frac{\partial f_2}{\partial x} - \frac{\partial f_1}{y}\right) \, dx dy = \int_\gamma f \, ds\]
\end{ntheorem*}

\end{document}
