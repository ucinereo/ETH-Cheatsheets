% Configuration
\documentclass[a4paper, 10pt]{article}

% Formatting
\usepackage[landscape, left=0.75cm, top=1.0cm, right=0.75cm, bottom=1.5cm, footskip=15pt]{geometry}
\setlength{\columnsep}{0.5cm}
\usepackage{flowfram}
\ffvadjustfalse
\Ncolumn{3}
\usepackage[compact]{titlesec}

% ------------------------
% Imports and commands
% ------------------------

% Language stuff
\usepackage[english]{babel}
\usepackage[utf8]{inputenc}

% Math stuff
\usepackage{amsthm}
\usepackage{amssymb}
\usepackage{amsmath}
\usepackage{bm}

\newtheorem*{corollary}{Cor}
\newtheorem*{lemma}{Lemma}

\theoremstyle{definition}
\newtheorem*{theorem}{Thm}
\newtheorem*{definition}{Def}

\newtheoremstyle{colored}{}{}{}{}{\bf}{}{.5em}{{\thmnote{(#3) }}}
\theoremstyle{colored}
\newtheorem*{note_wrapper}{}

\newtheoremstyle{ex}{2pt}{5pt}{}{}{\bf}{}{0pt}{{\thmnote{(#3) }}}
\theoremstyle{ex}
\newtheorem*{exercise}{}

\newtheoremstyle{named}{}{}{}{}{\bfseries}{.}{.5em}{\thmnote{#3}}
\theoremstyle{named} 
\newtheorem*{ntheorem_wrapper}{Theorem}

% Colored boxes
\usepackage{xcolor}
\usepackage{mdframed}
\usepackage{framed}
\mdfsetup{skipabove=-2pt,skipbelow=-2pt}

\definecolor{cwhite}{HTML}{d7dbd7}
\mdfdefinestyle{important}{
    linecolor=yellow,
    linewidth=0pt,
    innertopmargin=-6pt,
    innerbottommargin=2pt,
    innerrightmargin=2pt,
    innerleftmargin=2pt,
    leftmargin=0pt,
    rightmargin=0pt,
    outerlinewidth=0pt,
    backgroundcolor=cwhite,
}

\newenvironment{ntheorem}%
    {\begin{mdframed}[style=important]\begin{ntheorem_wrapper}}%
    {\end{ntheorem_wrapper}\end{mdframed}}

\definecolor{cgreen}{HTML}{2ecc71} 
\definecolor{bgreen}{HTML}{FFED8A}
\mdfdefinestyle{trick}{
    linecolor=yellow,
    linewidth=0pt,
    innertopmargin=-6pt,
    innerbottommargin=2pt,
    innerrightmargin=2pt,
    innerleftmargin=2pt,
    leftmargin=0pt,
    rightmargin=0pt,
    outerlinewidth=0pt,
    backgroundcolor=bgreen,
}

\newenvironment{note}%
    {\begin{mdframed}[style=trick]\begin{note_wrapper}}%
    {\end{note_wrapper}\end{mdframed}}

% Table stuff
\usepackage{tabularx} % tabularx since the width should be handled automatically
\usepackage{booktabs}
\usepackage{makecell}

% Graph stuff
\usepackage{pgfplots}

% Miscellaneous
\usepackage{hyperref}
\hypersetup{colorlinks=true, urlcolor=blue, linkcolor=blue, citecolor=blue}

\usepackage{enumitem}
\setitemize{itemsep=0.5pt, topsep=0pt}
\setenumerate{itemsep=0.75pt, topsep=0pt}

\usepackage{graphics}

\newlist{exanswers}{itemize}{2}
\setlist[exanswers]{itemsep=2pt, topsep=2pt}
\setlist[exanswers,1]{label=$\diamond$,leftmargin=5mm}
\setlist[exanswers,2]{label=\textbullet,leftmargin=1mm}

% Custom commands
\newcommand{\R}{\mathbb{R}}
\newcommand{\Q}{\mathbb{Q}}
\newcommand{\N}{\mathbb{N}}
\newcommand{\Z}{\mathbb{Z}}
\newcommand{\C}{\mathbb{C}}
\newcommand{\BO}{\mathcal{O}}
\renewcommand{\labelenumii}{\arabic{enumi}.\arabic{enumii}}

% Metadata
\title{Analysis II Summary}
\author{Nicola Studer \\ \href{mailto:nicstuder@student.ethz.ch}{nicstuder@student.ethz.ch}}
\date{\vspace{-5ex}}


% ------------------------
% Document
% ------------------------

\begin{document}
\maketitle

\section{Ordinary differential equations}
\[F(x, y(x), y'(x), \ldots, y^{(n)}) = 0\]
Given a function \(F\) of \(x, y\), where \(x\) and \(y\) are functions themselves. \(F\) is an implicit ODE of \textbf{order} \(n\).

\begin{note}[Linear ODE's]
    Linear ODE with \(a_{k-1}, \ldots, a_0, b\) as cont. functions of \(x\) in \(I \subset \R\):
    \[y^{(k)} + a_{k-1}(x)y^{(k-1)} + \ldots + a_1(x)y' + a_0(x)y = b(x)\]
    If \(\bm{b = 0}\) then the ODE is called \textbf{homogeneous}.
\end{note}

\begin{ntheorem}[Properties of linear ODEs]
    \(\;\)
    \begin{enumerate}
        \item all coefficients are continuous functions
        \item no products of \(y\) and its derivatives
        \item no powers of \(y\) and its derivatives
        \item no functions which depend on \(y\) or its derivatives
        \item no leading coefficient in front of the highest derivative
    \end{enumerate}
\end{ntheorem}

\begin{theorem}[Main result about linear ODEs]
    \(\;\)
    \begin{enumerate}
        \item Let \(\mathcal{S}_0\) be the set of solutions when \(b = 0\). Then \(\mathcal{S}_0\) is a vector space of dimension \(k\). If \(f_1, \ldots, f_k\) are the solutions, then so is \(a_1f_1+ \ldots a_kf_k\).
        \item For any \textbf{initial condition} (i.e. for any choice of \(x_0 \in I\)) there is a unique solution \(f \in \mathcal{S}_0\) such that: \\
        \(f(x_0) = y_0, f'(x_0) = y_1, \ldots, f^{(k-1)}(x_0) = y_{k-1}\)
        \item For any arbitrary \(b(x)\), the set of solutions of the ODE is \(\mathcal{S}_b = \{f + f_p \ | \ f \in \mathcal{S}_0\}\) where \(f_p\) is a particular solution of the ODE.
        \item For any initial condition there is a unique solution \(f \in \mathcal{S}_b\).
    \end{enumerate}
\end{theorem}

\subsection{Linear ODE of order 1}
\(y' + a(x)y = b(x)\) 

\begin{note}[How to solve]
        \(1.\) Solve the homogeneous ODE:
        \begin{align*}
            && y' + ay &= 0 & \\
            &\implies & y' &= - ay & \\
            &\implies & \tfrac{y'}{y} &= -a & (\text{assume \(y \neq 0\) no \(I\)}) \\
            &\implies & \ln(|y|) &= -A + C & (A(x) = \smallint a(x) \,dx) \\
            &\implies & y &= e^{-A + C} = z\cdot e^-A & (\text{simplify})
        \end{align*}
        \(2.\) Find a particular solution \(f_p: I \to \C\) with either an educated guess, variation of parameters or integration factor such that \(f'_p + a(x)f_p = b(x)\).
\end{note}

\subsubsection*{Method of undetermined coefficients}
\begin{tabular}{c|c}
    \(\bm{b(x)}\) & \textbf{Guess} \\
    \hline
    \(a e^{\alpha x}\) & \(c e^{\alpha x}\) \\
    \hline
    \makecell{\(a \sin(\beta x)\) \\ \(a \cos(\beta x)\)} & \(D \sin(\beta x) + E \cos(\beta x)\) \\
    \hline
    \makecell{\(a e^{\alpha x} \sin(\beta x)\) \\ \(a e^{\alpha x} \cos(\beta x)\)} & \(D e^{\alpha x} \sin(\beta x) + E e^{\alpha x} \cos(\beta x)\) \\
    \hline
    \(P_n(x) e^{\alpha x}\) & \(Q_n(x) e^{\alpha x}\) \\
    \hline
    \makecell{\(P_n(x) e^{\alpha x} \sin(\beta x)\) \\ \(P_n(x) e^{\alpha x} \cos(\beta x)\)} & \(e^{\alpha x} (Q_n(x) \sin(\beta x) + R_n(x) \cos(\beta x))\) \\
\end{tabular}

\begin{enumerate}
    \item If \(b(x)\) is a linear combination of the basis functions, try a linear combination.
    \item If \(f_p = f_0\), try to multiply it with \(x^m\) where \(m\) denotes the multiplicity of the root.
\end{enumerate}

\subsubsection*{Variation of parameters}
\begin{enumerate}
    \item Assume \(f_p = z(x) \cdot e^{-A(x)}\) for a function \(z: I \to \C\)
    \item Insert the equation and construct \(z\):
    \begin{align*}
        &&y' + ay &= b \\
        &\implies& z'e^{-A} &= b \\
        &\implies& z' &= be^{A} \\
        &\implies& z &= \smallint_{x_0}^x b(t)e^{A(t)}\,dt \\
        &\implies& f_p &= \smallint_{x_0}^x b(t) e^{A(t)}\,dt \cdot e^{-A(x)}
    \end{align*}
\end{enumerate}

\subsubsection*{Integration Factor}
\begin{align}
    \tag{\(\dagger\)} \frac{dy}{dx} + a(x) y = b(x)
\end{align}
\begin{enumerate}
    \item Multiply both sides of (\(\dagger\)) with \(e^{A(x)} = e^{\smallint a(x)\,dx}\) \par
    \centering
    \(\frac{dy}{dx} e^{\smallint a(x)\,dx} + ya(x)e^{\smallint a(x)\,dx} = b(x)e^{\smallint a(x)\,dx}\)
    \item \raggedright Observe the product rule on the left hand side: \par
    \centering
    \(\frac{d}{dx}ye^{\smallint a(x)\,dx} = b(x)e^{\smallint a(x)\,dx}\)
    \item \raggedright Call \(y e^{\smallint a(x)\,dx}:=z(x) \implies y = z(x)e^{-A(x)}\) (\(\ddagger\)) \par
    \centering
    \(\frac{d}{dx}z(x) = b(x)e^{\smallint a(x)\,dx}\)
    \item \raggedright Solve for \(z(x)\) \par
    \centering
    \(z(x) = \int b(x) e^{A(x)}\,dx\)
    \item \raggedright Insert (\(\ddagger\)):
    \(y = \left(\int b(x)e^{A(x)} \,dx\right) e^{-A(x)}\)
\end{enumerate}

\subsection{Linear ODE with constant coefficients}
\[Dy = b(x) \quad D = \frac{d^k}{dx^k} + a_{k - 1} \frac{d^{k-1}}{dx^{k-1}} + \ldots + a_0\]

\subsubsection*{1. Solve homogeneous equation}
Assume \(y = e^{\lambda x}\) for some \(\lambda \in \C\). We put that guess in the initial formula and get the following (simplified) form:
\[e^{\lambda x}(\lambda^k + a_{k-1}\lambda^{k-1} + a_{k-2}\lambda^{k-2} + \ldots + a_0) = e^{\lambda x} \cdot P(\lambda) = 0\]
Since \(e^{\lambda x}\) can never be \(0\) it follows that \(P(\lambda)\) must be \(0\). \(P(\lambda)\) is called the \textbf{characteristic polynomial} with its roots called \textbf{eigenvalues}.

\begin{theorem}
    \(D e^{\lambda x} = 0 \iff \lambda\) is a root of \(P_D(\lambda)\)
\end{theorem}

\begin{note}[Solutions]
    The functions \(f_{i, m}: x \mapsto x^m e^{\lambda_i x}\) span the solution space \(S_0\) with \(m\) as the multiplicity of \(\lambda_i\).

    \begin{itemize}
        \item If \(\lambda = a + ib\) is root \(P(\lambda)\), then \(P(\overline{\lambda})\) is a root as well.
        \item Root complex: \(e^{(a + bi) \cdot x} = e^{ax}[\cos(bx) + i \sin(bx)]\)
        \item The particular solution can be guessed
        \item If in the "undetermined coefficients" method \(b = e^{\alpha x}\), but \(\alpha\) is a root of \(P(\lambda)\) with \(m = k\), then we try \(zx^k \cdot e^{\alpha x}\)
    \end{itemize}
\end{note}

\begin{theorem}
    \(D(y_1 + y_2) = D(y_1) + D(y_2) = b_1 + b_2\)
\end{theorem}

\section{Differential calculus in \(\R^n\)}
\begin{definition}[Vector Field]
    \(f: \R^n \to \R^m \quad (m > 1)\)
\end{definition}

\begin{definition}[Scalar Field]
    \(f: \R^n \to \R\)
\end{definition}

\begin{definition}[Monomial]
    \[f :\begin{cases}
        \R^n \to \R \\
        (x_1, x_2, \ldots, x_n) \mapsto \alpha x_1^{d_1} x_2^{d_2} \ldots x_n^{d_n}
    \end{cases}\]
\end{definition}

\begin{definition}[Converges of sequences]
    Let \((x_k)_k \subset \R^n\) be a sequence.
     and \(y \in \R^n\). \(\lim\limits_{k \to \infty}x_k = y \iff\)
    \begin{enumerate}
        \item \(\forall \epsilon > 0 \, \exists N \geq 1 \, \forall k \geq N: ||x_k - y|| < \epsilon\)
        \item For each \(i\), \(1 \leq i \leq n\) the sequence \((x_{k, i}) \subset \R\) fo real numbers converges to \(y_i \in \R\)
        \item The sequence of real numbers \(||x_k - y|| \to 0\)
    \end{enumerate}
        
\end{definition}

\begin{definition}
    \(f: X \subset \R^n \to \R^m\), \(x_0 \in X\).
    \(f\) has a limit \(y \in \R^m\) as \(x \to x_0\) (with \(x \neq x_0\)) if
    \begin{enumerate}
        \item \(\forall \epsilon > 0 \, \exists \delta > 0 \, \forall x \in X, x \neq x_0: ||f(x) - y|| < \epsilon\)
        \item \(\forall\) sequences \((x_k)\) in \(X\) with \(\lim x_k = x_0\) and \(x_k \neq x_0\) converges the sequence \(f(x_k)\) to \(y\).
    \end{enumerate}
\end{definition}

\begin{definition}[Continuitity]
    \(f: X \to \R^m\) cont. at \(x_0\) if
    \begin{enumerate}
        \item \(\forall \epsilon > 0 \, \exists \delta > 0 \, \forall x \in X : \)
        \[||x-x_0|| < \delta \implies ||f(x) - f(x_0)|| < \epsilon\]
        \item \(\forall\)seq. \((x_k)\) with \(\lim x_k = x_0: \lim f(x_k) = f(x_0)\)
    \end{enumerate}
    \(f\) cont. on \(X\)if it is cont. \(\forall x_0 \in X\).
\end{definition}

\begin{corollary}
    \begin{enumerate}
        \item \(f_1: \R^n \to \R^m, f_2: \R^n \to \R^s\) cont., then \(f: (f_1, f_2): \R^n \to \R^{m + s}, x \mapsto (f_1(x), f_2(x))\) is cont.
        \item \(f: \R^n \to \R^m, x \mapsto (f_1(x), f_2(x), \ldots)\) cont.\\  \(\iff  \forall 1 \leq i \leq m f_i: \R^n \to \R\) are cont.
        \item \(f: \R^n \to \R^m, x \mapsto Ax\) and polynomials are cont.
        \item Sums/products of cont. functions are cont.
        \item Functions with separated variables are cont. if each variable is cont.
        \item Composition of cont. functions are cont.
        \item If \(f: \R^2 \to \R\) is cont. Fix \(y_0 \in \R\). \\ Define \(g_{y_0}(x) := f(x, y_0)\). Then \(g_{y_0}: \R \to \R\) is cont.
        \[\not\Rightarrow f: \R^2 \to \R \ \text{is cont.}\]
    \end{enumerate}
\end{corollary}

\begin{ntheorem}[Squeeze]
    \(f, g, h : \R^n \to \R\), \(\forall x \in \R^n: f(x) < g(x) < h(x)\)
    \[\lim_{x \to a} f(x) = L = \lim_{x \to a} h(x) \implies \lim_{x \to a}g(x) = L\]
\end{ntheorem}

\begin{note}[Polar Coordinates]
    For \(f: \R^2 \to \R\) polar coordinates are sometimes helpful.
    \(x = r \cos \theta \quad y = r \sin \theta\)
    \[\lim_{(x, y) \to (0, 0)}f(x, y) = \lim_{(r\cos\theta, r\sin\theta)  \to (0,0)}f(x, y) = \ldots = \lim_{r \to 0} \zeta\]
\end{note}

\end{document}
