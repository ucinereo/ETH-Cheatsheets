% Configuration
\documentclass[a4paper, 10pt]{article}

% Formatting
\usepackage[landscape, left=0.75cm, top=1.0cm, right=0.75cm, bottom=1.5cm, footskip=15pt]{geometry}
\setlength{\columnsep}{0.5cm}
\usepackage{flowfram}
\ffvadjustfalse
\Ncolumn{3}
\usepackage[compact]{titlesec}

% ------------------------
% Imports and commands
% ------------------------

% Language stuff
\usepackage[english]{babel}
\usepackage[utf8]{inputenc}

% Math stuff
\usepackage{amsthm}
\usepackage{amssymb}
\usepackage{amsmath}
\usepackage{bm}

\newtheorem*{corollary}{Cor}
\newtheorem*{lemma}{Lemma}

\theoremstyle{definition}
\newtheorem*{theorem}{Thm}
\newtheorem*{definition}{Def}

\newtheoremstyle{colored}{}{}{}{}{\bf}{}{.5em}{{\thmnote{(#3) }}}
\theoremstyle{colored}
\newtheorem*{note_wrapper}{}

\newtheoremstyle{ex}{2pt}{5pt}{}{}{\bf}{}{0pt}{{\thmnote{(#3) }}}
\theoremstyle{ex}
\newtheorem*{exercise}{}

\newtheoremstyle{named}{}{}{}{}{\bfseries}{.}{.5em}{\thmnote{#3}}
\theoremstyle{named} 
\newtheorem*{ntheorem_wrapper}{Theorem}

% Colored boxes
\usepackage{xcolor}
\usepackage{mdframed}
\mdfsetup{skipabove=-2pt,skipbelow=-2pt}

\definecolor{cwhite}{HTML}{d7dbd7}
\mdfdefinestyle{important}{
    linecolor=yellow,
    linewidth=0pt,
    innertopmargin=-6pt,
    innerbottommargin=2pt,
    innerrightmargin=2pt,
    innerleftmargin=2pt,
    leftmargin=0pt,
    rightmargin=0pt,
    outerlinewidth=0pt,
    backgroundcolor=cwhite,
}

\newenvironment{ntheorem}%
    {\begin{mdframed}[style=important]\begin{ntheorem_wrapper}}%
    {\end{ntheorem_wrapper}\end{mdframed}}

\definecolor{cgreen}{HTML}{2ecc71} 
\definecolor{bgreen}{HTML}{FFED8A}
\mdfdefinestyle{trick}{
    linecolor=yellow,
    linewidth=0pt,
    innertopmargin=-6pt,
    innerbottommargin=2pt,
    innerrightmargin=2pt,
    innerleftmargin=2pt,
    leftmargin=0pt,
    rightmargin=0pt,
    outerlinewidth=0pt,
    backgroundcolor=bgreen,
}

\newenvironment{note}%
    {\begin{mdframed}[style=trick]\begin{note_wrapper}}%
    {\end{note_wrapper}\end{mdframed}}

% Table stuff
\usepackage{tabularx} % tabularx since the width should be handled automatically
\usepackage{booktabs}

% Graph stuff
\usepackage{pgfplots}

% Miscellaneous
\usepackage{hyperref}
\hypersetup{colorlinks=true, urlcolor=blue, linkcolor=blue, citecolor=blue}

\usepackage{enumitem}
\setitemize{itemsep=0.5pt, topsep=0pt}
\setenumerate{itemsep=0.75pt, topsep=0pt}

\usepackage{graphics}

\newlist{exanswers}{itemize}{2}
\setlist[exanswers]{itemsep=2pt, topsep=2pt}
\setlist[exanswers,1]{label=$\diamond$,leftmargin=5mm}
\setlist[exanswers,2]{label=\textbullet,leftmargin=1mm}

% Custom commands
\newcommand{\R}{\mathbb{R}}
\newcommand{\Q}{\mathbb{Q}}
\newcommand{\N}{\mathbb{N}}
\newcommand{\Z}{\mathbb{Z}}
\newcommand{\C}{\mathbb{C}}
\newcommand{\BO}{\mathcal{O}}
\renewcommand{\labelenumii}{\arabic{enumi}.\arabic{enumii}}

% Metadata
\title{Analysis II Summary}
\author{Nicola Studer \\ \href{mailto:nicstuder@student.ethz.ch}{nicstuder@student.ethz.ch}}
\date{\vspace{-5ex}}


% ------------------------
% Document
% ------------------------

\begin{document}
\maketitle

\section{Ordinary differential equations}
\[F(x, y(x), y'(x), \ldots, y^{(n)}) = 0\]
Given a function \(F\) of \(x, y\), where \(x\) and \(y\) are functions themselves. \(F\) is an implicit ODE of \textbf{order} \(n\).

\begin{note}[Linear ODE's]
    Linear ODE with \(a_{k-1}, \ldots, a_0, b\) as cont. functions of \(x\) in \(I \subset \R\):
    \[y^{(k)} + a_{k-1}(x)y^{(k-1)} + \ldots + a_1(x)y' + a_0(x)y = b(x)\]
    If \(\bm{b = 0}\) then the ODE is called \textbf{homogeneous}.
\end{note}

\begin{ntheorem}[Properties of linear ODEs]
    \(\;\)
    \begin{enumerate}
        \item no coefficients before the highest derivative
        \item all coefficients are continuous functions
        \item no products of \(y\) and its derivatives
        \item no powers of \(y\) and its derivatives
        \item no functions which depend on \(y\) or its derivatives
    \end{enumerate}
\end{ntheorem}

\begin{theorem}[Main result about linear ODEs]
    \(\;\)
    \begin{enumerate}
        \item Let \(\mathcal{S}_0\) be the set of solutions when \(b = 0\). Then \(\mathcal{S}_0\) is a vector space of dimension \(k\). If \(f_1, \ldots, f_k\) are the solutions, then so is \(a_1f_1+ \ldots a_kf_k\).
        \item For any \textbf{initial condition} (i.e. for any choice of \(x_0 \in I\)) there is a unique solution \(f \in \mathcal{S}_0\) such that: \\
        \(f(x_0) = y_0, f'(x_0) = y_1, \ldots, f^{(k-1)}(x_0) = y_{k-1}\)
        \item For any arbitrary \(b(x)\), the set of solutions of the ODE is \(\mathcal{S}_b = \{f + f_p \ | \ f \in \mathcal{S}_0\}\) where \(f_p\) is a particular solution of the ODE.
        \item For any initial condition there is a unique solution \(f \in \mathcal{S}_b\).
    \end{enumerate}
\end{theorem}

\subsection{Linear ODE of order 1}
\[y' + a(x)y = b(x)\]
\begin{note}[How to solve]
    \(\;\)
    \begin{enumerate}
        \item Solve the homogeneous ODE:
        \begin{flalign*}
            && y' + ay &= 0 & \\
            &\implies & y' = - ay & \\
            &\implies & \tfrac{y'}{y} &= -a & (\text{assume \(y \neq 0\) no \(I\)}) \\
            &\implies & \ln(|y|) &= -A + C & (A(x) = \smallint a(x) \,dx) \\
            &\implies & y &= e^{-A + C} = z\cdot e^-A & (\text{simplify})
        \end{flalign*}
        \item Find a particular solution \(f_p: I \to \C\) with either an educated guess or variation of parameters such that \(f'_p + af_p = b\).
    \end{enumerate}
\end{note}
\end{document}
