\documentclass[a4paper, 11pt]{article}

% ------------------------------------------------------------------
% Imports
% ------------------------------------------------------------------

% Formatting
\usepackage[landscape, left=.2cm, top=.2cm, right=.2cm, bottom=.2cm]{geometry}
\usepackage{flowfram}
\usepackage[compact]{titlesec}
\usepackage{parskip}
\usepackage{mathptmx} % times font
\usepackage{mathtools}

% Language stuff
\usepackage[english]{babel}
\usepackage[utf8]{inputenc}

% Math imports
\usepackage{amsthm}
\usepackage{amssymb}
\usepackage{amsmath}
\usepackage{bm}
\usepackage{physics}

% Tables
\usepackage{tabularx} % tabularx since the width should be handled automatically

% Graphics
\usepackage{graphics}

% Miscellaneous
\usepackage{hyperref}
\usepackage{enumerate}
\usepackage[inline]{enumitem}
\usepackage{multicol}
\usepackage{xcolor}
\usepackage{etoolbox}
\usepackage{tikz}
\usetikzlibrary{positioning, shapes.multipart, fit}

% ------------------------------------------------------------------
% Formatting
% ------------------------------------------------------------------

% Colors
\definecolor{accent}{HTML}{2563eb}
\definecolor{H1}{HTML}{2ecc71}
\definecolor{H2}{HTML}{3498db}
\definecolor{H3}{HTML}{9b59b6}
\definecolor{H4}{HTML}{e74c3c}
\definecolor{H5}{HTML}{95a5a6}

% Column format
\setlength{\columnsep}{3pt}
\ffvadjustfalse
\Ncolumn{3}
\setlength{\parindent}{0pt}
\setlength{\parskip}{0cm}

% Compact titles
\newcommand{\colorsection}[2]{\colorbox{#1}{\parbox{\dimexpr\linewidth-2\fboxsep}{\centering\ #2}}}
\titlespacing{\section}{0pt}{0pt}{0pt}
\titleformat{\section}{\bfseries\color{white}}{}{0pt}{\colorsection{accent}}

\titlespacing{\subsection}{0pt}{0pt}{0pt}
\titleformat{\subsection}{\bfseries\color{white}}{}{0pt}{\colorsection{black!60}}

% Compact math mode
\makeatletter
\g@addto@macro\normalsize{%
  \setlength{\abovedisplayskip}{0pt}
  \setlength{\belowdisplayskip}{0pt}
  \setlength{\abovedisplayshortskip}{0pt}
  \setlength{\belowdisplayshortskip}{0pt}
  \setlength{\jot}{0pt}
}
\let\displaystyle\textstyle
\makeatother

% Set each display in math mode to be rendered as \textstyle
\renewcommand{\[}{\phantom{}\begin{center}\(}
\renewcommand{\]}{\)\end{center}}

% frames
\usepackage{tcolorbox}
\newenvironment{colored}
{\begin{tcolorbox}[colback=H2!10, colframe=white, boxrule=0.0pt,
                   enlarge top by=-0.1cm, enlarge bottom by=-0.1cm, 
                   enlarge left by=0cm, left=6pt, right=6pt,
                   boxsep=-1.5mm, outer arc=1pt,
                   arc=1pt]}
{\end{tcolorbox}}


% Misc
\hypersetup{colorlinks=true, urlcolor=accent, linkcolor=accent, citecolor=accent}
\setlist{itemsep=0.2pt, topsep=0.5pt, leftmargin=5mm}
\renewcommand{\labelenumii}{\arabic{enumi}.\arabic{enumii}}


% definition environment
\newtheoremstyle{compact_definition}{}{}{\normalfont}{}{\bfseries}{}{0em}{\thmnote{#3}: }
\theoremstyle{compact_definition}
\newtheorem*{definition}{Definition}

% ------------------------------------------------------------------
% Custom Commands
% ------------------------------------------------------------------

% Math general
\newcommand{\R}{\mathbb{R}}
\newcommand{\Q}{\mathbb{Q}}
\newcommand{\N}{\mathbb{N}}
\newcommand{\OB}{\mathcal{O}}
\renewcommand{\iff}{\Leftrightarrow}
\renewcommand{\implies}{\Rightarrow}
% \newcommand{\Z}{\mathbb{Z}}
\newcommand{\C}{\mathbb{C}}
\newcommand{\Pow}{\mathcal{P}}
\newcommand{\argmin}[1]{\text{argmin}_{#1}}
\newcommand{\argmax}[1]{\text{argmax}_{#1}}

% Probability Stuff
\renewcommand{\O}{\Omega}
\renewcommand{\P}{\mathbb{P}}
\newcommand{\PM}{\vb{P}}
\newcommand{\E}{\mathbb{E}}
\newcommand{\F}{\mathcal{F}}
\newcommand{\B}{\mathcal{B}}
\newcommand{\D}{\mathcal{D}}
\newcommand{\I}{1}
\renewcommand{\L}{\mathcal{L}}
\newcommand{\GP}{GP}
\DeclareMathOperator{\KL}{KL}
\DeclareMathOperator{\Var}{\mathbb{V}}
\DeclareMathOperator{\Cov}{Cov}
\DeclareMathOperator{\Ent}{H}
\DeclareMathOperator{\Ber}{Ber}
\DeclareMathOperator{\Bin}{Bin}
\DeclareMathOperator{\NB}{NB}
\DeclareMathOperator{\Geom}{Geom}
\DeclareMathOperator{\Cat}{Cat}
\DeclareMathOperator{\Poisson}{Poisson}
\DeclareMathOperator{\Unif}{\mathcal{U}}
\DeclareMathOperator{\Exp}{Exp}
\DeclareMathOperator{\Normal}{\mathcal{N}}
\DeclareMathOperator{\Laplace}{Laplace}
\DeclareMathOperator{\Ga}{Ga}
\DeclareMathOperator{\Atom}{Atom}
\DeclareMathOperator{\DX}{\mathcal{X}}

% Common Vectors and Matrices
\newcommand{\x}{\vb{x}}
\renewcommand{\r}{\vb{r}}
\newcommand{\y}{\vb{y}}
\renewcommand{\u}{\vb{u}}
\renewcommand{\v}{\vb{v}}
\newcommand{\z}{\vb{z}}
\newcommand{\h}{\vb{h}}
\newcommand{\w}{\vb{w}}
\newcommand{\W}{\vb{W}}
\newcommand{\Y}{\vb{Y}}
\newcommand{\X}{\vb{X}}
\newcommand{\Z}{\vb{Z}}
\newcommand{\f}{\vb{f}}
\newcommand{\K}{\vb{K}}
\renewcommand{\H}{\vb{H}}
\renewcommand{\k}{\vb{k}}
\renewcommand{\S}{\vb*{\Sigma}}
\newcommand{\s}{\vb*{\sigma}}
\renewcommand{\TH}{\vb*{\theta}}
\newcommand{\m}{\vb*{\mu}}
\newcommand{\eps}{\vb*{\epsilon}}
\renewcommand{\I}{\vb{I}}
\newcommand{\0}{\vb{0}}

% Statistical stuff
\DeclareMathOperator{\MSE}{MSE}
\DeclareMathOperator{\MLE}{MLE}
\DeclareMathOperator{\ML}{ML}
\DeclareMathOperator{\MI}{I}

% Machine learning
\newcommand{\p}{\Theta}
\renewcommand{\i}{^{(i)}}
\renewcommand{\l}[1]{^{[#1]}}

% Misc
\newcommand{\sdots}{\ifmmode\mathinner{\ldotp\kern-0.2em\ldotp\kern-0.2em\ldotp}\else.\kern-0.13em.\kern-0.13em.\fi}
\newcommand{\todo}[1]{{\color{red}\textbf{TODO}: #1}}
\newcommand{\adv}[1]{{\color{green}#1}}
\newcommand{\disadv}[1]{{\color{red}#1}}
\newcommand{\ROT}{\text{ROT}_{180}}


% ------------------------
% Document
% ------------------------

\begin{document}

% Title
% \begin{center}
%   \large Machine Perception \\
%   \small \(\langle\)\href{mailto:nicstuder@student.ethz.ch}{nicstuder@student.ethz.ch}\(\rangle\)
% \end{center}

% Chapters
\section{Mathematical Foundations}

\subsection{Vector Calculus}

\subsection{Probability Theory}

\begin{definition}[MLE]
    \(\p^{*} = \argmax{\p} p(\y \mid \X, \p) = \argmax{\p} \prod p(y\i \mid \x\i, \p)\)\(= \argmax\p \sum \log p(y\i \mid \x\i, \p)\) (Log-Likelihood)
\end{definition}

\begin{colored}
    \begin{enumerate}
        \item \(p(\y \mid \X, \p) = \Normal(y \mid \theta^\top \x, \sigma) \implies \p^* = \argmin{\theta} \norm{\theta^\top \x - y}_2^2\)
        \item \(p(\y \mid \X, \p)\) Laplace \(\implies \argmin\theta \norm{\theta^\top \x - y}_1\)
        \item Assume \(\p\) Gauss, then MAP is Ridge
        \item Assume \(\p\) Laplace, then MAP is LASSO
    \end{enumerate}

    \begin{itemize*}
        \item \(\p_{\MLE}\) unbiased and lowest variance. 
        \item \(\p_{\MLE} \overset{N \to \infty}{\to}\) true \(\p\).
    \end{itemize*}
\end{colored}


\begin{definition}[NLL]
    \(\L(\p) = - p(\y \mid \X, \p)\)
\end{definition}

\begin{definition}[BCE]
    Assume \(y\i \sim \Ber(\sigma(\theta^\top \x\i))\), i.e. \(p(\hat y) = p^{\hat y}(1 - p)^{1 - \hat y}\), then \(\L(\theta) = - \frac 1 N \sum y\i \log \hat y\i + ( 1 - y\i) \log(1 - \hat y\i)\).
\end{definition}

\begin{definition}[Softmax NLL]
    \(\L(\theta) = - \sum_i^k y\i \log (\hat y\i)\) (Generalized BCE)
\end{definition}

\subsection{Activation Functions}

\begin{definition}[Sigmoid]
    \(\sigma(x) = \frac{1}{1 + e^{-x}} = \frac{e^x}{e^x + 1}\), \(\partial_x \sigma(x) = \sigma(x) \odot (1 - \sigma(x))\)
    \adv{Mapping to prob space.}
    \disadv{Diminishing gradients}
\end{definition}

\begin{definition}[Tanh]
    \(\tanh(x) = \frac{e^x - e^{-x}}{e^x + e^{-x}} = 2\sigma(2x) - 1\), \(\partial_x \tanh(x) = 1 - \tanh^2(x)\)
    \adv{Better than \(\sigma\), as close to identity}
\end{definition}

\begin{definition}[Relu]
    \(f(x) = \max(0, x)\), \(\partial_x f(x) = 0\), if \(x < 0\), else \(1\)
\end{definition}

\subsection{Regularizatoin}
Any modification we make intended to reduce generalizaiton error, but not training error.
\begin{center}
    \(\tilde \L(\p, \X, \y) = \L(\p; \X, \y) + \lambda \Omega(\p)\)
\end{center}

\begin{definition}[Ridge]
    \(\Omega(\p) = \frac{1}{2} \norm{\p}_2^2 \to \p_t = (1 - \alpha \lambda)\p - \alpha \nabla_\p \L(\p; \X, \y)\) (weight decay).
\end{definition}

\begin{definition}[Lasso]
    \(\Omega(\p) = \norm{\p}_1 \to \nabla \tilde \L = \nabla \L + \lambda sign(\p)\) (sparse)
\end{definition}

\begin{definition}[Ensemble Methods]
    Train different models, acc. result.
    \begin{enumerate}
        \item Train diff. model classes (LR, Trees, NN)
        \item Train same model on diff. data
    \end{enumerate}
    Examples: dropout, bagging.
\end{definition}

\begin{definition}[Bagging (Bootstr. Agg.)]
    \(k\) models, \(k\) datasets w/ replacement; Agg. result. Independent models.
\end{definition}

\begin{definition}[Dropout]
    Rand. ign. neurons in training (\adv{robust}). In test, use weight scaling \(\tilde\p = p\p\). Models are dependent (share weights). Train small percentage of models.
\end{definition}

\begin{definition}[Data Normalization]
    Update params in similar scales. At test, use \(\mu, \sigma\) from training.
\end{definition}

\begin{definition}[Batch Normalization]
    Normalize weights for each layer. Solve internal covaraite shift (distribution of inputs changes as ealier layers change, gradient falsely assumes constancy). Smoothens loss landscape \(\to\) predictive and stable gradients \(\to\) enables higher LR \(\to\) faster convergence.
    Makes deeper layers more robust. Slight reg. effect since mean/variance introduce noise. During test, tracking running avg. of \(\mu, \sigma^2\).
    \(\tilde z_i = \gamma z_i^{\text{norm}} + \beta\), \(\z_i^{\text{norm}} = \frac{z_i - \mu}{\sqrt{\epsilon + \sigma^2}}\), \(\sigma^2 = \frac{1}{n-1}\sum_i(z_i - \mu)^2\) (\(\gamma, \beta\) learnable).
\end{definition}

\begin{definition}[Data Augmentation]
    Createa new fake data by augmenting existing. Exploit invariances (classification) and equivariances (regression) of target. Either \textit{geometric} or \textit{noise injective}.
\end{definition}

\begin{definition}[Transfer learning]
    If not sufficient data, train model on different task with more data available, then fine-tune on original task.
\end{definition}

\subsection{Optimizations}
\begin{definition}[GD]
    \(\p^{(t+1)} = \p^{(t)} - \eta \cdot \nabla_\p \L(\p)\)
\end{definition}

\begin{definition}[SGD]
    \(\p^{(t+1)} = \p^{(t)} - \eta \cdot \nabla_\p \L(\p; \x\i; y\i)\).
    \adv{Unbiased, high efficiency per iter, avoids local minima};
    \disadv{High \(\Var\); may overshoot; near min, dominated by stochastisity \(\to\) decrease \(\eta\) over time}.
\end{definition}

\begin{definition}[MB-GD]
    \(\p^{(t+1)} = \p^{(t)} - \frac 1 m \nabla_\p \sum_i \L(f(\x\i; \p), y\i)\).
    \adv{Faster than GD, reduces varainceof SGD}
\end{definition}

\begin{definition}[Polyak's Momentum]
    \(\p^{(t+1)} = \p^{(t)} + v\), \(v = \alpha v - \eta \nabla_\p\left(\frac 1 m \sum_i \L(f(x\i; \p), y\i)\right)\). If \(\alpha > \eta\), then previous gradient affect more \(\implies\) larger step size, for successive steps in same direction.
    \disadv{May not conerge.}
\end{definition}

\begin{definition}[Nesterov's Momentum]
    \(\p^{(t+1)} = \p + v\), \(v = \alpha v - \eta \nabla_\p\left(\frac 1 m \sum_i \L(f(x\i; \p + \alpha v), y\i)\right)\). Look ahead before taking a step.
\end{definition}

\begin{definition}[AdaGrad]
    Keep sum of past squared gradients (\(\nabla_2\)). Then \(\eta \frac{\nabla_\p}{\alpha \nabla_2} \).
\end{definition}

\begin{definition}[RMSprop]
    AdaGrad with exp. decay in \(\nabla_2\) to forget. 
\end{definition}

\begin{definition}[Adam]
    Momentum + Adaptive LR. \(\p^{(t+1)} = \p^t - \frac{\alpha m_t}{\sqrt{v_t} + \epsilon}\). \(m_t\) bias-corrected 1st order stimate of momentum. \(v_t\) bias corrected squared gradients.
\end{definition}

% TODO: Maybe add ensambling


\section{MLP and UAT}
Output layer \(l\): \(\z\l l = \sigma(\W\l l \z\l{l - 1})\).
\begin{itemize}
    \item Sens. to weights: \(\frac{\partial \L}{\partial w_{ij}\l{l}} = \frac{\partial \L}{\partial z_i\l l} \frac{\partial z_i\l l }{\partial w_{ij}\l l}\), \(\frac{\partial \L}{\partial \W\l l } = \frac{\partial \L}{\partial \z\l l } \frac{\partial \z \l l}{\partial \W \l l}\).
    \item Sens. to out.: \(\delta_i\l{l-1} = \frac{\partial \L}{\partial z_i\l{l-1}} = \sum_j \frac{\partial \L}{\partial z_j\l l} \frac{\partial z_j\l l}{\partial z_i \l{l-1}} = \sum_j \delta_j\l l \frac{\partial z_j\l l}{\partial z_i \l{l-1}}\).
    Or for full entire layer: \(\vb*{\delta}\l{l-1} = \frac{\partial \L}{\partial \z \l l} \frac{\partial \z\l l}{\partial \z \l {l-1}}\)
\end{itemize}

\begin{definition}[UAT]
    \(\sigma\) any act-f. bounded + cont. \(I_m\) \(m\)-dim hyperc and space of real f. on \(I_m\) is \(C(I_m)\). Then given \(\epsilon > 0\), \(N\) int, \(v_i, b_i \in \R\) and \(\w_i \in \R^m\) for \(i \in [N]\), then \(\forall f \in C(I_m) \forall \x \in I_m:\)
    \[f(\x) \approx \sum_{i=1}^N v_i \sigma(\w_i^\top \x + b_i) \ \text{and} \  \vert g(\x) - f(\x)\vert < \epsilon.\]
\end{definition}


\section{CNN}
\begin{definition}[Linearity]
    \(T(\alpha \u + \beta \v) = \alpha T(\u) + \beta T(\v)\)
\end{definition}

\begin{definition}[Invariance]
    \(T(f(\u)) = T(\u)\) (e.g. classification)
\end{definition}

\begin{definition}[Equivariance]
    \(T(f(\u)) = f(T(\u))\) (e.g. edge detection)
\end{definition}

Any linear, shift-equivariant transform can be a convolution.

\begin{definition}[Correlation]
    \(I'(i, j) = \sum_{m=-k}^k \sum_{n=-k}^k K(m, n) I(i+m, j+n)\)
\end{definition}

\begin{definition}[Convolution]
    \(I'(i, j) = \sum_{m=-k}^k \sum_{n=-k}^k I(i -m, j - n) K(m, n)\)

    \begin{itemize*}
        \item Corr \(\iff K(m, n) = K(-m, -n)\)
        \item Commutative
        \item Toepliz MM
    \end{itemize*}

    Fitler same depth as input map, but can stack multiple filters.
\end{definition}

\begin{definition}[CNN-formulas]
    \(\bm{C}\)han., \(\bm{K}\)er. s., \(m = \#\)Ker., \(\bm{{D}}\)ill., \(\bm{P}\)ad, \(\bm{S}\)tride
    \begin{itemize}
        \item Dim: \(f(W) \times f(H) \times m\), \(f(i) = \frac{i + 2P_i - D_i(K_i - 1) - 1}{S_i} + 1\)
        \item Params: \(p = (K_W \cdot K_H \cdot C + 1) \cdot m \ {\color{H3}+ 1 \hat{=} \ \text{Bias}}\).
        \item Recptive field increases with depth, stride and dilation.
    \end{itemize}
\end{definition}

\begin{definition}[Backprop]
    \(\delta\l{l-1}_{i, j} = \sum_{i'}\sum_{j'}\delta_{i', j'}\l l w_{i'-i, j'-j}\l l = \delta\l l \ast \ROT(W\l l)\)
\end{definition}

\begin{definition}[Update]
    \(\frac{\partial L}{\partial w_{m, n}\l l} = \sum_i \sum_j \delta_{i, j} \l l z_{i-m, j-n}\l{l -1} = \delta\l l \ast \ROT(\Z\l{l-1})\)
\end{definition}

\begin{definition}[Max-Pool]
    \(z_{i, j}\l l = \max\{Z_{is:is+k, js:js+k}\l{l-1}\}\) (\(s\) stride), \(\frac{\partial z_{i, j}\l l}{\partial z_{i', j'} \l{l-1}} = 1\) iff \((i',j') = (i^*, j^*)\) (max indices) and 0 otw. \(\delta\l{l-1} = \{\delta\l l\}_{i^*, j^*}\).
\end{definition}

\begin{definition}[Upsampling]
    Nearest Neighbor(fill all with values), Bed of nails (fill upper left with value, rest 0), Max Unpooling (remember max location), Learnable Upsampling (transpose conv: filter moves stride in output for 1 in input, sum overlap).
\end{definition}

\section{RNN}

Hidden state summarizing previous input. Learn \textbf{one} model:

\(\h^t = f(\h_{t-1}, x_t; \p)\): \(f\) transition f. Same \(f\) and \(\p\) over time.

\begin{definition}[Vanilla RNN]
    \(\h_t = \tanh(\W_{hh}\h_{t - 1} + \W_{xh}\x_t + b), \hat\y_t = \W_{hy} \h_t\).
    \begin{itemize*}
        \item \(\tanh\) because 0-center and higher grad norm \(\implies\) faster conv.
        \item Reparam: \(\h_t = \W_{hh} f(\h_{t-1}) + \W_{xh}\x_t + b\), \(L_t = \norm{\hat \y_t - \y_t}^2\)
    \end{itemize*}
\end{definition}

\begin{definition}[BPTT]
    \(\frac{\partial L}{\partial \W} = \sum_{t=1}^S \frac{\partial L_t}{\partial \W}\), 
    \(\frac{\partial \L_t}{\partial \W} = \sum_{k=1}^t \frac{\partial L_t}{\partial \hat\y_t} \frac{\partial \hat\y_t}{\partial \h_t} \frac{\partial \h_t}{\partial \h_k} \frac{\partial^+ \h_k}{\partial \W}\)

    \(\frac{\partial \h_t}{\partial \W} = \frac{\partial^+ \h_t}{\partial \W} + \frac{\partial \h_t}{\partial \h_{t-1}}\frac{\partial \h_{t-1}}{\partial \W} = \sum_{k=1}^t\frac{\partial \h_t}{\partial \h_k} \frac{\partial^+\h_k}{\partial \W}\)
    
    \(\frac{\partial \h_t}{\partial \h_k} = \prod_{i = k+1}^t \frac{\partial \h_i}{\partial \h_{i-1}} = \prod_{i = k+1}^t \W_{hh}^\top \text{diag}[f'(\h_{i-1})]\)
\end{definition}

\begin{definition}[Exploding/Vanishing Grad]
    Because \(\sigma'\) and \(\tanh'\) bounded, \(\norm{\text{diag}[f'(h_{i-1})]} < \gamma\) for some \(\gamma\). Then if \(\lambda_1 < \frac{1}{\gamma}\), we have vanishing and otherwise exploding gradients over many time steps.
\end{definition}

\begin{definition}[Trauncated BPTT]
    \(\frac{\partial L_t}{\partial \W} \approx \sum_{k = t - \kappa}^t \frac{\partial L_t}{\partial \hat\y_t} \frac{\partial \hat\y_t}{\partial \h_t}\frac{\partial \h_t}{\partial h_k}\frac{\partial^+ \h_k}{\partial \W}\).
\end{definition}

\begin{definition}[Gradient Clipping]
    If \(\norm{\vb{g}} \geq\), then \(\vb{g} = \frac{\tau}{\norm{\vb{g}}} \vb{g}\).
\end{definition}

\begin{definition}[LSTM]
Keep separate memory cell (state) with gated (protected) access. Can forget. All learned. Computationally expensive. GRU presents a more efficient simplification.
Input Gate (which values to write), Forget Gate (which values to forget), Output Gate (which values to output), Gate Gate (candidate values).
\[
\mathbf{
    \begin{bmatrix}
        \mathbf{i}_t \\
        \mathbf{f}_t \\
        \mathbf{o}_t \\
        \mathbf{g}_t
    \end{bmatrix}
} = 
\begin{bmatrix}
    \sigma \\
    \sigma \\
    \sigma \\
    \tanh
\end{bmatrix}
\left( W
\begin{bmatrix}
    \mathbf{x}_t {\color{gray}\h_t\l{l-1}}\\
    \mathbf{h}_{t-1} {\color{gray}\h_{t-1}\l l}
\end{bmatrix}
\right)
\quad
\begin{aligned}
\mathbf{c}_t &= \mathbf{f}_t \odot \mathbf{c}_{t-1} + \mathbf{i}_t \odot \mathbf{g}_t \\
\mathbf{h}_t &= \mathbf{o}_t \odot \tanh(\mathbf{c}_t)
\end{aligned}
\]

\end{definition}

\section{Autoencoder}
\begin{center}
    Input \(\quad\overset{f}{\underset{\text{encoder}}{\to}}\quad\) Latent Space \(\quad\overset{g}{\underset{\text{decoder}}{\to}}\quad\) Reconstruction
\[\hat{\p}_f, \hat{\p}_g = \argmin{\p} \sum \norm{\x_i - g(f(\x_i; \p_f); \p_g)}^2\]
\end{center}

\begin{definition}[Undercomplete]
    \(\dim(Z) < \dim(X)\): learn important features
\end{definition}

\begin{definition}[Overcomplete]
    \(\dim(Z) > \dim(X)\): used for denoising and impaiting. Enter noise/non-complete input and compare to noiseless/complete output.
\end{definition}

\begin{definition}[AE Limitations]
    Latent space not well structured, lacks continuity, good reconstruction, but bad generator.
\end{definition}

\subsection{Variational Autoencoder}
\textbf{Idea}: Latent space is continuous, by splitting it into \(\m\) and \(\s\).
Choose \(p(\z)\) to be simple distribution and NN to train \(p(x \mid z)\).
\begin{tabular}{@{}lll@{}}
 & \textbf{SL} & \textbf{VAE} \\
LH & $p(y \mid x, \theta)$ & $p_\theta(x \mid z)$ \\
Pri. & $p(\theta)$ & $p(z)$ \\
Post. & $p(\theta \mid X, y)$ & $p_\theta(z \mid x) \approx q_\phi(z \mid x)$ \\
MLH & $p(y \mid X) {\scriptstyle = \int p(\theta) p(y \mid X, \theta) d\theta}$ & $p_\theta(x) = \int p_\theta(x \mid z) p(z) dz$ \\
\end{tabular}

\begin{definition}[Objective]
    Find \(\p^*\) which maximizes Marginal Likelihood: \\ \(p_\p(x) = \int p(x \mid z) p(z) \, dz\) (intractable) \(\implies p(z \mid x)\) intractable.

    \(\implies\) use surrogate loss with approximate \(q_\phi(z \mid x) \approx p(z \mid x)\).

\end{definition}

\begin{definition}[Training]
    Maximize \(p_\p(x) \iff \) maximize \(\log p_\p(x)\):
    \begin{gather*}
        \log(p_\p(x)) = \E_{z \sim q_\phi(z \mid x)}[\log p_\p(x)] \\
        = \E_{z \sim q_\phi(z \mid x)}\left[\log \frac{p_\p(x \mid z) p(z)}{p_\p(z \mid x)} \frac{q_\phi(z \mid x)}{q_\phi(z \mid x)}\right] \\
        = \E_z[\log p_\p(x \mid z)] - \E_{z\mid x}\left[\log \frac{q_\phi(z \mid x)}{p_\p(z)}\right] + \E_{z \mid x}\left[\log\frac{q_\phi(z \mid x)}{p_\p(z \mid x)}\right]
    \end{gather*}
    \resizebox{\linewidth}{!}{
        \(= \underbrace{\E_z[\log p_\p(x \mid z)]}_{\substack{\text{reconstruction}\\\text{should match GT}}} - \underbrace{D_{KL}(q_\phi(z \mid x) \Vert p(z)))}_{\substack{\text{posterior should}\\ \text{be close to prior}}} + \underbrace{D_{KL}[q_\phi(z \mid x) \vert p_\p(z \mid x)]}_{\text{intractable, but \(\geq\) 0}}\)
    }
    \resizebox{\linewidth}{!}{
        \(\geq \underbrace{\E_z[\log p_\p(x \mid z)]}_{\text{Maximize}} - \underbrace{D_{KL}(q_\phi(z \mid x) \Vert p(z))}_{\text{Minimize}} = \text{ELBO} = -L(x, \theta, \phi)\)
    }
    \begin{itemize}
        \item Only decoder recon. likelihood: sharp recon, sparse latent.
        \item Only Prior KL: compact, bad recon. enforces smooth interp.
    \end{itemize}
\end{definition}

\begin{definition}[Reparameterization]
    Sampling problematic for backfrop: \(\z \sim \Normal(\m, \s)\), cannot take gradient w.r.t \(\m, \s\). Hence assume underlying \(\eps \sim \Normal(0, 1)\), then \(\z = \m + \s\eps\). Assuming \(\eps\) fixed enables backprop.
\end{definition}

\begin{definition}[Training]
    \begin{enumerate*}
        \item \(\text{Enc}(\x) = (\m, \s)\)
        \item Sample \(\eps\)
        \item \(\z = \m + \eps\s\)
        \item \(\text{Dec}(\z)\)
        \item Comp. \(L\) and Backprop.
    \end{enumerate*}
\end{definition}

\begin{definition}[\(\beta\)-VAE]
    \(L = - \E_{z \sim q_\phi(z \mid x)}[\log p_\theta(x \mid z)] + \beta D_{KL}(q\phi(z \mid x)\Vert p(z))\).

    Learn disentangled rep. w/o supervis. AE(\(\beta = 0\)), VAE \(\beta = 1\).
\end{definition}


\section{Autoregressive Models}
\textbf{Explicit density model}: \(p(\x) = \prod p(x_i \mid \x_{<i})\) which we want to maximize. \(p(\x)\)
\adv{tractable \(\implies\) easy to train and sample.}
\disadv{No natural latent repr., slow inference, tricky to train fast}.

\begin{definition}[FVSBN]
    Model \(f_i(\x_{<i}) = \sigma(\alpha_0^i + \alpha_1^ix_1 + \ldots + \alpha_{i-1}^i x_{i-1})\) via logistic regression. Thus \(\p = (\alpha_1, \ldots \alpha_{i-1})\) (\(\frac{n^2 + n}{2}\) params).
    
    \adv{simple, tractable, easy to train}.
    \disadv{no latent space, slow to enerate, not very expressive}.
\end{definition}

\begin{definition}[NADE]
    AE like for binary data.
    \(\h_i = \sigma(\vb{b} + \W_{:, <i \x_{<i}})\), \(\hat x_i = \sigma(c_i + \vb{V}_{i, :} \h_i)\).
    Training by maximizing average log-likelihood.
    During training, take samples from taining data. During inference, use prev. predictions.

    \adv{Effcient, optimizable, extend beyond binary}.
\end{definition}

\begin{definition}[MADE]
    AE, enforce AR property with random masking.
    \adv{Training same complexity as autoencoder}.
    \disadv{Inference longer because of seq. generation}.
    Negative log-likelihood for binary \(\x\).
\end{definition}

\begin{definition}[PixelRNN]
    Start at corner, generate pixels in some order using RNN (LSTM). \disadv{Seq. generation is slow}.
\end{definition}

\begin{definition}[PixelCNN]
    Start at corner, only condition on context region (\adv{fast training}). Masking for AR property. Explicit likelihood (\adv{good eval}, \disadv{slow inference}).
\end{definition}

\begin{definition}[Wavenet]
    Temporal dependencies w/ Dilated Convolutions.
\end{definition}

\begin{definition}[Transformer]
    Key \(K= XW_k\), Value \(V = XW_v\), Query \(Q = XW_q\) with learnable \(W_k, W_v, W_q \in \R^{D \times D}\) and \(X \in \R^{T \times D}\).
    Attention \(\alpha = \text{softmax}\left(\frac{QK^\top}{\sqrt D}\right)\), \(X = (\alpha + M)V\) with \(M = \begin{smallmatrix}-\infty & -\infty \\ \0 & -\infty\end{smallmatrix}\) mask for AR.
    Apply pos. encoding cause vanially attention doesn't give any notion of sequence.
    Complexity in \(\OB(T^2D)\).
\end{definition}

\section{Normalizing Flows}
Let \(\z\) lat. var. and obs. \(\x = f(\z)\) were \(f\) invertible \& cont. differentiable, then \(d\x = \left|\text{det}\frac{\partial f(\z)}{\partial \z}\right|d\z\) and hence we express:
\[p_{\x}(\x) = p_{\z}(f^{-1}(\x))\left|\text{det}\frac{\partial f^{-1}(\x)}{\partial \x}\right| = p_{\z}(f^{-1}(\x))\left|\text{det}\frac{\partial f(\z)}{\partial \z}\right|^{-1}\]
Note that for invertible matrices \(|\det(A^{-1})| = |\det(A)|\).

\textbf{Idea}: \(f\) as NN, which maps simple dist to complex w/ simple MLP. \textit{Needs}: diff., inv., preserve dim., Jac. comp. efficient.

\begin{definition}[Coupling Layers]
    Split \(\x \to \x^A, \x^B\) \textbf{randomly}. Have \textit{element-wise} \(h\) and arbitrary complex \(\beta\) (e.g. CNN). Then
    \[\textbf{FP}\begin{pmatrix}
        \y^A \\ \y^B
    \end{pmatrix} = \begin{pmatrix}
        h(\x^A, \beta(\x^B)) \\ \x^B
    \end{pmatrix} \,
    \textbf{BP} \begin{pmatrix}
        \x^A \\ \x^B
    \end{pmatrix} = \begin{pmatrix}
        h^{-1}(\y^A, \beta(\y^B)) \\ \y^B
    \end{pmatrix}\]
    and \(J = \left(\begin{smallmatrix}
        h' & h'\beta' \\ \0 &\I
    \end{smallmatrix}\right)\) and thus very easy to ompute it's determinant.
\end{definition}

\begin{definition}[Flow]
    Get complex output
    \(\x = f(\z) = f_k \circ \ldots \circ f_1(\z)\) from simple dist. \(\z\) and use \(p_{\x}(\x) = p_{\z}(f^{-1}(\x)) \prod_k \left|\det\left(\partial \frac{f_k^{-1}(\x)}{\partial \x}\right)\right|\).
\end{definition}

\begin{definition}[Training]
    Use NLL loss with i.i.d. samples over likelihood:
    \[\log p_{\x}(\mathcal D) = \sum_{\x \in \mathcal D} \log p_{\z}(f^{-1}(\x)) + \sum_k \log \left|\det \frac{\partial f_k^{-1}(\x)}{\partial \x}\right|\]
\end{definition}

\begin{definition}[Inference]
    To get \(p_{\x}(\x)\), use inv. \(\z = f^{-1}(\x)\) and calc \(p_{\z}(\z)\).
\end{definition}

\adv{
    \begin{itemize*}
        \item Can evaluate likelihood of new observation \(\x\) (can't with VAE or GAN)
        \item Invertibility
        \item Bijective mapping
        \item \textbf{Exact} LH
    \end{itemize*}
}

\begin{definition}[Conditional Coupling]
    \(\beta(\x^B, \w)\) for arbitrary conditional \(\w\).
\end{definition}

\begin{definition}[NICE]
    Additive coupling, swapping, single scale
\end{definition}

\begin{definition}[RealNVP]
    Affine coupling, checkboard split, multi-scale.

    \textit{Additive}: \(\y^A = \x^A + h(\x^B)\), \textit{Affine}: \(\y^A = \x^A \odot \exp(s(\x^B)) + t(\x^B)\) (\(s, t\) NN)
\end{definition}

\begin{definition}[GLOW]
    Affine coupling, inv. 1x1 conv, multi-scale. \(L\) levels of \(K\) steps (actnorm, inv 1x1 conv, coupling layer).
\end{definition}



\end{document}
