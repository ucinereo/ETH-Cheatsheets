\documentclass[a4paper, 11pt]{article}

% ------------------------------------------------------------------
% Imports
% ------------------------------------------------------------------

% Formatting
\usepackage[landscape, left=.2cm, top=.2cm, right=.2cm, bottom=.2cm]{geometry}
\usepackage{flowfram}
\usepackage[compact]{titlesec}
\usepackage{parskip}
\usepackage{mathptmx} % times font
\usepackage{mathtools}

% Language stuff
\usepackage[english]{babel}
\usepackage[utf8]{inputenc}

% Math imports
\usepackage{amsthm}
\usepackage{amssymb}
\usepackage{amsmath}
\usepackage{bm}
\usepackage{physics}

% Tables
\usepackage{tabularx} % tabularx since the width should be handled automatically

% Graphics
\usepackage{graphics}

% Miscellaneous
\usepackage{hyperref}
\usepackage{enumerate}
\usepackage[inline]{enumitem}
\usepackage{multicol}
\usepackage{xcolor}
\usepackage{etoolbox}
\usepackage{tikz}
\usetikzlibrary{positioning, shapes.multipart, fit}

% ------------------------------------------------------------------
% Formatting
% ------------------------------------------------------------------

% Colors
\definecolor{accent}{HTML}{2563eb}
\definecolor{H1}{HTML}{2ecc71}
\definecolor{H2}{HTML}{3498db}
\definecolor{H3}{HTML}{9b59b6}
\definecolor{H4}{HTML}{e74c3c}
\definecolor{H5}{HTML}{95a5a6}

% Column format
\setlength{\columnsep}{3pt}
\ffvadjustfalse
\Ncolumn{3}
\setlength{\parindent}{0pt}
\setlength{\parskip}{0cm}

% Compact titles
\newcommand{\colorsection}[2]{\colorbox{#1}{\parbox{\dimexpr\linewidth-2\fboxsep}{\centering\ #2}}}
\titlespacing{\section}{0pt}{0pt}{0pt}
\titleformat{\section}{\bfseries\color{white}}{}{0pt}{\colorsection{accent}}

\titlespacing{\subsection}{0pt}{0pt}{0pt}
\titleformat{\subsection}{\bfseries\color{white}}{}{0pt}{\colorsection{black!60}}

% Compact math mode
\makeatletter
\g@addto@macro\normalsize{%
  \setlength{\abovedisplayskip}{0pt}
  \setlength{\belowdisplayskip}{0pt}
  \setlength{\abovedisplayshortskip}{0pt}
  \setlength{\belowdisplayshortskip}{0pt}
  \setlength{\jot}{0pt}
}
\let\displaystyle\textstyle
\makeatother

% Set each display in math mode to be rendered as \textstyle
\renewcommand{\[}{\phantom{}\begin{center}\(}
\renewcommand{\]}{\)\end{center}}

% frames
\usepackage{tcolorbox}
\newenvironment{colored}
{\begin{tcolorbox}[colback=H2!10, colframe=white, boxrule=0.0pt,
                   enlarge top by=-0.1cm, enlarge bottom by=-0.1cm, 
                   enlarge left by=0cm, left=6pt, right=6pt,
                   boxsep=-1.5mm, outer arc=1pt,
                   arc=1pt]}
{\end{tcolorbox}}


% Misc
\hypersetup{colorlinks=true, urlcolor=accent, linkcolor=accent, citecolor=accent}
\setlist{itemsep=0.2pt, topsep=0.5pt, leftmargin=5mm}
\renewcommand{\labelenumii}{\arabic{enumi}.\arabic{enumii}}


% definition environment
\newtheoremstyle{compact_definition}{}{}{\normalfont}{}{\bfseries}{}{0em}{\thmnote{#3}: }
\theoremstyle{compact_definition}
\newtheorem*{definition}{Definition}

% ------------------------------------------------------------------
% Custom Commands
% ------------------------------------------------------------------

% Math general
\newcommand{\R}{\mathbb{R}}
\newcommand{\Q}{\mathbb{Q}}
\newcommand{\N}{\mathbb{N}}
\newcommand{\OB}{\mathcal{O}}
\renewcommand{\iff}{\Leftrightarrow}
\renewcommand{\implies}{\Rightarrow}
% \newcommand{\Z}{\mathbb{Z}}
\newcommand{\C}{\mathbb{C}}
\newcommand{\Pow}{\mathcal{P}}
\newcommand{\argmin}[1]{\text{argmin}_{#1}}
\newcommand{\argmax}[1]{\text{argmax}_{#1}}

% Probability Stuff
\renewcommand{\O}{\Omega}
\renewcommand{\P}{\mathbb{P}}
\newcommand{\PM}{\vb{P}}
\newcommand{\E}{\mathbb{E}}
\newcommand{\F}{\mathcal{F}}
\newcommand{\B}{\mathcal{B}}
\newcommand{\D}{\mathcal{D}}
\newcommand{\I}{1}
\renewcommand{\L}{\mathcal{L}}
\newcommand{\GP}{GP}
\DeclareMathOperator{\KL}{KL}
\DeclareMathOperator{\Var}{\mathbb{V}}
\DeclareMathOperator{\Cov}{Cov}
\DeclareMathOperator{\Ent}{H}
\DeclareMathOperator{\Ber}{Ber}
\DeclareMathOperator{\Bin}{Bin}
\DeclareMathOperator{\NB}{NB}
\DeclareMathOperator{\Geom}{Geom}
\DeclareMathOperator{\Cat}{Cat}
\DeclareMathOperator{\Poisson}{Poisson}
\DeclareMathOperator{\Unif}{\mathcal{U}}
\DeclareMathOperator{\Exp}{Exp}
\DeclareMathOperator{\Normal}{\mathcal{N}}
\DeclareMathOperator{\Laplace}{Laplace}
\DeclareMathOperator{\Ga}{Ga}
\DeclareMathOperator{\Atom}{Atom}
\DeclareMathOperator{\DX}{\mathcal{X}}

% Common Vectors and Matrices
\newcommand{\x}{\vb{x}}
\renewcommand{\r}{\vb{r}}
\newcommand{\y}{\vb{y}}
\newcommand{\z}{\vb{z}}
\newcommand{\w}{\vb{w}}
\newcommand{\W}{\vb{W}}
\newcommand{\Y}{\vb{Y}}
\newcommand{\X}{\vb{X}}
\newcommand{\Z}{\vb{Z}}
\newcommand{\f}{\vb{f}}
\newcommand{\K}{\vb{K}}
\renewcommand{\H}{\vb{H}}
\renewcommand{\k}{\vb{k}}
\renewcommand{\S}{\vb*{\Sigma}}
\renewcommand{\TH}{\vb*{\theta}}
\newcommand{\m}{\vb*{\mu}}
\newcommand{\eps}{\vb*{\epsilon}}
\renewcommand{\I}{\vb{I}}
\newcommand{\0}{\vb{0}}

% Statistical stuff
\DeclareMathOperator{\MSE}{MSE}
\DeclareMathOperator{\MLE}{MLE}
\DeclareMathOperator{\ML}{ML}
\DeclareMathOperator{\MI}{I}

% Machine learning
\newcommand{\p}{\Theta}
\renewcommand{\i}{^{(i)}}
\renewcommand{\l}[1]{^{[#1]}}

% Misc
\newcommand{\sdots}{\ifmmode\mathinner{\ldotp\kern-0.2em\ldotp\kern-0.2em\ldotp}\else.\kern-0.13em.\kern-0.13em.\fi}
\newcommand{\todo}[1]{{\color{red}\textbf{TODO}: #1}}
\newcommand{\adv}[1]{{\color{green}#1}}
\newcommand{\disadv}[1]{{\color{red}#1}}


% ------------------------
% Document
% ------------------------

\begin{document}

% Title
% \begin{center}
%   \large Machine Perception \\
%   \small \(\langle\)\href{mailto:nicstuder@student.ethz.ch}{nicstuder@student.ethz.ch}\(\rangle\)
% \end{center}

% Chapters
\section{Mathematical Foundations}

\subsection{Vector Calculus}

\subsection{Probability Theory}

\begin{definition}[MLE]
    \(\p^{*} = \argmax{\p} p(\y \mid \X, \p) = \argmax{\p} \prod p(y\i \mid \x\i, \p)\)\(= \argmax\p \sum \log p(y\i \mid \x\i, \p)\) (Log-Likelihood)
\end{definition}

\begin{colored}
    \begin{enumerate}
        \item \(p(\y \mid \X, \p) = \Normal(y \mid \theta^\top \x, \sigma) \implies \p^* = \argmin{\theta} \norm{\theta^\top \x - y}_2^2\)
        \item \(p(\y \mid \X, \p)\) Laplace \(\implies \argmin\theta \norm{\theta^\top \x - y}_1\)
        \item Assume \(\p\) Gauss, then MAP is Ridge
        \item Assume \(\p\) Laplace, then MAP is LASSO
    \end{enumerate}

    \begin{itemize*}
        \item \(\p_{\MLE}\) unbiased and lowest variance. 
        \item \(\p_{\MLE} \overset{N \to \infty}{\to}\) true \(\p\).
    \end{itemize*}
\end{colored}


\begin{definition}[NLL]
    \(\L(\p) = - p(\y \mid \X, \p)\)
\end{definition}

\begin{definition}[BCE]
    Assume \(y\i \sim \Ber(\sigma(\theta^\top \x\i))\), i.e. \(p(\hat y) = p^{\hat y}(1 - p)^{1 - \hat y}\), then \(\L(\theta) = - \frac 1 N \sum y\i \log \hat y\i + ( 1 - y\i) \log(1 - \hat y\i)\).
\end{definition}

\begin{definition}[Softmax NLL]
    \(\L(\theta) = - \sum_i^k y\i \log (\hat y\i)\) (Generalized BCE)
\end{definition}

\subsection{Activation Functions}

\begin{definition}[Sigmoid]
    \(\sigma(x) = \frac{1}{1 + e^{-x}} = \frac{e^x}{e^x + 1}\), \(\partial_x \sigma(x) = \sigma(x) \odot (1 - \sigma(x))\)
    \adv{Mapping to prob space.}
    \disadv{Diminishing gradients}
\end{definition}

\begin{definition}[Tanh]
    \(\tanh(x) = \frac{e^x - e^{-x}}{e^x + e^{-x}} = 2\sigma(2x) - 1\), \(\partial_x \tanh(x) = 1 - \tanh^2(x)\)
    \adv{Better than \(\sigma\), as close to identity}
\end{definition}

\begin{definition}[Relu]
    \(f(x) = \max(0, x)\), \(\partial_x f(x) = 0\), if \(x < 0\), else \(1\)
\end{definition}

\subsection{Regularizatoin}
Any modification we make intended to reduce generalizaiton error, but not training error.
\begin{center}
    \(\tilde \L(\p, \X, \y) = \L(\p; \X, \y) + \lambda \Omega(\p)\)
\end{center}

\begin{definition}[Ridge]
    \(\Omega(\p) = \frac{1}{2} \norm{\p}_2^2 \to \p_t = (1 - \alpha \lambda)\p - \alpha \nabla_\p \L(\p; \X, \y)\) (weight decay).
\end{definition}

\begin{definition}[Lasso]
    \(\Omega(\p) = \norm{\p}_1 \to \nabla \tilde \L = \nabla \L + \lambda sign(\p)\) (sparse)
\end{definition}

\begin{definition}[Ensemble Methods]
    Train different models, acc. result.
    \begin{enumerate}
        \item Train diff. model classes (LR, Trees, NN)
        \item Train same model on diff. data
    \end{enumerate}
    Examples: dropout, bagging.
\end{definition}

\begin{definition}[Bagging (Bootstr. Agg.)]
    \(k\) models, \(k\) datasets w/ replacement; Agg. result. Independent models.
\end{definition}

\begin{definition}[Dropout]
    Rand. ign. neurons in training (\adv{robust}). In test, use weight scaling \(\tilde\p = p\p\). Models are dependent (share weights). Train small percentage of models.
\end{definition}

\begin{definition}[Data Normalization]
    Update params in similar scales. At test, use \(\mu, \sigma\) from training.
\end{definition}

\begin{definition}[Batch Normalization]
    Normalize weights for each layer. Solve internal covaraite shift (distribution of inputs changes as ealier layers change, gradient falsely assumes constancy). Smoothens loss landscape \(\to\) predictive and stable gradients \(\to\) enables higher LR \(\to\) faster convergence.
    Makes deeper layers more robust. Slight reg. effect since mean/variance introduce noise. During test, tracking running avg. of \(\mu, \sigma^2\).
    \(\tilde z_i = \gamma z_i^{\text{norm}} + \beta\), \(\z_i^{\text{norm}} = \frac{z_i - \mu}{\sqrt{\epsilon + \sigma^2}}\), \(\sigma^2 = \frac{1}{n-1}\sum_i(z_i - \mu)^2\) (\(\gamma, \beta\) learnable).
\end{definition}

\begin{definition}[Data Augmentation]
    Createa new fake data by augmenting existing. Exploit invariances (classification) and equivariances (regression) of target. Either \textit{geometric} or \textit{noise injective}.
\end{definition}

\begin{definition}[Transfer learning]
    If not sufficient data, train model on different task with more data available, then fine-tune on original task.
\end{definition}

\subsection{Optimizations}
\begin{definition}[GD]
    \(\p^{(t+1)} = \p^{(t)} - \eta \cdot \nabla_\p \L(\p)\)
\end{definition}

\begin{definition}[SGD]
    \(\p^{(t+1)} = \p^{(t)} - \eta \cdot \nabla_\p \L(\p; \x\i; y\i)\).
    \adv{Unbiased, high efficiency per iter, avoids local minima};
    \disadv{High \(\Var\); may overshoot; near min, dominated by stochastisity \(\to\) decrease \(\eta\) over time}.
\end{definition}

\begin{definition}[MB-GD]
    \(\p^{(t+1)} = \p^{(t)} - \frac 1 m \nabla_\p \sum_i \L(f(\x\i; \p), y\i)\).
    \adv{Faster than GD, reduces varainceof SGD}
\end{definition}

\begin{definition}[Polyak's Momentum]
    \(\p^{(t+1)} = \p^{(t)} + v\), \(v = \alpha v - \eta \nabla_\p\left(\frac 1 m \sum_i \L(f(x\i; \p), y\i)\right)\). If \(\alpha > \eta\), then previous gradient affect more \(\implies\) larger step size, for successive steps in same direction.
    \disadv{May not conerge.}
\end{definition}

\begin{definition}[Nesterov's Momentum]
    \(\p^{(t+1)} = \p + v\), \(v = \alpha v - \eta \nabla_\p\left(\frac 1 m \sum_i \L(f(x\i; \p + \alpha v), y\i)\right)\). Look ahead before taking a step.
\end{definition}

\begin{definition}[AdaGrad]
    Keep sum of past squared gradients (\(\nabla_2\)). Then \(\eta \frac{\nabla_\p}{\alpha \nabla_2} \).
\end{definition}

\begin{definition}[RMSprop]
    AdaGrad with exp. decay in \(\nabla_2\) to forget. 
\end{definition}

\begin{definition}[Adam]
    Momentum + Adaptive LR. \(\p^{(t+1)} = \p^t - \frac{\alpha m_t}{\sqrt{v_t} + \epsilon}\). \(m_t\) bias-corrected 1st order stimate of momentum. \(v_t\) bias corrected squared gradients.
\end{definition}

% TODO: Maybe add ensambling


\section{MLP and UAT}
Output layer \(l\): \(\z\l l = \sigma(\W\l l \z\l{l - 1})\).
\begin{itemize}
    \item Sens. to weights: \(\frac{\partial \L}{\partial w_{ij}\l{l}} = \frac{\partial \L}{\partial z_i\l l} \frac{\partial z_i\l l }{\partial w_{ij}\l l}\), \(\frac{\partial \L}{\partial \W\l l } = \frac{\partial \L}{\partial \z\l l } \frac{\partial \z \l l}{\partial \W \l l}\).
    \item Sens. to out.: \(\delta_i\l{l-1} = \frac{\partial \L}{\partial z_i\l{l-1}} = \sum_j \frac{\partial \L}{\partial z_j\l l} \frac{\partial z_j\l l}{\partial z_i \l{l-1}} = \sum_j \delta_j\l l \frac{\partial z_j\l l}{\partial z_i \l{l-1}}\).
    Or for full entire layer: \(\vb*{\delta}\l{l-1} = \frac{\partial \L}{\partial \z \l l} \frac{\partial \z\l l}{\partial \z \l {l-1}}\)
\end{itemize}

\begin{definition}[UAT]
    \(\sigma\) any act-f. bounded + cont. \(I_m\) \(m\)-dim hyperc and space of real f. on \(I_m\) is \(C(I_m)\). Then given \(\epsilon > 0\), \(N\) int, \(v_i, b_i \in \R\) and \(\w_i \in \R^m\) for \(i \in [N]\), then \(\forall f \in C(I_m) \forall \x \in I_m:\)
    \[f(\x) \approx \sum_{i=1}^N v_i \sigma(\w_i^\top \x + b_i) \ \text{and} \  \vert g(\x) - f(\x)\vert < \epsilon.\]
\end{definition}


\section{CNN}
\begin{definition}[Linearity]
    \(T(\alpha \u + \beta \v) = \alpha T(\u) + \beta T(\v)\)
\end{definition}

\begin{definition}[Invariance]
    \(T(f(\u)) = T(\u)\) (e.g. classification)
\end{definition}

\begin{definition}[Equivariance]
    \(T(f(\u)) = f(T(\u))\) (e.g. edge detection)
\end{definition}

Any linear, shift-equivariant transform can be a convolution.

\begin{definition}[Correlation]
    \(I'(i, j) = \sum_{m=-k}^k \sum_{n=-k}^k K(m, n) I(i+m, j+n)\)
\end{definition}

\begin{definition}[Convolution]
    \(I'(i, j) = \sum_{m=-k}^k \sum_{n=-k}^k I(i -m, j - n) K(m, n)\)

    \begin{itemize*}
        \item Corr \(\iff K(m, n) = K(-m, -n)\)
        \item Commutative
        \item Toepliz MM
    \end{itemize*}

    Fitler same depth as input map, but can stack multiple filters.
\end{definition}

\begin{definition}[CNN-formulas]
    \(\bm{C}\)han., \(\bm{K}\)er. s., \(m = \#\)Ker., \(\bm{{D}}\)ill., \(\bm{P}\)ad, \(\bm{S}\)tride
    \begin{itemize}
        \item Dim: \(f(W) \times f(H) \times m\), \(f(i) = \frac{i + 2P_i - D_i(K_i - 1) - 1}{S_i} + 1\)
        \item Params: \(p = (K_W \cdot K_H \cdot C + 1) \cdot m \ {\color{H3}+ 1 \hat{=} \ \text{Bias}}\).
        \item Recptive field increases with depth, stride and dilation.
    \end{itemize}
\end{definition}

\begin{definition}[Backprop]
    \(\delta\l{l-1}_{i, j} = \sum_{i'}\sum_{j'}\delta_{i', j'}\l l w_{i'-i, j'-j}\l l = \delta\l l \ast \ROT(W\l l)\)
\end{definition}

\begin{definition}[Update]
    \(\frac{\partial L}{\partial w_{m, n}\l l} = \sum_i \sum_j \delta_{i, j} \l l z_{i-m, j-n}\l{l -1} = \delta\l l \ast \ROT(\Z\l{l-1})\)
\end{definition}

\begin{definition}[Max-Pool]
    \(z_{i, j}\l l = \max\{Z_{is:is+k, js:js+k}\l{l-1}\}\) (\(s\) stride), \(\frac{\partial z_{i, j}\l l}{\partial z_{i', j'} \l{l-1}} = 1\) iff \((i',j') = (i^*, j^*)\) (max indices) and 0 otw. \(\delta\l{l-1} = \{\delta\l l\}_{i^*, j^*}\).
\end{definition}

\begin{definition}[Upsampling]
    Nearest Neighbor(fill all with values), Bed of nails (fill upper left with value, rest 0), Max Unpooling (remember max location), Learnable Upsampling (transpose conv: filter moves stride in output for 1 in input, sum overlap).
\end{definition}


\end{document}
