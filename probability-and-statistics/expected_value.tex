\section{Expected value}

For any random variable \(X: \Omega \to \R_+\) with non-negative values, the expected value is defined with the CDF as:
\vspace{-8pt}
\[\E(X) = \int_\R x \,dF_X(x) \color{gray!70} = \int_\R x \frac{F_X(x)}{dx} \, dx\]

\begin{ddefinition*}[Expected Value, Discrete]
  Let \(X\) be a discrete random variable with value in \(\chi\). The expected value is defined as (if the sum is well defined):
  \vspace{-10pt}
  \[\E(X) := \sum_{x \in \chi}x \cdot p_X(x) = \sum_{\omega \in \O} X(\omega) \cdot p(\omega)\]
\end{ddefinition*}

\begin{proposition}
  {\small \(\E(\I_A) = \P(A)\)}.
\end{proposition}

\begin{proposition}
  If \(\chi \subseteq \N_0\), then \(\E(X) = \sum_{j=1}^\infty \P(X > j)\).
\end{proposition}

\begin{dtheorem*}[Expected Value of transformed RV, Discrete]
  Let \(X\) be discrete with values in \(\chi\). For all \(\phi: \R \to \R\):
  \[\E(\phi(X)) = \sum_{x \in \chi}\phi(x) \cdot p_X(x)\]
\end{dtheorem*}

\begin{cdefinition*}[Expected Value, Continuous]
  Let \(X\) be a continuous random variable with density \(p_x\). The expected value is defined as:
  \vspace{-10pt}
  \[\E(X) := \int_{\R} x \cdot p_X(x) \,dx\]
\end{cdefinition*}

\begin{ctheorem*}[Expected Value of transformed RV, continuous]
  Let \(X\) be continuous with density \(p_x\). For all \(\phi: \R \to \R\):
  \[\E(\phi(X)) = \int_\R \phi(x) \cdot p_X(x) \, dx\]
\end{ctheorem*}

\begin{theorem*}[Linearity of expected value]
  \[\E(\alpha X + \beta Y) = \alpha \E(X) + \beta \E(Y) \quad (a, b \in \R)\]
\end{theorem*}

\begin{theorem*}[Expected value of \(XY\), independence]
  \[X, Y \ \text{ind.} \Leftrightarrow \E(\phi(X) \psi(Y)) = \E(\phi(X)) \E(\psi(Y))\]
\end{theorem*}
This theorem also holds for \(X_1, \ldots X_n\) and \(\phi_1, \ldots, \phi_n\).

\subsection{Variance and Covariance}
\begin{definition*}[Variance]
  Let \(X\) be a random variable such that \(\E(X^2) < \infty\). The variance of \(X\) is defined as:
  \[\Var(X) = \sigma_X^2 = \E((X - \E(X))^2) = \E(X^2) - \E(X)^2\]
  \(\sigma_X\) is called the \textbf{standard deviation} of \(X\).
\end{definition*}

\begin{itemize}
  \item If \(\E(X^2) < \infty, \lambda, \alpha \in \R\), then \(\Var(\lambda X + \alpha) = \lambda^2 \Var(X)\).
  \item If \(S = X_1 + \ldots + X_n\), with \(X_1, \ldots, X_n\) pairwise independent (or uncorrelated), then \(\sigma_S^2 = \sum_{i = 1}^n \sigma_{X_i}^2\)
\end{itemize}

\begin{definition*}[Covariance]
  Let \(X, Y\) be two random variables with \(\E(X^2) < \infty\) and \(\E(Y^2) < \infty\), then their covariance is defined as
  \vspace{-7pt}
  \[\Cov(X, Y) = \E((X - \E(X))(Y - \E(Y)))\]
  If \(\Cov(X, Y) = 0\), then they are uncorrelated.
\end{definition*}
\begin{itemize}
  \item \(\Cov(X, X) = \sigma_X^2 = \Var(X)\)
  \item \(X, Y\) independent \(\implies \Cov(X, Y) = 0\)
\end{itemize}

\subsection{Inequalities}
\begin{lemma}[Monotonicity]
  Let \(X, Y\) be two random variables such that \(X \leq Y\), then \(\E(X) \leq \E(Y)\)
\end{lemma}

\begin{theorem*}[Markov's Inequality]
  Let \(X\) be a random variable with \(X \geq 0\), then for \(a > 0\):
  \vspace{-7pt}
  \[\P(X \geq a) \leq \frac{\E(X)}{a}\]
\end{theorem*}

\begin{theorem*}[Jensen's Inequality]
  Let \(X\) be a random variable and \(\phi: \R \to \R\) a convex function, then
  \vspace{-7pt}
  \[\phi(\E(X)) \leq \E(\phi(X))\]
\end{theorem*}

\begin{theorem*}[Chebychev's Inequality]
  Let \(X\) be a random variable with \(\E(X^2) < \infty\), then for \(t > 0\)
  \[\P(|X - \E(X)| \geq t) \leq \frac{\Var(X)}{t^2}\]
\end{theorem*}
