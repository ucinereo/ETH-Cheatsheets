\section{Confidence Interval}
\begin{definition*}[Confidence Interval]
  A confidence interval with \textbf{confidence level} \(1 - \alpha\) is a subset \(C(x_1, \ldots, x_n) \subseteq \Theta\) such that
  \[\forall \theta \in \Theta \quad \P_\theta(C(X_1, \ldots, X_n) \ni \theta) \geq 1 - \alpha.\]
  Mostly this is a random interval \([u(X), v(X)]\).
\end{definition*}

\subsection*{Confidence Interval \(\to\) Test}
\(C(x_1, \ldots, x_n)\) with confidence level \(1 - \alpha\) known. We define a test statistic \(\I_{\{\theta_0 \notin C(X_1, \ldots, X_n)\}}\). Thus we get:
\[\P_{\theta_0}(\theta_0 \notin C(X_1, \sdots, X_n)) = 1 - \P_{\theta_0}(C(X_1, \sdots, X_n)\ni \theta_0) \leq \alpha.\]

\subsection*{Confidence Interval \(\leftarrow\) Test}
Given \(\forall \theta_0: T_{\theta_0}(x_1, \ldots, x_n) \leq \alpha\). We define the confidence interval \(C(X_1, \ldots, X_n) = \{\theta \in \Theta \mid T_\theta(X_1, \ldots, X_n) \notin K_\theta\}\). Thus we get:
\begin{align*}
  \P_\theta(\theta \in C(X_1, \sdots, X_n)) &= \P_\theta(T_\theta(X_1, \sdots, X_n) \notin K_\theta) \\
  &= 1 - \P_\theta(T_\theta(X_1, \sdots, X_n) \in K) \\
  &\geq 1 - \alpha.
\end{align*}

\subsection*{Approximative Confidence Interval}
If \(X_i \sim \P(\cdot \mid \theta)\) and we want to find a confidence interval with some confidence level \(1 - \alpha\) for \(\theta\), we can use the CLT:

\begin{enumerate}
  \item Calculate \(c_{\neq} = \Phi^{-1}\left(1 - \frac{\alpha}{2}\right)\)
  \item Transform \(\P(c_{\neq} \leq \frac{S_n - n \mu}{\sqrt{\sigma^2 n}} \leq c_{\neq}) = \P(\alpha \leq \theta \leq \beta)\)
  \item Then \(C(X_1, \ldots, X_n) = [\alpha, \beta]\)
\end{enumerate}

\begin{theorem*}[Useful (introduced) inequalities]
  \begin{itemize}
    \item \(\theta \in [\frac{1}{2}, 1] \implies \frac{\sqrt{1 - \theta}}{\theta} \leq \sqrt{2}\)
    \item \(\theta \in [0, 1] \implies \theta(1-\theta) \leq \frac{1}{4}\)
  \end{itemize}
  Can be used for approximative confidence intervals.
\end{theorem*}
