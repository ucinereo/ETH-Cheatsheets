\section{Joint Distributions}
\begin{definition*}[Joint Distribution]
  Let \(X_1, \ldots, X_n\) be random variables defined on the same probability space \((\O, \F, \P)\). Consider \(X = (X_1, \ldots, X_n)\). The joint distribution \(\P_X\) on the hyper set \(A \in \B^n\) where \(\B^n = \sigma(\{A_1 \times \cdots \times A_n \mid A_i \in \B\})\) is defined as:
  \[\P_X(A) = \P(X^{-1}(A)) = \P(\{\omega \in \O \mid X(\omega) \in A\})\]
\end{definition*}

\begin{ddefinition*}[Joint PMF]
  Let \(X = X_1, \ldots, X_n\) be discrete random variables, defined on the same probability space.
  \[p_{X}(x_1, \ldots, x_n) = \P(X_1 = x_1, \ldots, X_n = x_n)\]
\end{ddefinition*}

\begin{ddefinition*}[Joint CDF, discrete]
  Let \(X = X_1, \ldots, X_n\) be discrete random variables, defined on the same probability space.
  \[\scriptstyle F_X(x_1, \ldots, x_n) = \P(X_1 \leq x_1, \ldots, X_n \leq x_n) = \smashoperator{\sum_{y_1 \leq x_1, \ldots, y_n \leq x_n}} p(y_1, \ldots, y_n)\]
\end{ddefinition*}

\begin{proposition}
  Let \(X_1, \ldots X_n\) be discrete random variables with \(X_i \in \DX_i\). Then \(Z = \phi(X_1, \ldots, X_n)\) is a discrete random variable with values in \(\DX = (\DX_1 \times \ldots \times \DX_n)\) and with distribution given by \(\forall z \in \DX\):
  \[\P(Z = z) = \sum_{\substack{x_1 \in \DX_1, \ldots, x_n \in \DX_n \\ \phi(x_1, \ldots, x_n) = z}} \P(X_1 = x_1, \ldots, X_n = x_n)\]
\end{proposition}

\begin{dtheorem*}[Marginal distribution, discrete]
  Let \(X_1, \ldots, X_n\) be discrete random variables with joint PMF. For every \(i\) we have \(\forall z \in \DX_i\):
  \[\P(X_i = z) = \smashoperator[r]{\sum_{x_1, \ldots x_{i-1}, x_{i+1}, \ldots, x_n}} p(x_1, \ldots, x_{i-1}, z, x_{x+1}, \ldots, x_n)\]
\end{dtheorem*}

\begin{proposition}
  The expected values of joint discrete random variables can be written as:
  \[\E(\phi(X_1, \ldots, X_n)) = \smashoperator[r]{\sum_{x_1 \in \DX_1, \ldots, x_n \in \DX_n}} \phi(x_1, \ldots, x_n) \cdot p(x_1, \ldots, x_n)\]
\end{proposition}

\pagebreak
\begin{cdefinition*}[Joint PDF]
  Let the random variables \(X = X_1, \ldots, X_n\) be defined on the same probability space. We say these random variables have a joint PDF \(f_{X_1, \ldots, X_n} = f_X\) if for all subsets \(A \in \B^n\) it holds
  \begin{align*}
    \P_X(A) = \P((X_1, \ldots, X_n) \in A) \\
    = \int_A f_X(x_1, \ldots, x_n) \, dx_1 \ldots \,dx_n
  \end{align*}
\end{cdefinition*}

\begin{cdefinition*}[Joint CDF, continuous]
  Let \(X = X_1, \ldots, X_n\) be continuous random variables, defined on the same probability space.
  \begin{align*}
    F_{X}(x_1, \ldots, x_n) = \P(X_1 \leq x_1, \ldots, X_n \leq x_n) \\
    = \int_{-\infty}^{x_1} \ldots \int_{-\infty}^{x_n} f_{X}(x_1, \ldots, x_n) \,dx_n \ldots \, dx_1
  \end{align*}
\end{cdefinition*}

\begin{proposition}
  Let \(X_1, \ldots, X_n\) be continuous random variables with joint PDF and \(\phi: \R^n \to \R\). Then the expected value is defined as
  \begin{multline*}
    \E(\phi(X_1, \ldots, X_n)) = \\ \int_\R \ldots \int_\R \phi(x_1, \ldots, x_n) \cdot f(x_1, \ldots, x_n) \, dx_n \ldots \, dx_1
  \end{multline*}
\end{proposition}

\begin{ctheorem*}[Marginal density, continuous]
  Let \(X_1, \ldots, X_n\) be continuous random variables with a joint PDF. For every \(i\) we have \(\forall z \in \DX_i\):
  \[f_{X_i}(z) = \smashoperator{\int_{x_1, \ldots, x_{i-1}, x_{x+1}, \ldots, x_n}} f(x_1, \ldots, x_{i-1}, z, x_{i+1}, \ldots, x_n) \, dx_n \ldots \, dx_1\]
\end{ctheorem*}

\begin{theorem*}[Independence of joint random variables]
  Let \(X_1, \ldots, X_n\) be random variables with \(p_{X_1}, \ldots, p_{X_n}\).
  \begin{itemize}
    \item[] \(X_1, \ldots X_n\) are independent
    \item[\(\Leftrightarrow\)] \(p(x_1, \ldots, x_n) = p_{X_1}(x_1) \cdot \ldots \cdot p_{X_n}(x_n)\)
    \item[\(\Leftrightarrow\)] \(\E(\phi_1(X_1) \cdot \ldots \cdot \phi_n(X_n)) = \E(\phi_1(X_1)) \cdot \ldots \cdot \E(\phi_n(X_n))\)
  \end{itemize}
\end{theorem*}

\pagebreak
\subsection{Useful lemmas and theorems}

\begin{definition*}[Distribution of transformed random variables]
  Let \(X = (X_1, \ldots, X_n)\) and \(Y(\omega) = g(X(\omega))\) be two random variable with \(g: \R^n \to \R^m\). Then for any \(A \in \B(\R)\)
  \[\mu_Y(A) = \mu_X(g^{-1}(A)).\]
\end{definition*}

\begin{theorem*}[Transformation Theorem]
  Let \(g(x) = m + Bx\) with \(\det(B) \neq 0\). If \(\mu_X\) is abs. continuous, then \(\mu_Y\) is abs. continuous and furthermore
  \[f_Y(x) = \frac{1}{|\det(B)|} f_X(B^{-1}(x - m)).\]
\end{theorem*}

\begin{example}
  \(X_1, X_2 \sim U(0, 1)\), what is the joint CDF of \(Y = X_1 + X_2\) and \(Z = X_1 - X_2\)? \\
  Then \(g(x_1, x_2) = (x_1 + x_2, x_1 - x_2)\) and \(B = \begin{pmatrix}
    1 & 1 \\
    1 & -1
  \end{pmatrix}\) with \(\det (B) = 2\), \(g(x) = Bx\). Thus \((Y, Z) = g(X_1, X_2)\) and with the transformation theorem we get
  \[f_{Y, Z}(y, z) = f_{X_1, X_2}(g^{-1}(y, z)) = \frac{1}{2}f_{X_1, X_2}\left(\frac{y+z}{2}, \frac{y-z}{2}\right)\]
\end{example}

\begin{proposition}
  If the R.V. \((X, Y)\) has a density function \(f\), then \(Z = X + Y\) has a density 
  \(f_Z(z) = \int_\R f(x, z - x) \, dx\).
\end{proposition}

\begin{definition*}[Convolution]
  If \(X\) and \(Y\) are independent, then \(Z = X + Y\) has a density function, called the convolution:
  \[(f_X * f_Y) := f_Z(z) = \int_\R f_X(x)f_Y(z - x) \, dx\]
\end{definition*}

\begin{proposition}
  Suppose \(X_1, \ldots, X_n\) are i.i.d with CDF \(F\). Then \(V = \max\{X_1, \ldots, X_n\}\) has CDF \(F_V(x) = F^n(x)\) and
  \(U = \min \{X_1, \ldots, X_n\}\) has CDF \(F_U(x) = 1 - (1 - F(x))^n\).
\end{proposition}

\begin{ctheorem*}[Transformation of Bivariate Random Variables]
  Let \((X, Y)\) be two continuous R.V. with density \(f_{X, Y}\). If we have a bijective transformation \(U = g_1(X, Y), V = g_2(X, Y)\) and define the inverse as \(X = h_1(U, V)\) and \(Y = h_2(U, V)\), then the joint PDF for \((U, V)\) is:
  \[f_{U, V}(u, v) = f_{X, Y}(h_1(u, v), h_2(u, v))|J| \quad J = \begin{bmatrix}
    \frac{\partial h_1}{\partial u} & \frac{\partial h_1}{\partial v} \\
    \frac{\partial h_2}{\partial u} & \frac{\partial h_2}{\partial v}
  \end{bmatrix}\]
\end{ctheorem*}
