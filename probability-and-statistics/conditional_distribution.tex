\section{Conditional Distribution}
\begin{ddefinition*}[Conditional PMF]
  Let \(X, Y\) be two discrete random variables on the same probability space taking values in \(\chi\) and \(\mathcal{Y}\). For \(y \in \mathcal{Y}\) such that \(\P(Y = y) \neq 0\), we define the \textit{conditional distribution of \(X\) given \(Y = y\)} as
  \[p_{X \mid Y}(x \mid y) := \frac{p_{X, Y}(x, y)}{p_Y(y)} = \frac{p_{X, Y}(x, y)}{\int_\R p_{X, Y}(x, y) \,dx}\]
\end{ddefinition*}

If \(X\) and \(Y\) are independent, then \(p(x \mid y) = p(x)\). For a fixed \(y \in \mathcal{Y}\) with \(\P(Y = y) > 0\) the function \(p_{X \mid Y}(\cdot \mid y)\) is a probability mass function. Thus, we can consider a random variable with this PMF written as \(X \mid Y = y\).

\begin{dtheorem*}[Multiple variables]
  \vspace*{-18pt}
  \begin{multline*}
    p_{X_{k+1}, \ldots, X_n \mid X_1, \ldots, X_k}(x_{k+1}, \ldots, x_n \mid x_1, \ldots x_k) \\
    := \frac{p_{X_1, \ldots, X_n}(x_1, \ldots, x_n)}{p_{X_1, \ldots, X_k}(x_1, \ldots, x_k)}
  \end{multline*}
\end{dtheorem*}

\begin{cdefinition*}[Conditional Density]
  Let \(X, Y\) be two continuous random variables with joint density \(p_{X, Y}\). The \textit{conditional density of \(X\) given \(Y = y\)} is defined as
  \[f_{X \mid Y}(x \mid y) = \begin{cases}
    \frac{f_{X, Y}(x, y)}{p_Y(y)} & \text{if} \ f_Y(y) > 0 \\
    0 & \text{otherwise}
  \end{cases}\]
\end{cdefinition*}

\subsection{Mixed case}
\begin{definition*}[Conditional probability density]
  Let \(X\) be a continuous and \(Y\) a discrete random variable. For each \(y \in \mathcal{Y}\), we have the conditional probability distribution \(\P_{X 
  mid y}\) on \(\R\) for each subset \(A \in \B(\R)\) defined as \(\P_{X \mid y}(A) :=\)
  
  \vspace{-15pt}
  \[\frac{\P_{X, Y}(\{(x, y) \in \R^2 \mid x \in A\})}{\P_Y(\{y\})} = \frac{\P(X \in A, Y = y)}{p_Y(y)}\]
  
  If this conditional distribution has a density, we call it the \textit{conditional probability density of \(X\) given \(Y = y\)}
\end{definition*}

\begin{ctheorem*}[Law of Total Probability, \(Y\) continuous]
  \[f_X(x) = \int f_{X \mid Y}(x \mid y) f_Y(y) \, dy\]
\end{ctheorem*}

\begin{dtheorem*}[Law of Total Probability, \(Y\) discrete]
  \[p_X(x) = \sum_{y \in \mathcal{Y}} p_{X \mid Y}(x \mid y)p_Y(y)\]
\end{dtheorem*}

\begin{ctheorem*}[Bayes' Rule, \(X\) continuous]
  \[f_{X \mid Y}(x \mid y) = \frac{f_{Y \mid X}(y \mid x) f_X(x)}{\int f_{Y \mid X}(y \mid x') f_X(x') \, dx'}\]
\end{ctheorem*}

\begin{dtheorem*}[Bayes' Rule, \(X\) discrete]
  \[p_{X \mid Y}(x \mid y) = \frac{p_{Y \mid X}(y \mid x) p_X(x)}{\sum_{x' \in \chi} p_{Y \mid X}(y \mid x')}\]
\end{dtheorem*}

\begin{theorem*}[Chain Rule]
  Let \(X_1, \ldots, X_n\) be random variables. Then
  \[p(x_1, \ldots, x_n) = p(x_1) p(x_2 \mid x_1) \cdots p(x_n \mid x_1, \ldots, x_n)\]
\end{theorem*}

\begin{theorem*}[Conditional Independence]
  Let \(X, Y, Z\) be random variables. The random variables \(X\) and \(Y\) are said to be conditionally independent given \(Z\) if \(\forall x, y, z\):
  \vspace{-10pt}
  \[p_{X, Y \mid Z}(x, y \mid z) = p_{X \mid Z}(x \mid z) p_{Y \mid Z}(y \mid z)\]
\end{theorem*}
