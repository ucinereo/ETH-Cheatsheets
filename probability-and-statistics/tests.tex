\section{Tests}
\begin{definition*}[Null hypothesis \(\bm{H_0}\), alternative Hypothesis \(\bm{H_A}\)]
  Two subsets \(\Theta_0 \subseteq \Theta, \Theta_A \subseteq \Theta\), s.t. \(\Theta_0 \cap \Theta_A = \emptyset\). A hypothesis is \textbf{simple} if \(\Theta_0 = \{\alpha \in \Theta\}\) or else \textbf{composite}.
\end{definition*}

\begin{definition*}[Test]
  A \textbf{test} is a tuple \((T, K)\), where \(T\) is a random variable of form \(T = t(X_1, \ldots, X_n)\) and \(K \subseteq \R\) a deterministic subset of \(\R\). \(T\) is called the \textbf{test statistic} and \(K\) is the \textbf{rejection region} or critical region. 
\end{definition*}

We reject a hypothesis \(H_0\) \(\iff\) \(\I_{t(x_1, \ldots, x_n) \in K} = 1\).

\begin{definition*}[Error types]
  \begin{itemize}
    \item \textbf{Type I Error}: Rejecting a true \(H_0\). \\
    Probability: \(\P_\theta(T \in K), \ \theta \in \Theta_0\).
    \item \textbf{Type II Error}: Failing to reject a false \(H_0\). \\
    Probability: \(\P_\theta(T \not\in K) = 1 - \P_\theta(T \in K), \ \theta \in \Theta_A\).
  \end{itemize}
\end{definition*}
1. Minimize type I errors:
\begin{definition*}[Significance level of a test]
  A test has a significance level \(\alpha \in [0, 1]\) if
  \[\underset{\theta \in \Theta_0}{\sup} \P_\theta(T \in K) \leq \alpha\]
\end{definition*}

2. Maximize the power of the test (reducing type II errors)
\begin{definition*}[Power of a test]
  The power of a test is defined as:
  \[\beta: \Theta_A \to [0, 1], \quad \theta \mapsto \beta(\theta) := \P_\theta(T \in K)\]
\end{definition*}

As this is asymmetric and thus makes it harder to reject \(H_0\) than accepting it, we take the \textbf{negation} of the wanted statement as \(H_0\).

\subsection{How to find the hypothesis}
\begin{enumerate}
  \item List the things to avoid.
  \item Take that as Type I Error.
  \item Conclude the \(H_0\) with reverse engineering.
\end{enumerate}

\pagebreak
\subsection{z-test}
\textbf{Prerequisites:} \(X_1, \ldots, X_n \sim \Normal(\mu, \sigma^2)\) iid; \(\bm{\sigma^2}\) \textbf{known}.

\textbf{Hypothesis:} \(H_0: \theta = \theta_0\), \(H_A: \theta > \theta_0 \, \text{or} \, \theta < \theta_0 \, \text{or} \, \theta \neq \theta_0\).

\textbf{Teststatistic:} \(T := \frac{\bar{X_n} - \theta_0}{\sigma / \sqrt{n}} \sim \Normal(0, 1)\) under \(\P_{\theta_0}\).

\textbf{Rejection Region:}
\begin{align*}
  H_A: \theta > \theta_0 &\implies K_> = (c_>, \infty) \\
  H_A: \theta < \theta_0 &\implies K_< = (-\infty, c_<) \\
  H_A: \theta \neq \theta_0 &\implies K_{\neq} = (-\infty, -c_{\neq}) \cup (c_{\neq}, \infty)
\end{align*}

\textbf{How to find \(c_>, c_<, c_{\neq}\):}
\begin{align*}
  \alpha = \P_{\theta_0}(T > c_>) &\implies c_> = \Phi^{-1}(1 - \alpha) \\
  \alpha = 1 - \P_{\theta_0}(T \leq -c_<) &\implies c_< = -\Phi^{-1}(1 - \alpha) \\
  \alpha = 2(1 - \P(T \leq c_{\neq})) &\implies c_{\neq} = \Phi^{-1}\left(1 - \frac{\alpha}{2}\right)
\end{align*}

\subsection{t-test}
\textbf{Prerequisites:} \(X_1, \ldots, X_n \sim \Normal(\mu, \sigma^2)\) iid.

\textbf{Hypothesis:} \(H_0: \mu = \mu_0\), \(H_A:\) same as z-Test.

\textbf{Teststatistic:} \(T := \frac{\bar{X_n} - \mu_0}{S / \sqrt{n}} \sim t_{n-1} \qquad (S = \sqrt{S^2})\).

\textbf{Rejection Region:} Same as z-Test.

\textbf{How to find \(c_>, c_<, c_{\neq}\):}
\begin{align*}
  c_> = t_{n-1} \quad c_< = t_{n-1, \alpha} = -t_{n-1, 1-\alpha} \quad c_{\neq} = t_{n-1, 1 - \frac{\alpha}{2}}
\end{align*}

\subsection{Unpaired two-sample z-, t-test}
\textbf{Pre.:} \(X_1, \sdots, X_{n} \sim \Normal(\mu_1, \sigma^2), \, Y_1, \sdots, Y_{m} \sim \Normal(\mu_2, \sigma^2)\) iid.

\textbf{Hypothesis:} \(H_0: \mu_1 - \mu_2 = \omega_0\), \(H_A:\) same as z-Test.

\textbf{Rejection Region:} Same as z-Test.

\textbf{Teststatistic (known \(\bm{\sigma^2}\)):}
\begin{center}
  \(T := \frac{(\bar{X_{n}} - \bar{Y_{m}}) - (\mu_1 - \mu_2)}{\sigma\sqrt{\frac{1}{n} + \frac{1}{m}}} \sim \Normal(0, 1)\).
\end{center}

\textbf{Teststatistic (unknown \(\bm{\sigma^2}\)):} \\
\(T := \frac{(\bar{X_{n}} - \bar{Y_{m}}) - (\mu_1 - \mu_2)}{S_p\sqrt{\frac{1}{n} + \frac{1}{m}}} \sim t_{m + n - 2}\) with sample variances \(S_1\), \(S_2\) and 
\(S_p = \sqrt{\frac{(n - 1)S_1^2 + (m - 1)S_2^2}{n + m - 2}}\).

\textbf{How to find \(c_>, c_<, c_{\neq}\):} Same as z-, t-Test.

\subsection{Paired two-sample z-, t-test}
\(Z_i := X_i - Y_i \sim \Normal(\mu_1 - \mu_2, 2\sigma^2)\). Then do z-, t-Test.
