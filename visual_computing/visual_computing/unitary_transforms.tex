\section{Unitary Transforms}

\begin{definition}[Unitary Transform]
  Transform \(A\) unitary \(\iff A^{-1} = A^H\)
\end{definition}

\begin{theorem}
  \(A\) unitary \(\implies \forall f: ||Af||^2 = ||f||^2\).
  Thus rotation. Energy conserved, but often unevenly distributed among coeffs.
\end{theorem}

\begin{definition}[Vectorization]
  Image rewritten as vector of dimension \(H \cdot W\). 
\end{definition}

\begin{definition}[Linear Image Processing]
  \(g = Hf\) and \(H\) need not be square.
\end{definition}

\begin{definition}[Image collection]
  \(F = [f_1 \ f_2 \ \ldots \ f_n]\)
\end{definition}

\begin{definition}[Autocorrelation Function]
  \(R_{ff} = \E[f_i \cdot f_i^H] = \frac{1}{n} F \cdot F^H\). 
\end{definition}

\begin{definition}[Autocorrelation Matrix]
  \(R_{cc} = \E[c \cdot c^H] = A R_{ff} A^H\) w/ \(c = Af\)
\end{definition}

\begin{definition}[Eigenmatrix of \(R_{ff}\)]
  \(\Phi\) s.t. \(R_{ff}\Phi = \Phi \Lambda\), columns as EVs and unitary.
  \(\Lambda = \text{diag}(\lambda_0, \sdots, \lambda_{HW - 1})\).
  Normal: \(R_{ff}^HR_{ff} = R_{ff}R_{ff}^H\).
\end{definition}

\subsection{Eigenimages / Eigenfaces}

\begin{algorithm}[Karhunen-Loeve Transform (PCA)]
  Unitary transform with \(A = \Phi^H\).
  \(\Phi\) ordered by decreasing eigenvalues. Transform coefficients are pairwise uncorrelated: \(R_{cc} = AR_{ff}A^H = \Lambda\).

  \textit{Energy concentration property}: No other unitary transform packs as much energy in the first \(J\) coeffs and mean squared approx error by choosing only first \(J\) coeffs is minimized.
\end{algorithm}

\begin{definition}[PCA Steps]
  \begin{enumerate}
    \item Standardize image collection.
    \item Compute covariance matrix + its eigenvalue decomposition or do SVD on original matrix.
    \item Compute projection \(U_J\) from largest \(J\) components.
    \item \(\text{PCA}(x) = U_J^\top(x - \mu)\) with \(\mu\) being the mean.
  \end{enumerate}
\end{definition}

\begin{definition}[Uses of PCA]
  Lossy compression of \(I\) by keeping only the most important \(k\) component. Reconstruct original by \(UI_c + \mu\).
\end{definition}

\begin{algorithm}[Eigenspace matching]
  Do PCA (with mean subtraction) to get closest rank-\(k\) approx. of database images. For a new query: normalize, subtract mean (of database), project to subspace, then do similarity matching with eigenfaces.
  \[\text{argmin}_i D_i = ||I_i - I|| \approx ||p_i - p||\]
\end{algorithm}

\begin{theorem}
  Lighting differences much larger than face differences. Thus works better by removing the first 3 coefficients.
\end{theorem}

\subsection{Fisherfaces}
Find directions where ratio between:within individual variance is maximized. Linearly project to basis where dimensions with good SNR is maximized.

\begin{algorithm}[Fisher LDA]
  Maximize between class scatter while minimizing within class scatter.
\end{algorithm}

FDA tries to create subspace in the direction of the class averages.

\includegraphics*[width=\linewidth]{assets/variances.png}

Fisher is better for light independent classification.
