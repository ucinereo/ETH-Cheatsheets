\section{Image Filtering}
Modify pixels based on linear-combination of neighborhood.
\[I'(x, y) = \sum_{(i, j) \in N(x, y)} K(x, y; i, j)I(i, j)\]

\begin{definition}[Shift-Invariant]
  \(K\) independent of \((x, y) \iff\) same for all px
\end{definition}

\begin{definition}[Separable \(K\)]
  \(K(i, j) = f(i) \cdot g(j) \iff \text{rank}(K) = 1\)
\end{definition}

\begin{definition}[Correlation (\(K \circ I\))]
  Also called \textit{pattern matching}.
  \[I'(x, y) = \sum_{(i, j) \in N(x, y)} K(i, j) I(x + i, y + j)\]
\end{definition}

\begin{definition}[Convolution (\(K \ast I\))]
  Also called \textit{point spread}.
  \begin{align*}
    I'(x, y) &= \sum_{(i, j) \in N(x, y)} K(i, j) I(x - i, y - j) \\
    &= \sum_{(i, j) \in N(x, y)} K(-i, -j) I(x + i, y + j) \\
  (f \ast g)(t) &= \int_{\R} f(t')g(t - t') \, dt' \qquad (\text{cont. case})
  \end{align*}

  \textit{Properties}:
  \begin{itemize*}
    \item Local (only \(N(x, y)\))
    \item Associative
    \item Shift invariant
    \item Linear (\(k \ast (\alpha I_1 + \beta I_2) = \alpha(k \ast I_1) + \beta(k \ast I_2)\))
    \item Commutative
    \item Distributive over +
  \end{itemize*}
\end{definition}

\begin{theorems}
  \begin{itemize*}
    \item \(K(i, j) = K(-i, -j) \implies \circ \equiv \ast\) \\
    \item convolution is correlation with point-mirrored kernel \(K\)
  \end{itemize*}
\end{theorems}

\begin{definition}[Edge Handling]
  \begin{itemize*}
    \item clip filter (black)
    \item wrap around
    \item copy edge
    \item reflect across edge
    \item vary filter near edge
  \end{itemize*}
\end{definition}


\subsection{Special Filters}

\begin{tabularx}{\linewidth}{cc}
  Identity \(\delta\): \(\begin{bmatrix}
    0 & 0 & 0 \\
    0 & 1 & 0 \\
    0 & 0 & 0
  \end{bmatrix}\) & 
  Left Shift (conv.): \(\begin{bmatrix}
    0 & 0 & 0 \\
    1 & 0 & 0 \\
    0 & 0 & 0
  \end{bmatrix}\)
\end{tabularx}

\subsection{Low-Pass Filters (Smoothing)}
Suppress high frequencies (e.g. fine scale details, edges, noise).
Result is equivalent to \textit{smoothing}.

\begin{definition}[Box filter]
  All same values, normalized so sum = 1.
\end{definition}

\begin{definition}[Gaussian Kernel]
  \(G_\sigma = \frac{1}{2 \pi \sigma^2} \exp\left(-\frac{(x^2 + y^2)}{2\sigma^2}\right)\)

  \textit{Properties}:
  \begin{itemize*}
    \item rotationally symmetric
    \item single lobe in both domains
    \item simple relation to \(\sigma\)
    \item efficient
  \end{itemize*}
\end{definition}

\begin{theorem}
  \(G \sim \normal(0, \sigma^2) \implies G \ast G \sim \normal(0, (\sigma \sqrt{2})^2)\)
\end{theorem}

\subsection{High-Pass Filters (Sharpening)}
Suppresses low frequencies which is equivalent to sharpening the edges. Suppressed low frequencies correspond to areas of constant gray level. \textit{Result equivalent to difference between original image and image filtered by Gaussian}.

\begin{definition}[Construction]
  HPF = Identity - LPF
\end{definition}

\begin{theorem}
  Subtracting 1 from centre of LPF, gives negative HPF.
  \((f - \delta) \ast a = f \ast a - \delta \ast a = f \ast a - a = -(a - (f \ast a))\)
\end{theorem}

\begin{definition}[Image Sharpening]
  \(I' = I + \alpha |k \ast I|\), w/ \(k\) HPF and \(\alpha \in [0, 1]\)
\end{definition}

\begin{tabularx}{\linewidth}{cc}
  Laplacian: \(\begin{bmatrix}
    0 & 1 & 0 \\
    1 & -4 & 1 \\
    0 & 1 & 0
  \end{bmatrix}\) & 
  High-pass: \(\begin{bmatrix}
    -1 & -1 & -1 \\
    -1 & 8 & -1 \\
    -1 & -1 & -1
  \end{bmatrix}\)
\end{tabularx}

\subsection{Band Filters}

\begin{definition}[Band pass filter]
  LPF and HPF with cutoffs \(D_{LP} < D_{HP}\)
\end{definition}

\begin{definition}[Band reject filter]
  LPF and HPF with cutoffs \(D_{LP} > D_{HP}\)
\end{definition}

\subsection{Differential Filters}
Edges correspond to points with \textit{high} gradient magnitude. Thus we can approximate the derivative with a simple convolution:
\begin{center}
  \(\frac{\partial f}{\partial x} = \lim_{\epsilon \to 0}\left(\frac{f(x + \epsilon, y) - f(x, y)}{\epsilon}\right) \approx \left(\frac{f(x_{n+1}, y) - f(x_n, y)}{\varDelta x}\right)\)
\end{center}

\begin{tabularx}{\linewidth}{cc}
  Prewitt\(_x\): \(\begin{bmatrix}
    -1 & 0 & 1 \\
    -1 & 0 & 1 \\
    -1 & 0 & 1
  \end{bmatrix}\) & 
  Sobel\(_x\): \(\begin{bmatrix}
    -1 & 0 & 1 \\
    -2 & 0 & 2 \\
    -1 & 0 & 1
  \end{bmatrix}\) \\
  \(\frac{\partial I}{\partial x} = \begin{bmatrix}
    -1 & 1
  \end{bmatrix} \ast I\) &
  Roberts: \(\begin{bmatrix}
    0 & 1 \\
    -1 & 0
  \end{bmatrix} \begin{bmatrix}
    1 & 0 \\
    0 & -1
  \end{bmatrix}\)
\end{tabularx}

