\section{Video Compression}

\begin{itemize}
    \item Perception limited to $<24Hz$ (images will be perceived as continuous if frequency is sufficiently high)
    \item Still need to avoid aliasing (wheel effect) (high FPS desired in games, due to absence of motion blur)
    \item Flicker can be perceived up to $>60Hz$.
\end{itemize}

\begin{definition}[Bloch's Law]
  We perceive same image even if the intensity different, but longer screen time. Thus FPS \(> 10 \text{Hz}\).
\end{definition}

\begin{definition}[Interlaced Video Format]
  2 temporally shifted half images, increase of frequency. \(\to\) reduction of spatial resolution and full image representation is progressive.
\end{definition}

\begin{definition}[Lossy video compression]
  Use redundancy: spatial correlation in px's and temporal correlation in frames. Drop details.
\end{definition}

\begin{algorithm}[Temporal Redundancy Reduction]
  I-B-B-P-B-B-P-I-\(\ldots\)
  \begin{itemize}
    \item \textit{I-frame}: Intra-coded, coded independently
    \item \textit{P-frame}: Predictively coded, based on previous
    \item \textit{B-frame}: Bi-directionally coded, based on prev. \& future
  \end{itemize}
\end{algorithm}

\begin{theorem}
  Ineffective with scene changes and high motion. \(\to\) use MC prediction and Block-Matching Estimation.
\end{theorem}

\begin{algorithm}[Block-Matching Motion Estimation (ME)]
  Is a type of temporal redundancy reduction without object identification.

  \begin{enumerate}
    \item Partition frame into blocks (e.g. 16 x 16 pixel)
    \item Per block, find best match in new frame with e.g. SSD.
  \end{enumerate}

  \textit{Is an approximation of ME}.
\end{algorithm}

\begin{definition}[Candidate blocks]
  All blocks in e.g. 32 x 32 pixel area.
\end{definition}

\begin{definition}[Search strategies]
  Full search, partial (fast) search
\end{definition}

\begin{definition}[Advantages]
  \begin{itemize*}
    \item good, robust performance
    \item one MV per block \(\to\) useful for compression
    \item simple periodic structure
  \end{itemize*}
\end{definition}

\begin{definition}[Disadvantages]
  \begin{itemize*}
    \item Assumes translational motion model (code blocks without prediction)
    \item produces blocking artifacts
  \end{itemize*}
\end{definition}

\begin{algorithm}[Motion Compensation Algorithm (MC)]
  Use best matching of reference frame as prediction of blocks in current frame.

  \(\to\) Gives motion vectors \& MC prediction error or residual.
\end{algorithm}

\begin{definition}[Motion Vector]
  Relative horizontal and vertical offsets of a given block from one frame to another (not limited by integers).
\end{definition}

\begin{algorithm}[Half-pixel ME (corse-fine) algorithm]
  Bilinear interpolate spatially to get sub-integer px offsets \(\to\) averaging effect and reduced prediction error \& noise \(\to\) improved compression.
  \begin{enumerate}
    \item \textit{Coarse step}: find best integer MV
    \item \textit{Fine step}: Refine by spatial interp. and best-matching
  \end{enumerate}
\end{algorithm}

\pagebreak
\begin{definition}[Bidirectional MC predicition]
  Use \(I\), \(P\) and \(B\)-frames, the average of a block from the previous frame and a block from the future frame (skip \(B\) frames i multiple in a row).
\end{definition}

\begin{algorithm}[MPEG GOP]
  {\color{H1}I-B-B-P-B-B-P-B-B}-{\color{H3}I-B-B-\(\ldots\)}
\end{algorithm}

\begin{definition}[Basic Video Compression Architecture]
  \textit{Redundancies}: temporal with MC-prediction (P and B frames).
  Spatial with Block DCT.
  Color with color space conversion.

  \textit{Scalar quantization} of DCT coeffs. \textit{Zigzag} scanning, runlength and Huffman coding of nonzero quantized DCT coeffs.
\end{definition}

\begin{definition}[Error Measures]
  Pixel wise MSE with decoded or PSNR.
  \(PSNR = \frac{\max x^2}{e_{\text{MSE}}^2} = 10 \cdot \log_10 (\frac{255^2}{e^2_{\text{MSSE}}}) dB\).
\end{definition}

\begin{definition}[Subjective evaluation]
  1 very annoying to 5 imperceptible
\end{definition}
