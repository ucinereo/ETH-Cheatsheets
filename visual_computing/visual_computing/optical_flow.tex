\section{Optical Flow}

\begin{definition}[Optical Flow]
  Apparent motion of brightness patterns. Projection of 3D velocity vectors on the image. Hope: equal to motion field!
\end{definition}

\begin{itemize}
  \item \textit{Motion/Scene flow}: projection of 3D motion field
  \item \textit{Normal flow}: observed tangent motion (\(\bot\) to motion line)
\end{itemize}

\begin{definition}[Problem]
  Cannot distinguish between motion / changing light.
\end{definition}

\subsection{Assumptions}

\begin{definition}[1. Brightness Constancy]
  Point brightness remains the same:
  \begin{center}
    \(I(x, y, t) = I(x + \delta x, y + \delta y, t + \delta t)\)
  \end{center}
\end{definition}

\begin{definition}[2. Small motion]
  Points to not move far over time.
\end{definition}

\(\implies\) By applying Taylor expansion with the constraints:
\begin{flalign*}
  &&&I(x, y, t) \approx I(x, y, t) + \partial_x I \delta x +\partial_y I \delta y + \partial_t I \delta t  &\\
  &\iff &&\frac{\partial I}{\partial x}\frac{dx}{dt} + \frac{\partial I}{\partial y}\frac{dy}{dt} + \frac{\partial I}{\partial t} = I_x u + I_y v + I_t \approx 0
\end{flalign*}

With \(I_x, \ I_y\) spatial derivatives; \(u, \ v\) OF and \(I_t\) frame difference.

\begin{definition}[Aperture Problem]
  1 equation, 2 unknowns: cannot determine exact location!
  Take normal flow: 
  \(u_\bot = - \frac{I_t}{|\nabla I|} \frac{\nabla I}{|\nabla I|}\).
\end{definition}

\subsection{Algorithms}

\begin{algorithm}[Horn \& Schunck]
  Add additional smoothness constraints:
  \(\begin{rcases*}
    e_s = \iint ((u_x^2 + u_y^2) + (v_x^2 + v_y^2)) \,dx \,dy \\
    e_c = \iint (I_x u + I_yv +I_t)^2 \,dx \,dy
  \end{rcases*} \ \text{minimize} \ e_s + \lambda e_c \)

  Really hard to solve. Info spreads from corner-type patterns.

  It is an example for regularization with parameter \(\lambda\).
\end{algorithm}

\begin{algorithm}[Lucas-Kanade]
  Assume \textbf{spatial coherency}: the same flow for pixels withing a patch \(\Omega = \{p_1, \sdots , p_n\}\). Minimize error:
  \[E(u, v) = \sum_{x, y \in \Omega} (I_x(x, y) u + I_y(x, y) v + I_t)^2\]
  
  Compute derivatives and solve the LSE:
  \begin{gather*}
    \left[\begin{smallmatrix}
    I_x(p_1) & I_y(p_1) \\ \cdots & \cdots \\ I_x(p_n) & I_y(p_n)
  \end{smallmatrix}\right] \left[\begin{smallmatrix}
    u \\ v
  \end{smallmatrix}\right] =  - \left[\begin{smallmatrix}
    I_t(p_1) \\ \cdots \\ I_t(p_n)
  \end{smallmatrix}\right]
  \end{gather*}
\[
\underbrace{
\left(\begin{array}{cc}
\sum I_x^2 & \sum I_x I_y \\
\sum I_x I_y & \sum I_y^2
\end{array}\right)}_{=M= \sum (\nabla I)(\nabla I)^T}
\underbrace{\left(\begin{array}{l}
u \\
v
\end{array}\right)}_{U}=
\underbrace{-\left(\begin{array}{l}
\sum I_x I_t \\
\sum I_y I_t
\end{array}\right)}_{=b=-\sum \nabla I I_t}
\]
  Solve with normal equations \(A^\top A x = b\) (which the above is).

  \textbf{Problems}: \(A^\top A\) singular at straight edges / uniform planes. And thus when gradient vectors point in the same direction.

  \textbf{Solvable}: \(A^\top A\) not singular (\(\lambda_1, \lambda_2\) large and \(\lambda_1 / \lambda_2\) small).
\end{algorithm}

\begin{definition}[Iterative Refinement]
  Estimate velocity, warp, refine, repeat.
\end{definition}

\begin{definition}[Limits of local gradient method]
  \begin{itemize}
    \item Brightness is not constant in images (use pre-filter to make images look similar)
    \item Fails when intensity structure within window is poor
    \item Fails when the displacement is large (\(> 1 px\)); linearization of brightness is suitable only for small displacements.
  \end{itemize}
\end{definition}

\begin{algorithm}[Coarse-to-Fine Estimation]
  Get larger motion: start small, compute OF, rescale, take larger and init w/ last OF estimate.
\end{algorithm}
\begin{center}
  \includegraphics*[width=.7\linewidth]{assets/coarse-to-fine.jpg}
\end{center}

\begin{definition}[Applications]
  \begin{itemize*}
    \item Image stabilization: get flow between frames, warp image such that OF close to 0.
    \item frame interpolation
    \item video compression
    \item object tracking
    \item motion segmentation
  \end{itemize*}
\end{definition}

\subsection{Parametric Motion Models}
Offer more constrained solutions than smoothness (Horn and Schunck). Integration over a larger area than a translation-only model can accommodate (LK).

\begin{definition}[2D Models]
  \begin{itemize*}
    \item Translation
    \item Affine
    \item Quadratic
    \item Planar projective transform (Homography)
  \end{itemize*}
\end{definition}

\begin{definition}[3D Models]
  \begin{itemize*}
    \item Instantaneous camera motion models
    \item Homography + epipole
    \item Plane + Parallax
  \end{itemize*}
\end{definition}

\begin{definition}[Affine Motion]
  \(I_x(a_i + a_2x + a_3y) + I_y(a_4 + a_5x +a_6y) + I_t \approx 0\).
  Extension to planar perspective
\end{definition}

\begin{definition}[SSD tracking]
  For big displacements: match template against each pixel in small area around, match measure can be (normalized) correlation or SSD, choose max as match.
\end{definition}

\begin{definition}[Bayesian Optic Flow]
  Some low-level motion illusions can be explained by adding an underlying model to LK-tracking e.g. brightness constancy with noise.
\end{definition}
