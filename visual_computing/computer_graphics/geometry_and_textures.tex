\section{Geometry \& Textures}
\begin{definition}[Sources of Geometry]
  \begin{itemize*}
    \item 3D Scanning
    \item Point clouds
    \item 3D modeling
    \item procedural modeling
  \end{itemize*}
\end{definition}

\subsection{Geometric Representations}

\begin{definition}[Parametric Surfaces]
  i.e. \(f(x, y , z) = 0\).
  \begin{itemize*}
    \item implicit
    \item simple storage
    \item fst in/out test
    \item only works for easy surfaces
  \end{itemize*}
\end{definition}

\begin{definition}[Subdivision surfaces]
  \begin{itemize*}
    \item explicit
    \item surface piecewise linear
    \item increase sub.div. for more precision
  \end{itemize*}
\end{definition}

\begin{definition}[Point Set]
  \begin{itemize*}
    \item explicit
    \item collection of points
  \end{itemize*}
\end{definition}

\begin{definition}[Polygonal Meshes]
  \begin{itemize*}
    \item explicit
    \item boundary of objects
    \item geometry + connectivity
    \item attributes (color, normals, \(\sdots\))
    \item fast on GPU
    \item should support rendering, geometry queries, modifications
  \end{itemize*}
\end{definition}

\subsection{Mesh Datastructures}

\begin{definition}[Polygon]
  \(\langle V, E\rangle\), planar and non self-intersecting.
\end{definition}

\begin{definition}[Polygonal Mesh]
  \(M = \langle V, E, F\rangle\).
  \textit{Properties}:
  \begin{itemize*}
    \item every edge belongs to at least one polygon
    \item intersection of two polygons in \(M\) is either empty, a vertex, or an edge.
  \end{itemize*}
\end{definition}

\begin{definition}[Valence]
  Number of edges incident to a vertex
\end{definition}

\begin{definition}[Boundary]
  The set of all edges that belong to only 1 polygon.
\end{definition}

\begin{definition}[Manifold]
  Surface locally homeomorphic to a disk. Closed manifolds divide space into two.
\end{definition}

\begin{definition}[Triangle List]
  List of triangles.
  \begin{itemize*}
    \item simple
    \item no connectivity
    \item redundant
    \item STL file format
    \item easy queries
  \end{itemize*}
\end{definition}

\begin{definition}[Indexed Face Set]
  Two arrays, one describes the triangles with vertex indices and the index array contains the coordinates per vertex index.
  \begin{itemize*}
    \item avoids redundancy
    \item stores connectivity (costly geometric queries and mesh modifications)
    \item OBJ, OFF, WRL
  \end{itemize*}
\end{definition}

\subsection{Maps}

\begin{definition}[Texture Mapping]
  Increase detail without more geometry.

  \textit{Desirable}:
  \begin{itemize*}
    \item low distortion
    \item bijective
    \item efficient to compute
  \end{itemize*}

  \textit{Issues}:
  \begin{itemize*}
    \item texture atlases
    \item finding cuts
    \item anti-aliasing
    \item details
  \end{itemize*}
\end{definition}

\begin{theorem}
  Sphere Param:
  \((u, v)^\top \to (\sin u \sin v, \cos v, \cos u \sin v)^\top\)
\end{theorem}

\begin{theorem}
  Mapping textures could cause aliasing \(\to\)
  Use LPF, but isotropic Gauss in screen space is anisotropic in texture space.
  Use an anisotropic spatially-varying texture filter.
\end{theorem}

\begin{theorem}
  Textures can be procedurally generated via \textbf{Perlin noise} or \textbf{Gabor noise} (more control over spectral properties)
\end{theorem}

\begin{definition}[Solid Textures]
  Textures for each 3D coordinate.
\end{definition}

\begin{definition}[Light Map]
  Brightness map, similar to texture mapping.
\end{definition}

\begin{definition}[Environment Map]
  For reflective objects. Intersect ray with surrounding sphere/cube instead of scene. Really fast.
\end{definition}

\begin{definition}[Bump Mapping]
  small-scale geometry per fragment.
  Adapt normal to texture.
  Does not affect silhouette shadow.
  No self-occlusions and self-shadowing.
\end{definition}

\begin{definition}[Displacement Map]
  Per vertex, changes actual geometry.
\end{definition}

\begin{definition}[Normal Map]
  Each pixel has its own normal.
  For each pixel: \(n' = (2r - 1, 2g - 1, 2b - 1)^\top\).
\end{definition}

\begin{theorem}
  Normals are stored in texture space \((u, v, n)\) and thus need to be transformed into the object space \((N, T, B)\).
  Each point \(p = (x, y, z)^\top\) has linear relation \(A\) to its texture coordinate \(\vec{u} = (u, v)^\top\).
  For each triangle compute Tangent and Bitangent.
  \(T\) and \(B\) correspond to \((1, 0)\) and \((0, 1)\) in texture space:
  \[p = A \vec{u} \to \begin{bmatrix}
    \vec{T} & \vec{B}
  \end{bmatrix} = A I_{2 x 2} = A\]
  For triangle \(p_1, p_2, p_3\) got two conditions:
  \[p_2 = A \vec{u}_2 \ \land \ p_3 = A \vec{u}_3 \implies \begin{bmatrix}
    p_2 & p_3
  \end{bmatrix} = A \begin{bmatrix}
    \vec{u}_2 & \vec{u}_3
  \end{bmatrix}\]
  Solve for \(A\) and you got \(\begin{bmatrix}
    \vec{T} & \vec{B}
  \end{bmatrix}\).
\end{theorem}

\begin{definition}[Mip Map]
  Store texture with image pyramid. Choose res based on projected size of triangle. LERP in-between depths.
\end{definition}
