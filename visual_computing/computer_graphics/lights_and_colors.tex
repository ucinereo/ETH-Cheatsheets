\section{Light and Colors}

\begin{definition}[Light]
  Electromagnetic radiation (x-rays, micro-, radiowaves). Can be a mixture of many wavelengths.
\end{definition}

\begin{definition}[Color]
  \(P(\lambda)\) = intensity at wavelength \(\lambda\). \(P\) is the \textit{spectral power distribution} (SPD), which we perceive as color, by projecting this spectrum onto a lower-dimensional subspace.
\end{definition}

\subsection{Human Color Perception}

\begin{definition}[Cones]
    {\color{blue} S-short},
    {\color{green} M-medium},
    {\color{red} L-long} wavelength cones.
    \[P(\lambda) \to \left(\int\limits_{380nm}^{780nm}P(\lambda){\color{blue}S}(\lambda), \int\limits_{380nm}^{780nm}P(\lambda){\color{green}M}(\lambda), \int\limits_{380nm}^{780nm}P(\lambda){\color{red}L}(\lambda)\right)\]
    Thus project light into visible 3D-subspace.
\end{definition}

\begin{definition}[Metamers]
  Light with different spectrum map to same color.
\end{definition}

\subsection{CIE Primary System}

\begin{definition}[Primaries]
  Three \textit{single} frequency lights.
  \(\color{red} r(\lambda) = \delta(\lambda - 700)\), \(\color{green}g(\lambda) = \delta(\lambda - 546.1)\) and \(\color{blue}b(\lambda) = \delta(\lambda - 435.8)\).
\end{definition}

Experiment tried to find color matching functions for the mixture of these primaries to generate any reference light:
\[\delta(\lambda - v) \hat{=} \overline{r}(v)r(\lambda) + \overline{g}(v)g(\lambda) + \overline{b}(v) b(\lambda)\]

\(\overline{r}\), \(\overline{g}\), \(\overline{b}\) denote the color target matching functions. Thus we get:
\begin{align*}
  P(\lambda) &= \int P(v)\delta(\lambda - v)dv \\
  &= \left({\color{red}\int P(v)\overline{r}(v)dv}\right)\cdot r(\lambda) + {\color{green}G} \cdot g(\lambda) + {\color{blue}B} \cdot b(\lambda)
\end{align*}

We get negative values for \(R\)! Thus add red to reference light.

\subsection{CIE XYZ Color Space}
Basis transformation of RGB space to represent \textit{all perceptible} colors with new \textit{imaginary} primaries:
\(X = \int_{380}^{780} P(\lambda)\overline{x}(\lambda) d\lambda\), \(Y = \int_{380}^{780} P(\lambda)\overline{y}(\lambda) d\lambda\) and \(Z = \int_{380}^{780} P(\lambda)\overline{z}(\lambda) d\lambda\).

\begin{theorem}
  Value ranges: \(X \in [0, 1.2], Y \in [0, 1], Z \in [0, 1.6]\)
\end{theorem}

\begin{definition}[xyY Color Space]
  Normalization of XYZ space, where \((x, y)\) characterize chromaticity and \(Y\) characterizes brightness.
\end{definition}

\begin{theorem}
  Value ranges: \(x \in [0, 0.75], y \in [0, 0.86], Y\) see XYZ.
\end{theorem}

\pagebreak
\textbf{CIE Chromaticity Diagram}
\vspace{-10pt}
\begin{multicols}{2}
  \includegraphics*[width=\linewidth]{assets/chromaticity.png}
  
  \begin{definition}[Color Gamut]
     Linear combination of 3 colors in \(\triangle\).
  \end{definition}

  \begin{definition}[Purple Line]
    Non-spectral colors between 380 and 770.
  \end{definition}

  \begin{definition}[Dominant Wavelength]
    From color through whitepoint, boundary intersection.
  \end{definition}

  \begin{definition}[Saturation]
    Distance from color to white point.
  \end{definition}

  \begin{definition}[Isoline]
    Line with constant distance to border (w/o PL).
  \end{definition}
\end{multicols}

\vspace{-10pt}
\subsection{Other Color Spaces}

\begin{definition}[RGB]
  Based on three primaries. Used in monitors. Does not cover the whole XYZ space.
\end{definition}

\begin{theorem}
  Value ranges: \(R, G, B \in [0, 1]\)
\end{theorem}

\begin{definition}[CMY]
  Used in passive color systems (printers) and is the inverse to RGB.
  \textit{CMYK} adds black as color.
\end{definition}

\begin{definition}[YIQ]
  Advantages for natural and skin color. Y = Luminance, I = In-Phase (orange-blue) and Q = quadrature (purple-green) components.
\end{definition}

\begin{definition}[HSV]
  Used for interactive color picking.
  Dimensions no longer primaries:
  \textit{Hue} = base color, \textit{saturation} = purity of color and \textit{value/lightness/brightness}. Take RGB, CMY cubes and project to hexagon.
\end{definition}

\begin{theorem}
  Hue \(\in [0, 360]\), Saturation \& Value \(\in [0, 100]\)
\end{theorem}

\begin{definition}[CIELAB/CIELUV]
  Perceptually "uniform": color difference can be measured as distance in chart.
  \textbf{MaxAdams ellipses} become nearly circular.
\end{definition}

\subsection{Transformations}

\begin{definition}[RGB \(\to\) XYZ]
  \[\begin{bmatrix}
    \overline{x}(\lambda) \\ \overline{y}(\lambda) \\ \overline{z}(\lambda)
  \end{bmatrix} = \begin{bmatrix}
    2.36 & -0.515 & 0.005 \\ -0.89 & 1.426 & 0.014 \\ -0.46 & 0.088 & 1.009
  \end{bmatrix} \begin{bmatrix}
    \overline{r}(\lambda) \\ \overline{g}(\lambda) \\ \overline{b}(\lambda)
  \end{bmatrix}\]
\end{definition}

\begin{definition}[XYZ \(\to\) xyY]
  \[x = \frac{X}{X + Y + Z} \quad y = \frac{Y}{X + Y + Z} \quad Y = Y \quad \color{H3}X = \frac{xY}{y} \quad Z = \frac{(1 - x - y)Y}{y}\]
\end{definition}

\begin{definition}[RGB \(\to\) CMY]
  \(\begin{bmatrix}
    C \ M \ Y
  \end{bmatrix}^\top = \begin{bmatrix}
    1 \ 1 \ 1
  \end{bmatrix}^\top - \begin{bmatrix}
    R \ G \ B
  \end{bmatrix}^\top\)
\end{definition}

\begin{definition}[CMY \(\to\) CMYK]
  \(K = \min(C, M, Y), c \in \{C, M, Y\}: c' = c - K\)
\end{definition}

\begin{definition}[RGB \(\to\) YIQ]
  \[\begin{bmatrix}
    Y \\ I \\ Q
  \end{bmatrix} = \begin{bmatrix}
    0.299 & 0.587 & 0.114 \\ 0.596 & -0.275 & -0.321 \\ 0.212 & -0.523 & 0.311
  \end{bmatrix} \begin{bmatrix}
    R \\ G \\ B
  \end{bmatrix}\]
\end{definition}

\begin{definition}[RGB \(\to\) HSV] \
  % \vspace{-8pt}
  % \lstset{basicstyle=\ttfamily\footnotesize,breaklines=true}
  % \begin{center}
  %   \begin{lstlisting}
  % min = min(R, G, B)
  % max = max(R, G, B)
  % V = max;
  % If (max != 0) S = (max - min) / max
  % Else S = 0;
  % H = Hue (V, S, R, G, B); // proced.
  %   \end{lstlisting}
  % \end{center}
  \[C_{\max} = \max(R, G, B) \quad C_{\min} = (R, G, B) \quad \Delta = C_{\max} - C_{\min}\]

  \resizebox{\linewidth}{!}{
    \(
    H = \begin{cases}
        \frac{\pi}{3} \cdot \left(\frac{G - B}{\Delta} \mod 6 \right) & C_{\max} = R \\
        \frac{\pi}{3} \cdot \left(\frac{B - R}{\Delta}    + 2 \right) & C_{\max} = G \\
        \frac{\pi}{3} \cdot \left(\frac{R - G}{\Delta}    + 4 \right) & C_{\max} = R
    \end{cases}
    \
    \begin{array}{c}
    S = \begin{cases}
        0 & C_{\max} = 0 \\
        \frac{\Delta}{C_{\max}} & C_{\max} \neq 0
    \end{cases} \\
        V = C_{\max}
    \end{array}
    \)
  }

\end{definition}

\begin{definition}[HSV \(\to\) RGB]
  \[C = V \cdot S \qquad X = C \cdot \left(1 - \left|\frac{H}{\sfrac{\pi}{3}} mod 2 - 1\right|\right) \qquad m = V - C\]

  \resizebox{\linewidth}{!}{
    \(
    (R', G', B') = \begin{cases}(C, X, 0) & , 0^{\circ} \leq H< \sfrac{\pi}{3} \\ (X, C, 0) & , \sfrac{\pi}{3} \leq H< \sfrac{2\pi}{3} \\ (0, C, X) & , \sfrac{2\pi}{3} \leq H< \pi \\ (0, X, C) & , \pi \leq H< \sfrac{4\pi}{3} \\ (X, 0, C) & , \sfrac{4\pi}{3} \leq H<\sfrac{5\pi}{3} \\ (C, 0, X) & , \sfrac{5\pi}{3} \leq H<2 \pi\end{cases}
    \qquad
    \begin{array}{c}
        R = R' + m \\
        G = G' + m \\
        B = B' + m
    \end{array}
    \)
  }
\end{definition}

