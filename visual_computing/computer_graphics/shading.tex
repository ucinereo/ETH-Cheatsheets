\section{Shading}

\begin{definition}[Shading]
  Process of determining the color of a pixel.
\end{definition}

\begin{definition}[Flat Shading]
  One color per primitive. Screen space shading.
\end{definition}

\begin{theorem}
  Color is a per vertex attribute normally!
\end{theorem}

\begin{algorithm}[Gouraud Shading]
  Linear interpolation of vertex intensities. Screen space shading.
  \begin{enumerate}
    \item Calculate face normals
    \item Calculate vertex normal by averaging
    \item Evaluate illumination model for each vertex
    \item Interpolate vertex color bilinearly on current scan line.
  \end{enumerate}

  \textit{Problems}:
  \begin{itemize*}
    \item perspective distortion
    \item orientation dependence
    \item shared vertices
    \item quality depends on vertices
  \end{itemize*}
\end{algorithm}


\begin{algorithm}[Phong Shading]
  Barycentric interpolation of normal on the triangle. Thus color of fragment \(x\) is determined by interpolated normals.
  Object space shading.

  \includegraphics*[width=\linewidth]{assets/phong-shading.png}

  \(x = a \implies n_x = n_a\), \(\lambda_a + \lambda_b + \lambda_c = 1\), \(\lambda_a a + \lambda_b b + \lambda_c c = x\)

  \textit{Problems}:
  \begin{itemize*}
    \item Normal not defined/representative.
  \end{itemize*}
\end{algorithm}


\subsection{Transparency}

Use \textbf{RGBA} channels to calculate the overlapping objects.

\begin{algorithm}[Alpha Blending]
  Linearizes exponential attenuation of light intensity.
  \(I_\lambda = I_A \alpha_1 \Delta t + \alpha_2 I_Be^{-\alpha_1\Delta t} \approx I_A \alpha_1 + \alpha_2 I_B(1 - \alpha_1)\).

  Where object \(A\) is in front of \(B\) and has a thickness of \(\Delta t\).
\end{algorithm}

\begin{definition}[Back to front rendering]
  Sorted traversal of polygons from back to front.
  \textit{Problems}: overlapping objects.
\end{definition}

\begin{definition}[Depth peeling]
  Multiple passes which render the next closest fragment.
\end{definition}
