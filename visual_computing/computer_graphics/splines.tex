\section{Splines}

\begin{definition}[Local Coordinate System]
  \(u \in \R: x(u) = (x(u), y(u), z(u))^\top\)
  Thus map the 1D param space into 3D or 2D.
\end{definition}

\subsection{Bézier Curves}

\begin{center}
  \(x(t) = b_0B_0^n(t) + \ldots + b_n B_n^n(t)\) where \(b_i \in \R^3\)
\end{center}

\begin{itemize}
  \item \textbf{Affine invariance}: Affine transform of all points is accomplished by the affine transform of its control points.
  \item \textbf{Convex Hull Property}: The curve lies in the convex hull.
  \item Control polygon gives rough sketch of curve.
  \item \textbf{Endpoint interpolation}, since \(B_0^n(0) = B_n^n(1) = 1\)
  \item \textbf{Variation diminishing property}: Maximum number of intersections of a line with the curve is less or equal to the number of intersection with its control polygon.
  \color{red}
  \item \textbf{Global Support}: Control points influence whole curve.
  \item Insertion of new control point comes along with degree evaluation.
\end{itemize}

\begin{definition}[Control Point]
  Points defining the hull of the spline: \(b_i\).
\end{definition}

\begin{definition}[Bernsteinpolynomial]
  \(B_i^n(t) = \binom{n}{i} t^i(1 - t)^{n-1}\), \(i \notin [1, n]: 0\)

  \textit{Properties}:
  \begin{itemize*}
    \item Partition of unity (meaning at every point the sum of polynomials is 1)
    \item positive definite
    \item recursion
    \item symmetry
  \end{itemize*}
\end{definition}

\begin{definition}[Polynomial Degree]
  Highest exponent of the polynomial = number of control points.
\end{definition}

\begin{definition}[Polynomial Order]
  order = degree + 1
\end{definition}

\begin{theorem}
  Cubic Bézier curve has 4 control points!
\end{theorem}

\begin{algorithm}[DeCasteljau Algorithm]
  \lstset{
    basicstyle=\small\ttfamily,
    language=Pascal,
    captionpos=b
  }
  Interpolation in \(\O(n^2)\) \vspace{-5pt}
  \begin{lstlisting}[mathescape=true]
fun DeCasteljau(t, {b_0, b_1, ..., b_n}):
  for i from n to 1 do
    for j from 0 to i-1 do
      $b_j$ = (1 - t) * $b_j$ + t * $b_{j+1}$
  return $b_0$ // Final interpolated point
\end{lstlisting}
Basically interpolate, till you no longer can interpolate. If you write \(b_0\) out and factor out the original control points, you get the sum of Bernsteinpolynomials again.
\end{algorithm}

\begin{theorem}
  Derivative is another Bézier of order \(n-1\).
\end{theorem}

\begin{theorem}
  Combining Bézier curves makes the whole curve \(C^0\).
\end{theorem}

\begin{definition}[Piecewise Bézier Curves]
  Two composed Bézier curves (e.g. segments \(b_0, \ldots b_n\) in \([u_0, u_1]\) and \(b_n, \ldots b_{2n}\) in \([u_1, u_2]\)) need the following constraint to enforce \(C^r\)-continuity:
  \(b_{n+i} = b_{n-i}^i(t)\) for \(i = 0, \ldots, r\) where \(t = (u - u_0) / (u_1 - u_0)\) stands for the local coordinate of \(u_2\) relative to \(u_0, u_1\).
\end{definition}

\begin{definition}[Joining Cubic Béziers]
  Curve \(A \cup B\) is \(C^n\) continuous if segment \(A^{(i)}(t_{end}) = B^{(i)}(t_{start})\) for \(i = 0, \ldots n\).
\end{definition}

\begin{theorem}
  Thus for \(C^1\), we need \(b_{n-1}\) to be mirrored on \(b_n\) by \(b_{n+1}\).
\end{theorem}

\begin{definition}[Matrix Form]
  We can rewrite the curve equation with vectors:
  \(x(t) = \sum_i^n b_i B_i^n(t) = (b_0, \sdots, b_n)(B_0^n(t), \sdots, B_n^n(t))^\top\).
  Then we use a basis transformation to get a monomial representation:
  \[\begin{bmatrix}
    B_0^n(t) \\ \vdots \\ B_n^n(t)
  \end{bmatrix} = \begin{bmatrix}
    m_{00} & \ldots & m_{0n} \\ \vdots & & \vdots \\ m_{m0} & \ldots & m_{mn}
  \end{bmatrix} \begin{bmatrix}
    t^0 \\ \vdots \\ t^n
  \end{bmatrix}\]
\end{definition}

\begin{theorem}
  For Bernstein polynomials we get  \(m_{ij} = (-1)^{j-i}\binom{n}{j} \binom{j}{i}\).
\end{theorem}

\begin{definition}[Spline Interpolation]
  Interpolate \(\{p_0, \ldots, p_n\}\) with \(x(t_i) = p_i\).
  
  \begin{center}
    \(\begin{bmatrix}
      1 & t_0 & \ldots & t_0^n \\ \vdots & \vdots & & \vdots \\ 1 & t_n & \ldots & t_n^n
    \end{bmatrix} \begin{bmatrix}
      a_0 \\ \vdots \\ a_n
    \end{bmatrix} = \begin{bmatrix}
      p_0 \\ \vdots \\p_n
    \end{bmatrix}\)
  \end{center}

  Where \(a_i\) are the polynomial coefficients.
\end{definition}

\subsection{B-Spline Curves}

\begin{center}
  \(s(u) = \sum_{i=0}^n d_i N_i^n(u)\) where \(d_i \in \R^3\)
\end{center}

Coefficients \(d_i\) are \textit{de Boor} points. Bases are piecewise, recursively defined polynomials over a sequence of knots defined in the knot vector \(T = (u_0, \ldots, u_{k+n+1})\).

\begin{itemize}
  \item Different degrees
  \item Piecewise polynomial
  \item Local support (de Boor points only have local influence)
  \item uniform/non-uniform (distance from \(u_i\) to \(u_{i+1}\))
\end{itemize}

\begin{definition}[B-Spline Functions]
  \[N_i^n(u) = (u - u_i)\frac{N_i^{n-1}(u)}{u_{i+n} - u_i} + (u_{i+n+1} - u)\frac{N_{i+1}^{n-1}(u)}{u_{i+n+1} - u_{i+1}}\]
  where \(N_i^0(u) = \begin{cases}
    1 & u \in [u_i, u_{i+1}] \\ 0 & \text{otw.}
  \end{cases}\)

  \textit{Properties}:
  \begin{itemize*}
    \item Partition of unity
    \item Positivity (\(N_i^n(u) \geq 0\))
    \item Compact (\(N_i^n(u) = 0, u \notin [u_i, u_{i+1}]\))
    \item Continuity (\(N_i^n\) is \(C^{n-1}\)).
  \end{itemize*}
\end{definition}

\begin{definition}[B-Spline filters]
  Cardinal B-Splines over uniform knot sequences can be computed using the convolution:
  \[N_i^n = N^{n-1} \ast N^0 \quad N^0: \text{box} - \text{function}\]
\end{definition}

\begin{theorem}
  Curve is globally \(C^{n-1}\) except multiple knots of order \(p\) with \(u_j = \ldots u_{j+p-1}\) lead to \(C^{n-p}\).
\end{theorem}

\begin{algorithm}[deBoor Algorithm]
  Successive linear interpolation:
  \[d_i^k = (1 - a_i^k)d_{i-1}^{k-1} + a_i^k d_i^{k-1} \quad a_i^k = \frac{t - u_i}{u_{i+n+1-k} - u_i}\]
  where \(d_i^0 = d_i\) and \(d_n^n = s(t)\).
\end{algorithm}

\begin{theorem}
  With first and last knot multiplicity of \(n+1\) we have that the deBoor Algorithm = DeCasteljau Algorithm.
\end{theorem}

\begin{definition}[Spline Interpolation]
  System is under-determined. Need \(k+n\) control points to interpolate \(k+1\) points. \(k-1\) define the interior intervals and \(n+1\) the boundaries.
\end{definition}

\subsection{Tensor Product Surfaces}
From curve to a surface involves taking a tensor product.
\[x(u, v) = \sum_{i=0}^{n}\sum_{j=0}^{m}\alpha_{i, j} F_i(u)G_j(v)\]

\begin{definition}[Bézier Patches]
  \(F_i = B_i^m\) and \(G_j = B_i^n\).

  \textit{Properties}:
  \begin{itemize*}
    \item Affine invariance
    \item convex hull property
    \item variation diminishing property
    \item boundary curves (patch boundaries are Bézier curves)
  \end{itemize*}
\end{definition}

\begin{algorithm}[2D DeCasteljau]
  There exists the 2D DeCasteljau algorithm.
\end{algorithm}

\begin{definition}[Warping as a 2D Parametric Function]
  Given a matrix of vector valued landmark points:
  \[m(u_i, v_j) = \begin{bmatrix}
    B_0(u_i) & \ldots & B_n(u_i)
  \end{bmatrix} \begin{bmatrix}
    b_{00} & \ldots & b_{0m} \\ \vdots && \vdots \\ b_{n0} & \ldots & b_{nm}
  \end{bmatrix} \begin{bmatrix}
    B_0(v_j) \\ \vdots \\ B_m(v_j)
  \end{bmatrix}\]
  Solve this interpolation problem.
\end{definition}

\begin{definition}[Matrix Form]
  Same as in 1D. Rewrite in monomial basis:
  \[b^{m, n}(u, v) = (u^0, \ldots, u^m)M^\top[b_{ij}]_{i\in[m], j\in[n]}N (v^0, \ldots, v^n)^\top\]
  With \(m_{ij} = (-1)^{j - i} \binom{m}{j}\binom{j}{i}\) and \(n_{ij} = (-1)^{j - i} \binom{n}{j}\binom{j}{i}\)
\end{definition}

\begin{definition}[B-Spline Patches]
  \(F_i = N_i^n\) and \(G_j = M_j^m\).
\end{definition}

\begin{algorithm}[2D DeBoor Algorithm]
  It exists :)
\end{algorithm}

\begin{definition}[Rational B-Spline Patches (Nurbs)]
  Most advanced modelling and animation systems are based on NURBS. They are \textbf{not} tensor product surfaces, since bases are non-separable.
  \[s(u,v) = \frac{\sum_i \sum_j w_{i, j} d_{i, j}N_i^m(u)N_j^n(v)}{\sum_i \sum_j w_{i, j}N_i^m(u)N_j^n(v)}\]
\end{definition}
