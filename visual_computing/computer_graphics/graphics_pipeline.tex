\section{Graphics Pipeline}

\begin{definition}[Input]
  \begin{itemize*}
    \item Geometry representation of primitives
    \item material and lighting models
    \item virtual camera
  \end{itemize*}
\end{definition}

\begin{definition}[The pipeline]
  \begin{enumerate}
    \item \textit{Modeling Transform}: Object to world space
    \item \textit{Viewing Transform}: World to camera space
    \item \textit{Primitive Processing}: Output from transformed vertices
    \item \textit{3D Clipping}: Remove parts of primitives outside frustum
    \item \textit{Projection to Screen Space}: 3D to 2D screen space
    \item \textit{Scan Conversion}: Discretize primitives and interpolate
    \item \textit{Lighting, Shading, Texturing}: Compute color
    \item \textit{Occlusion Handling}: depth handling via \(z\)-buffer.
    \item \textit{Display}: Display the final result
  \end{enumerate}
\end{definition}

\subsection{Programmers View}

\begin{center}
  CPU \(\to\) \textit{Vertex Proc.} \(\to\) Rast. \(\to\) \textit{Fragment Proc.} \(\to\) Display
\end{center}

\begin{definition}[Vertex Processing]
  Per-vertex operations: transform, lighting, flow control. Programmed in \textit{vertex shader}.
\end{definition}

\begin{definition}[Fragment Processing]
  Per-fragment operations: shading, texturing, blending. Programmed in \textit{fragment shader}.
\end{definition}


\begin{tikzpicture}[
	node distance = 1cm, on grid,
	mvertex/.style = {draw, rectangle, align=center, draw=H2, color=white, fill=H3, thick},
	vertex/.style = {draw, rectangle, align=center, draw=H1, thick},
  every edge/.style={draw, ->, auto, semithick}
]
  % Define vertices
  \node[vertex] at (0, 0) (attr) {Attributes};
  \node[mvertex] at (1, -0.6) (vs) {Vertex Shader};
  \node[vertex] at (1.5, 0.2) (vvar) {Varying \\ (per vertex)};
  \node[mvertex] at (2.8, -0.6) (inter) {Interpolation};
  \node[vertex] at (3.3, 0.2) (fvar) {Varying \\ (per fragment)};
  \node[mvertex] at (4.8, -0.6) (fs) {Fragment Shader};
  \node[vertex] at (5.3, 0) (fc) {Fragment Color};
  \node[vertex] at (2.8, -1.2) (unif) {Uniforms (constants)};

  \draw (unif) edge (vs);
  \draw (unif) edge (fs);
  \draw (attr) edge (vs);
  \draw (vs) edge (vvar);
  \draw (vvar) edge (inter);
  \draw (inter) edge (fvar);
  \draw (fvar) edge (fs);
  \draw (fs) edge (fc);
\end{tikzpicture}

\begin{definition}[Fragment]
  Fragment \(\neq\) Pixel! A fragment can store:
  \begin{itemize*}
    \item color
    \item position
    \item depth
    \item depth
    \item texture coordinates
    \item window id
    \item \(\sdots\)
  \end{itemize*}
\end{definition}

\begin{definition}[Graphics API]
  Access to hardware with universal interface.
\end{definition}
